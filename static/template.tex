\documentclass[a4paper,10pt]{article}
\usepackage[utf8]{inputenc}
\usepackage{latexsym, 
			amsfonts, 
			amssymb, 
			amsthm, 
			calc,  %#			enumerate, 
			fontawesome5,
			graphicx,
			longtable, 
            makecell,
            ngerman, %# Deutsche Namen Inhaltsverzeichnis etc
			nicefrac, %# for \nicefrac{1}{2}
			paralist,    %# for compactenum
            pifont,
            rotating,
            tabularx,
            xcolor,
			}
\usepackage[hidelinks,linkbordercolor={1 1 1}]{hyperref}
\usepackage{pifont} 
\newcommand{\xmark}{\ding{55}}
\usepackage{draftwatermark}
\SetWatermarkText{\VAR{data['wasserzeichen']}}
\SetWatermarkScale{1}

\setlength{\topmargin}{-2.5cm}
\setlength{\oddsidemargin}{-1cm}
\setlength{\textwidth}{18cm}
\setlength{\textheight}{26.5cm}
\parindent0em
\parskip1ex

\begin{document}

\hrule\vskip1pt\hrule\medskip

\resizebox{\textwidth}{!}{
%% if data["lang"] == "de"
Universität Freiburg -- Mathematisches Institut
%% else 
University of Freiburg -- Mathematical Institute
%% endif
}

\medskip
\resizebox{\textwidth}{!}{\VAR{data["semester"]}}

\bigskip
\resizebox{\textwidth}{!}{\VAR{data["titel"]}}

\medskip\hrule\vskip1pt\hrule

\bigskip
\bigskip

%#\setlength{\baselineskip}{20pt}
\hfill Version \today

\thispagestyle{empty}
\clearpage
\tableofcontents

\clearpage
\addcontentsline{toc}{section}{{Hinweise}}\addtocontents{toc}{\medskip\hrule\medskip}

%#%%%%%%%%%%%%%%%%%%%%%%%%%%%%%%%%%%%%%%%%%%%%%%%%%%%%%%%%%%%%%%%%%%%%%%%%%%%%%%%%
%#%% Über dieses Dokument
%#%%%%%%%%%%%%%%%%%%%%%%%%%%%%%%%%%%%%%%%%%%%%%%%%%%%%%%%%%%%%%%%%%%%%%%%%%%%%%%%%

\section*{Über dieses Dokument}

\addcontentsline{toc}{subsection}{Über dieses Dokument}
Dies ist ein automatisch generiertes Dokument aus dem VVZ.

%#%%%%%%%%%%%%%%%%%%%%%%%%%%%%%%%%%%%%%%%%%%%%%%%%%%%%%%%%%%%%%%%%%%%%%%%%%%%%%%%%
%#%%  BEGINN VERANSTALTUNGSTEIL
%#%%%%%%%%%%%%%%%%%%%%%%%%%%%%%%%%%%%%%%%%%%%%%%%%%%%%%%%%%%%%%%%%%%%%%%%%%%%%%%%%

%% for rubrik in data["rubriken"]
\clearpage
\phantomsection
\thispagestyle{empty}
\vspace*{\fill}
\begin{center}
\Huge\bfseries \VAR{rubrik['titel']}
\end{center}
\addcontentsline{toc}{section}{\textbf{\VAR{rubrik['titel']}}}
\addtocontents{toc}{\medskip\hrule\medskip}\vspace*{\fill}\vspace*{\fill}\clearpage
\vfill
\thispagestyle{empty}
\clearpage

%% for veranstaltung in rubrik["veranstaltung"]
\clearpage\hrule\vskip1pt\hrule 
%% if veranstaltung["url"] != ""
\section*{\Large \href{\VAR{veranstaltung["url"]}}{\VAR{veranstaltung["titel"]}}}
%% else 
\section*{\Large \VAR{veranstaltung["titel"]}}
%% endif
\addcontentsline{toc}{subsection}{\VAR{veranstaltung["titel"]}\ \textcolor{gray}{(\em \VAR{veranstaltung["dozent"]})}}
\vskip-2ex  
\VAR{veranstaltung["person"]}\\
%% for line in veranstaltung["raumzeit"]
\VAR{line}\\
%% endfor 
%% if veranstaltung["url"] != ""
Webseite: \url{\VAR{veranstaltung["url"]}}
%% endif

%% if data["include_inhalt"]
%% if veranstaltung["inhalt"] != ""
\subsubsection*{\Large 
%% if data["lang"] == "de"
Inhalt:
%% else
Content:
%% endif
}
\VAR{veranstaltung["inhalt"]}
%% endif 
%% if veranstaltung["literatur"] != ""
\subsubsection*{\Large 
%% if data["lang"] == "de"
Literatur:
%% else
Literature:
%% endif
}
\VAR{veranstaltung["literatur"]}
%% endif 
%% if veranstaltung["vorkenntnisse"] != ""
\subsubsection*{\Large 
%% if data["lang"] == "de"
Vorkenntnisse:
%% else 
Prerequisites:
%% endif
}
\VAR{veranstaltung["vorkenntnisse"]}
%% endif
%% if veranstaltung["kommentar"] != ""
\subsubsection*{\Large 
%% if data["lang"] == "de"
Bemerkungen:
%% else 
Remarks:
%% endif 
}
\VAR{veranstaltung["kommentar"]}
%% endif 
%% endif
%% if data["verw_kurz"]
\subsubsection*{\Large 
%% if data["lang"] == "de"
Verwendbar in folgenden Modulen:
%% else
Usable in the following modules:
%% endif 
}
%% for m in veranstaltung["verwendbarkeit_modul"]
\VAR{m["titel"]}\\
%% endfor 
%% endif 

%% if not data["verw_kurz"]
\subsubsection*{\Large 
%% if data["lang"] == "de"
Verwendbarkeit, Studien- und Prüfungsleistungen:
%% else 
Usability and assessments:
%% endif 
}
\begin{tabularx}{\textwidth}{ p{.5\textwidth}
    %% for i in veranstaltung["verwendbarkeit_modul"]
    X
    %% endfor
    }
    %% for y in veranstaltung["verwendbarkeit_modul"]
    & 
    \makecell[c]{\rotatebox[origin=l]{90}{\parbox{
    %% if veranstaltung["verwendbarkeit_modul"] | length > 3 
    8
    %% else 
    4
    %%endif
        cm}{\begin{flushleft}
        \VAR{y["titel"]}
    \end{flushleft} }}} 
    %% endfor
    \\[2ex] \hline 
    %% for x in veranstaltung["verwendbarkeit_anforderung"]
    \rule[0mm]{0cm}{.6cm}\VAR{x["titel"]} \rule[-3mm]{0cm}{0cm}
    %% for y in veranstaltung["verwendbarkeit_modul"]    
    &
    %% if {"modul": y["id"], "anforderung": x["id"]} in veranstaltung["verwendbarkeit"]
    \makecell[c]{\xmark}
    %% endif 
    %% endfor
    \\
    %% endfor
\end{tabularx}
%% endif 
%% endfor
%% endfor

\end{document}
