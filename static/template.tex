\documentclass[a4paper,10pt]{article}
\usepackage[utf8]{inputenc}
\usepackage{latexsym, 
			amsfonts, 
			amssymb, 
			amsthm, 
			calc,  %#			enumerate, 
			%fontawesome5,
			graphicx,
			longtable, 
            makecell,
			nicefrac, %# for \nicefrac{1}{2}
			paralist,    %# for compactenum
            pifont,
            rotating,
            tabularx,
            colortbl
			}
\usepackage[table]{xcolor}
\arrayrulecolor{lightgray} % Ändert die Farbe der Linien
% \arrayrulewidth=1pt   % Setzt die Dicke der Linien

%% if data["lang"] == "de"
\usepackage{ngerman}
%% endif
\usepackage{hyperref}
\hypersetup{
  colorlinks = true, %Colours links instead of ugly boxes
  urlcolor = blue, %Colour for external hyperlinks
  linkcolor = blue, %Colour of internal links
  citecolor = blue %Colour of citations
}
\frenchspacing

\newcommand{\mailto}[1]{\href{mailto:#1}{#1}}
\renewenvironment{itemize}{\begin{list}{$\bullet$\ }{\itemsep.5ex\setlength{\topsep}{0.5\itemsep}\parsep0ex\labelsep1ex\settowidth{\labelwidth}{$\bullet$\ }\setlength{\leftmargin}{\labelwidth}\addtolength{\leftmargin}{3ex}\addtolength{\leftmargin}{\labelsep}}}{\end{list}} 
\usepackage{pifont} 
\newcommand{\xmark}{\ding{55}}
\usepackage{draftwatermark}
\SetWatermarkText{\VAR{data['wasserzeichen']}}
\SetWatermarkScale{1}

\setlength{\topmargin}{-2.5cm}
\setlength{\oddsidemargin}{-1cm}
\setlength{\textwidth}{18cm}
\setlength{\textheight}{26.5cm}
\parindent0em
\parskip1ex

\begin{document}

\hrule\vskip1pt\hrule\medskip

\resizebox{\textwidth}{!}{
    %% if data["lang"] == "de"
    Universität Freiburg -- Mathematisches Institut
    %% else 
    University of Freiburg -- Mathematical Institute
    %% endif
}

\medskip
\resizebox{\textwidth}{!}{\VAR{data["semester"]}}

\bigskip
\resizebox{\textwidth}{!}{\VAR{data["titel"]}}

\medskip\hrule\vskip1pt\hrule

\bigskip
\bigskip

%#\setlength{\baselineskip}{20pt}
\hfill Version \today

\thispagestyle{empty}
\clearpage
\tableofcontents

\clearpage
%% if data["lang"] == "de"
\addcontentsline{toc}{section}{{Hinweise}}\addtocontents{toc}{\medskip\hrule\medskip}
%% else
\addcontentsline{toc}{section}{{Comments}}\addtocontents{toc}{\medskip\hrule\medskip}
%% endif

%#%%%%%%%%%%%%%%%%%%%%%%%%%%%%%%%%%%%%%%%%%%%%%%%%%%%%%%%%%%%%%%%%%%%%%%%%%%%%%%%%
%#%% Input eines zusätzlichen Files
%#%%%%%%%%%%%%%%%%%%%%%%%%%%%%%%%%%%%%%%%%%%%%%%%%%%%%%%%%%%%%%%%%%%%%%%%%%%%%%%%%

\VAR{data["vorspann"]}

%#%%%%%%%%%%%%%%%%%%%%%%%%%%%%%%%%%%%%%%%%%%%%%%%%%%%%%%%%%%%%%%%%%%%%%%%%%%%%%%%%
%#%%  BEGINN VERANSTALTUNGSTEIL
%#%%%%%%%%%%%%%%%%%%%%%%%%%%%%%%%%%%%%%%%%%%%%%%%%%%%%%%%%%%%%%%%%%%%%%%%%%%%%%%%%

%% for rubrik in data["rubriken"]
\clearpage
\phantomsection
\thispagestyle{empty}
\vspace*{\fill}
\begin{center}
    \Huge\bfseries \VAR{rubrik['titel']}
\end{center}
\addcontentsline{toc}{section}{\textbf{\VAR{rubrik['titel']}}}
\addtocontents{toc}{\medskip\hrule\medskip}\vspace*{\fill}\vspace*{\fill}\clearpage
\vfill
\thispagestyle{empty}
\clearpage

%% for veranstaltung in rubrik["veranstaltung"]
\clearpage\hrule\vskip1pt\hrule
%% if veranstaltung["url"] != ""
\section*{\Large \href{\VAR{veranstaltung["url"]}}{\VAR{veranstaltung["titel"]}}}
%% else 
\section*{\Large \VAR{veranstaltung["titel"]}}
%% endif
\addcontentsline{toc}{subsection}{\VAR{veranstaltung["titel"]}\ \textcolor{gray}{(\em \VAR{veranstaltung["dozent"]})}}
\vskip-2ex
{\it \VAR{veranstaltung["person"]}}
%% if veranstaltung["code"] != ""
\hfill{\VAR{veranstaltung["code"]}}\\
%% endif
%% for line in veranstaltung["raumzeit"]
\VAR{line}\\
%% endfor 
%% if veranstaltung["url"] != ""
% Webseite: \url{\VAR{veranstaltung["url"]}}
%% endif

%% if data["include_inhalt"]
%% if veranstaltung["inhalt"] != ""
\subsubsection*{\large
    %% if data["lang"] == "de"
    Inhalt:
    %% else
    Content:
    %% endif
}
\VAR{veranstaltung["inhalt"]}
%% endif 
%% if veranstaltung["literatur"] != ""
\subsubsection*{\large
    %% if data["lang"] == "de"
    Literatur:
    %% else
    Literature:
    %% endif
}
\VAR{veranstaltung["literatur"]}
%% endif 
%% if veranstaltung["vorkenntnisse"] != ""
\subsubsection*{\large
    %% if data["lang"] == "de"
    Vorkenntnisse:
    %% else 
    Prerequisites:
    %% endif
}
\VAR{veranstaltung["vorkenntnisse"]}
%% endif
%% if veranstaltung["kommentar"] != ""
\subsubsection*{\large
    %% if data["lang"] == "de"
    Bemerkungen:
    %% else 
    Remarks:
    %% endif 
}
\VAR{veranstaltung["kommentar"]}
%% endif 
%% endif
%% if data["verw_kurz"]
\subsubsection*{\large
    %% if data["lang"] == "de"
    Verwendbar in folgenden Modulen:
    %% else
    Usable in the following modules:
    %% endif 
}
%% for m in veranstaltung["verwendbarkeit_modul"]
\VAR{m["titel"]}\\
%% endfor 
%% endif 

%% if not data["verw_kurz"]
\cleardoublepage
\subsubsection*{\large
    %% if data["lang"] == "de"
    Verwendbarkeit, Studien- und Prüfungsleistungen:
    %% else 
    Usability and assessments:
    %% endif 
}

\begin{tabularx}{\textwidth}{ X
    %% for i in range(veranstaltung["verwendbarkeit"].shape[1])
    |c
    %% endfor
}
%% for y in veranstaltung["verwendbarkeit"].columns
 &
\makecell[c]{\rotatebox[origin=l]{90}{\parbox{
            %% if veranstaltung["verwendbarkeit"].columns | length < 2
            5
            %% else 
            7
            %%endif
            cm}{\raggedright
                \begin{itemize}\item
                    \VAR{y} 
                \end{itemize}             }}}
%% endfor
\\[2ex] \hline
%% for index, row in veranstaltung["verwendbarkeit"].iterrows()
%% if index.split(" ")[0] != "Kommentar:"
\hline \rule[0mm]{0cm}{.6cm}\VAR{index} \rule[-3mm]{0cm}{0cm}
%% for y in veranstaltung["verwendbarkeit"].columns
 &
%% if row[y] 
\makecell[c]{\xmark}
%% endif 
%% endfor
\\
%% endif 
%% endfor
\hline
%% if veranstaltung["verwendbarkeit_hat_kommentar"]
%% for y in veranstaltung["verwendbarkeit"].columns
& \makecell[c]{\vphantom{$\displaystyle\int$}\ding{\VAR{veranstaltung["verwendbarkeit"].columns.get_loc(y)+172}}}
%% endfor
%% endif
\\
\end{tabularx}

\medskip

%% for index, row in veranstaltung["verwendbarkeit"].iterrows()
    %% if index.split(" ")[0] == "Kommentar:"
        %% for y in veranstaltung["verwendbarkeit"].columns
            %% if row[y] 
            \ding{\VAR{veranstaltung["verwendbarkeit"].columns.get_loc(y)+172}}
            %% endif 
        %% endfor
        \VAR{index[11:]} 

    %% endif 
%% endfor

\VAR{veranstaltung["kommentar_verwendbarkeit"]}

%% endif 
%% endfor
%% endfor

\end{document}
