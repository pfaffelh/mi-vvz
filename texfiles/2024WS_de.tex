\documentclass[a4paper,10pt]{article}
\usepackage[utf8]{inputenc}
\usepackage{latexsym, 
			amsfonts, 
			amssymb, 
			amsthm, 
			calc,
			%fontawesome5,
			graphicx,
			longtable, 
            makecell,
			nicefrac,
			paralist,
            pifont,
            rotating,
            tabularx,
            colortbl
			}
\usepackage[table]{xcolor}
\arrayrulecolor{lightgray} % Ändert die Farbe der Linien
% \arrayrulewidth=1pt   % Setzt die Dicke der Linien

\usepackage{ngerman}
\usepackage{hyperref}
\hypersetup{
  colorlinks = true, %Colours links instead of ugly boxes
  urlcolor = blue, %Colour for external hyperlinks
  linkcolor = blue, %Colour of internal links
  citecolor = blue %Colour of citations
}
\frenchspacing

\newcommand{\mailto}[1]{\href{mailto:#1}{#1}}
\renewenvironment{itemize}{\begin{list}{$\bullet$\ }{\itemsep.5ex\setlength{\topsep}{0.5\itemsep}\parsep0ex\labelsep1ex\settowidth{\labelwidth}{$\bullet$\ }\setlength{\leftmargin}{\labelwidth}\addtolength{\leftmargin}{3ex}\addtolength{\leftmargin}{\labelsep}}}{\end{list}} 
\usepackage{pifont} 
\newcommand{\xmark}{\ding{55}}
\usepackage{draftwatermark}
\SetWatermarkText{}
\SetWatermarkScale{1}

\setlength{\topmargin}{-2.5cm}
\setlength{\oddsidemargin}{-1cm}
\setlength{\textwidth}{18cm}
\setlength{\textheight}{26.5cm}
\parindent0em
\parskip1ex

\begin{document}

\hrule\vskip1pt\hrule\medskip

\resizebox{\textwidth}{!}{
    Universität Freiburg -- Mathematisches Institut
}

\medskip
\resizebox{\textwidth}{!}{Wintersemester 2024/25}

\bigskip
\resizebox{\textwidth}{!}{Ergänzungen des Modulhandbuchs}

\medskip\hrule\vskip1pt\hrule

\bigskip
\bigskip


\hfill Version \today

\thispagestyle{empty}
\clearpage
\tableofcontents

\clearpage
\addcontentsline{toc}{section}{{Hinweise}}\addtocontents{toc}{\medskip\hrule\medskip}











\clearpage
\phantomsection
\thispagestyle{empty}
\vspace*{\fill}
\begin{center}
    \Huge\bfseries 1a. Einführende Pflichtvorlesungen der verschiedenen Studiengänge
\end{center}
\addcontentsline{toc}{section}{\textbf{1a. Einführende Pflichtvorlesungen der verschiedenen Studiengänge}}
\addtocontents{toc}{\medskip\hrule\medskip}\vspace*{\fill}\vspace*{\fill}\clearpage
\vfill
\thispagestyle{empty}
\clearpage

\clearpage\hrule\vskip1pt\hrule
\section*{\Large \href{https://aam.uni-freiburg.de/agru/lehre/ws24/ana1/index.html}{Analysis I}}
\addcontentsline{toc}{subsection}{Analysis I\ \textcolor{gray}{(\em Michael Růžička)}}
\vskip-2ex
{\it Michael Růžička, Assistenz: Alexei Gazca}
\hfill{D, 9.0 ECTS}\\
Vorlesung: Di, Mi, 8--10 Uhr, HS Rundbau, \href{https://www.openstreetmap.org/?mlat=48.001613&mlon=7.849254#map=19/48.001613/7.849254}{Albertstr. 21}\\
Übung: 2-stündig, verschiedene Termine \\
% Webseite: \url{https://aam.uni-freiburg.de/agru/lehre/ws24/ana1/index.html}
\subsubsection*{\large
    Inhalt:
}
Analysis I ist eine der beiden Grundvorlesungen des Mathematikstudiums. Es werden Konzepte behandelt, die auf dem Begriff des Grenzwerts beruhen. Die zentralen Themen sind: vollständige Induktion, reelle und komplexe Zahlen, Konvergenz von Folgen und Reihen, Vollständigkeit, Exponentialfunktion und trigonometrische Funktionen, Stetigkeit, Ableitung von Funktionen einer Variablen, Regelintegral.
\subsubsection*{\large
    Literatur:
}
Wird in der Vorlesung bekanngegeben.
\subsubsection*{\large
    Vorkenntnisse:
}
Oberstufenmathematik.  \\
Der Besuch des Vorkurses wird empfohlen.
\subsubsection*{\large
    Bemerkungen:
}
This course is only offered in German.
\subsubsection*{\large
    Verwendbarkeit, Studien- und Prüfungsleistungen:
}

\begin{tabularx}{\textwidth}{ p{.5\textwidth}
    |X
    |X
}
 &
\makecell[c]{\rotatebox[origin=l]{90}{\parbox{
            4
            cm}{\begin{flushleft}
                Analysis (2HfB21, BSc21, MEH21, MEB21) (9.0 ECTS)
            \end{flushleft} }}}
 &
\makecell[c]{\rotatebox[origin=l]{90}{\parbox{
            4
            cm}{\begin{flushleft}
                Analysis I (fachfremd) (BScInfo19, BScPhys20) (9.0 ECTS)
            \end{flushleft} }}}
\\
& \Var{veranstaltung["verwendbarkeit"].columns.index(y)}
& \Var{veranstaltung["verwendbarkeit"].columns.index(y)}
\\[2ex] \hline
\hline \rule[0mm]{0cm}{.6cm}PL: Mündliche Prüfung (Dauer ca. 30 Minuten) über Analysis I und II am Ende des Moduls. (Die bestandene Klausur zu Analysis I und die bestandene Übung zu Analysis II sind Zulassungsvoraussetzungen). \rule[-3mm]{0cm}{0cm}
 &
\makecell[c]{\xmark}
 &
\\
\hline \rule[0mm]{0cm}{.6cm}SL: Bestehen der Abschlussklausur (Dauer 1 bis 3 Stunden). \rule[-3mm]{0cm}{0cm}
 &
\makecell[c]{\xmark}
 &
\makecell[c]{\xmark}
\\
\hline \rule[0mm]{0cm}{.6cm}SL: Erreichen von mindestens 50\% der Punkte, die insgesamt durch die Bearbeitung der für die Übung ausgegebenen Übungsaufgaben erreicht werden können. \rule[-3mm]{0cm}{0cm}
 &
\makecell[c]{\xmark}
 &
\makecell[c]{\xmark}
\\
\hline \rule[0mm]{0cm}{.6cm}SL: Regelmäßige Teilnahme am Tutorat wie in der Prüfungsordnung definiert. \rule[-3mm]{0cm}{0cm}
 &
\makecell[c]{\xmark}
 &
\makecell[c]{\xmark}
\\
\end{tabularx}




\clearpage\hrule\vskip1pt\hrule
\section*{\Large Lineare Algebra I}
\addcontentsline{toc}{subsection}{Lineare Algebra I\ \textcolor{gray}{(\em Stefan Kebekus)}}
\vskip-2ex
{\it Stefan Kebekus, Assistenz: Marius Amann}
\hfill{D, 9.0 ECTS}\\
Vorlesung: Mo, Do, 8--10 Uhr, HS Rundbau, \href{https://www.openstreetmap.org/?mlat=48.001613&mlon=7.849254#map=19/48.001613/7.849254}{Albertstr. 21}\\
Übung: 2-stündig, verschiedene Termine \\
\subsubsection*{\large
    Inhalt:
}
Lineare Algebra I ist eine der beiden Einstiegsvorlesungen des Mathematikstudiums, die die Grundlage für weiteren Veranstaltungen bilden. Behandelt werden u.a: Grundbegriffe (insbesondere Grundbegriffe der Mengenlehre und Äquivalenzrelationen), Gruppen, Körper, Vektorräume über beliebigen Körpern, Basis und Dimension, lineare Abbildungen und darstellende Matrix, Matrizenkalkül, lineare Gleichungssysteme, Gauß-Algorithmus, Linearformen, Dualraum, Quotientenvektorräume und Homomorphiesatz, Determinante, Eigenwerte, Polynome, charakteristisches Polynom, Diagonalisierbarkeit, affine Räume. Ideen- und mathematikgeschichtliche Hintergründe der mathematischen Inhalte werden erläutert.
\subsubsection*{\large
    Literatur:
}
Wird in der Vorlesung bekanntgegeben.
\subsubsection*{\large
    Vorkenntnisse:
}
Oberstufenmathematik. \\ Der Besuch des Vorkurses wird empfohlen.
\subsubsection*{\large
    Bemerkungen:
}
This course is only offered in German.
\subsubsection*{\large
    Verwendbarkeit, Studien- und Prüfungsleistungen:
}

\begin{tabularx}{\textwidth}{ p{.5\textwidth}
    |X
    |X
}
 &
\makecell[c]{\rotatebox[origin=l]{90}{\parbox{
            4
            cm}{\begin{flushleft}
                Lineare Algebra (2HfB21, BSc21, MEH21) (9.0 ECTS)
            \end{flushleft} }}}
 &
\makecell[c]{\rotatebox[origin=l]{90}{\parbox{
            4
            cm}{\begin{flushleft}
                Lineare Algebra (MEB21) (6.0 ECTS) \newline Lineare Algebra I (fachfremd) (BScInfo19, BScPhys20) (9.0 ECTS)
            \end{flushleft} }}}
\\
& \Var{veranstaltung["verwendbarkeit"].columns.index(y)}
& \Var{veranstaltung["verwendbarkeit"].columns.index(y)}
\\[2ex] \hline
\hline \rule[0mm]{0cm}{.6cm}PL: Mündliche Prüfung (Dauer ca. 30 Minuten) über Lineare Algebra I und II am Ende des Moduls. (Die bestandene Klausur zu Lineare Algebra I und die bestandene Übung zu Lineare Algebra II sind Zulassungsvoraussetzungen). \rule[-3mm]{0cm}{0cm}
 &
\makecell[c]{\xmark}
 &
\\
\hline \rule[0mm]{0cm}{.6cm}SL: Bestehen der Abschlussklausur (Dauer 1 bis 3 Stunden). \rule[-3mm]{0cm}{0cm}
 &
\makecell[c]{\xmark}
 &
\makecell[c]{\xmark}
\\
\hline \rule[0mm]{0cm}{.6cm}SL: Erreichen von mindestens 50\% der Punkte, die insgesamt durch die Bearbeitung der für die Übung ausgegebenen Übungsaufgaben und Kurztests (''Quizzes'') erreicht werden können. \rule[-3mm]{0cm}{0cm}
 &
\makecell[c]{\xmark}
 &
\makecell[c]{\xmark}
\\
\end{tabularx}




\clearpage\hrule\vskip1pt\hrule
\section*{\Large Numerik I}
\addcontentsline{toc}{subsection}{Numerik I\ \textcolor{gray}{(\em Sören Bartels)}}
\vskip-2ex
{\it Sören Bartels, Assistenz: Tatjana Schreiber}
\hfill{D, 5.0 ECTS}\\
Vorlesung: Mi, 14--16 Uhr, HS Weismann-Haus, \href{https://www.openstreetmap.org/?mlat=48.001683&mlon=7.849361#map=19/48.001683/7.849361}{Albertstr. 21a}\\
Übung: 2-stündig 14-täglich, verschiedene Termine \\
\subsubsection*{\large
    Inhalt:
}
Die Numerik ist eine Teildisziplin der Mathematik, die sich mit der praktischen Lösung mathematischer Aufgaben beschäftigt. Dabei werden Probleme in der Regel nicht exakt sondern approximativ gelöst, wofür ein sinnvoller Kompromiss aus Genauigkeit und Rechenaufwand zu finden ist. Im ersten Teil des zweisemestrigen Kurses stehen Fragestellungen der Linearen Algebra wie das Lösen linearer Gleichungssysteme und die Bestimmung von Eigenwerten einer Matrix im Vordergrund. Der Besuch der begleitenden praktischen Übung wird empfohlen. Diese finden 14-täglich im Wechsel mit der Übung zur Vorlesung statt.
\subsubsection*{\large
    Literatur:
}
\begin{itemize}
\item
S.~Bartels: \emph{Numerik~3x9}. Springer, 2016.
\item
R.~Plato: \emph{Numerische Mathematik kompakt}. Vieweg, 2006.
\item
R.~Schaback, H.~Wendland: \emph{Numerische Mathematik}. Springer, 2004.
\item
J.~Stoer, R.~Burlisch: \emph{Numerische Mathematik~I, II}. Springer, 2007, 2005.
\item
G.~Hämmerlin, K.-H.~Hoffmann: \emph{Numerische Mathematik}. Springer, 1990.
\item
P.~Deuflhard, A.~Hohmann, F.~Bornemann: \emph{Numerische Mathematik~I, II}. DeGruyter, 2003.
\end{itemize}
\subsubsection*{\large
    Vorkenntnisse:
}
Notwendig: Lineare Algebra~I \\
Nützlich: Lineare Algebra~II und Analysis~I (notwendig für Numerik~II)
\subsubsection*{\large
    Bemerkungen:
}
Begleitend zur Vorlesung wird eine Praktische Übung angeboten.
\subsubsection*{\large
    Verwendbarkeit, Studien- und Prüfungsleistungen:
}

\begin{tabularx}{\textwidth}{ p{.5\textwidth}
    |X
    |X
}
 &
\makecell[c]{\rotatebox[origin=l]{90}{\parbox{
            4
            cm}{\begin{flushleft}
                Numerik (2HfB21, MEH21) (4.5 ECTS) \newline Numerik (BSc21) (4.5 ECTS)
            \end{flushleft} }}}
 &
\makecell[c]{\rotatebox[origin=l]{90}{\parbox{
            4
            cm}{\begin{flushleft}
                Numerik I (MEB21) (5.0 ECTS)
            \end{flushleft} }}}
\\
& \Var{veranstaltung["verwendbarkeit"].columns.index(y)}
& \Var{veranstaltung["verwendbarkeit"].columns.index(y)}
\\[2ex] \hline
\hline \rule[0mm]{0cm}{.6cm}PL: Klausur über Numerik I und II am Ende des Sommersemesters (Dauer: 1 bis 3 Stunden). \rule[-3mm]{0cm}{0cm}
 &
\makecell[c]{\xmark}
 &
\\
\hline \rule[0mm]{0cm}{.6cm}PL: Mündliche Prüfung (Dauer: max. 30 Minuten). \rule[-3mm]{0cm}{0cm}
 &
 &
\makecell[c]{\xmark}
\\
\hline \rule[0mm]{0cm}{.6cm}SL: Erreichen von mindestens 50\% der Punkte, die insgesamt durch die Bearbeitung der für die Übung ausgegebenen Übungsaufgaben erreicht werden können. \rule[-3mm]{0cm}{0cm}
 &
\makecell[c]{\xmark}
 &
\makecell[c]{\xmark}
\\
\end{tabularx}




\clearpage\hrule\vskip1pt\hrule
\section*{\Large \href{https://www.stochastik.uni-freiburg.de/de/lehre/ws-2024-2025/vorlesung-stochastik-I-ws-2024-2025}{Stochastik I}}
\addcontentsline{toc}{subsection}{Stochastik I\ \textcolor{gray}{(\em Angelika Rohde)}}
\vskip-2ex
{\it Angelika Rohde, Assistenz: Johannes Brutsche}
\hfill{D, 5.0 ECTS}\\
Vorlesung: Fr, 10--12 Uhr, HS Weismann-Haus, \href{https://www.openstreetmap.org/?mlat=48.001683&mlon=7.849361#map=19/48.001683/7.849361}{Albertstr. 21a}\\
Übung: 2-stündig 14-täglich, verschiedene Termine \\
% Webseite: \url{https://www.stochastik.uni-freiburg.de/de/lehre/ws-2024-2025/vorlesung-stochastik-I-ws-2024-2025}
\subsubsection*{\large
    Inhalt:
}
Stochastik ist, lax gesagt, die „Mathematik des Zufalls“, über den sich – womöglich entgegen der ersten Anschauung – sehr viele präzise und gar nicht zufällige Aussagen formulieren und beweisen lassen. Ziel der Vorlesung ist, eine Einführung in die stochastische Modellbildung zu geben, einige grundlegende Begriffe und Ergebnisse der Stochastik zu erläutern und an Beispielen zu veranschaulichen. Sie ist darüber hinaus auch, speziell für Studierende im B.Sc. Mathematik, als motivierende Vorbereitung für die Vorlesung „Wahrscheinlichkeitstheorie“ im Sommersemester gedacht. Behandelt werden unter anderem: Diskrete und stetige Zufallsvariablen, Wahrscheinlichkeitsräume und -maße, Kombinatorik, Erwartungswert, Varianz, Korrelation, erzeugende Funktionen, bedingte Wahrscheinlichkeit, Unabhängigkeit, Schwaches Gesetz der großen Zahlen, Zentraler Grenzwertsatz.

Die Vorlesung Stochastik~II im Sommersemester wird sich hauptsächlich statistischen Themen widmen. Bei Interesse an einer praktischen, computergestützen Umsetzung einzelner Vorlesungsinhalte wird (parallel oder nachfolgend) zusätzlich die Teilnahme an der regelmäßig angebotenen "`Praktischen Übung Stochastik"' empfohlen.
\subsubsection*{\large
    Literatur:
}
\begin{itemize}
\item L.~Dümbgen: \emph{Stochastik für Informatiker}, Springer, 2003.
\item H.-O.~Georgii: \emph{Stochastik: Einführung in die Wahrscheinlichkeitstheorie und Statistik} (5.~Auf\/lage), De Gruyter, 2015.
\item N.~Henze: \href{https://www.redi-bw.de/start/unifr/EBooks-springer/10.1007/978-3-662-63840-8}{\emph{Stochastik für Einsteiger}}, (13.~Auf\/lage), Springer Spektrum, 2021. 
\item  N.~Henze: \href{https://www.redi-bw.de/start/unifr/EBooks-springer/10.1007/978-3-662-59563-3}{\emph{Stochastik: Eine Einführung mit Grundzügen der Maßtheorie}}, Springer Spektrum, 2019. 
\item  G.~Kersting, A.~Wakolbinger: \href{http://www.redi-bw.de/start/unifr/EBooks-springer/10.1007/978-3-0346-0414-7}{\emph{Elementare Stochastik}} (2. Auf\/lage), Birkhäuser, 2010. 
\end{itemize}
\subsubsection*{\large
    Vorkenntnisse:
}
Lineare Algebra~I sowie Analysis~I und II, wobei Lineare Algebra~I gleichzeitig gehört werden kann.
\subsubsection*{\large
    Bemerkungen:
}
This course is only offered in German.
\subsubsection*{\large
    Verwendbarkeit, Studien- und Prüfungsleistungen:
}

\begin{tabularx}{\textwidth}{ p{.5\textwidth}
    |X
    |X
}
 &
\makecell[c]{\rotatebox[origin=l]{90}{\parbox{
            4
            cm}{\begin{flushleft}
                Stochastik (2HfB21, MEH21) (4.5 ECTS)
            \end{flushleft} }}}
 &
\makecell[c]{\rotatebox[origin=l]{90}{\parbox{
            4
            cm}{\begin{flushleft}
                Stochastik I (BSc21, MEB21, MEdual24) (5.0 ECTS)
            \end{flushleft} }}}
\\
& \Var{veranstaltung["verwendbarkeit"].columns.index(y)}
& \Var{veranstaltung["verwendbarkeit"].columns.index(y)}
\\[2ex] \hline
\hline \rule[0mm]{0cm}{.6cm}PL: Klausur über Stochastik I am Ende des Wintersemesters (Dauer: 1 bis 2 Stunden). \rule[-3mm]{0cm}{0cm}
 &
 &
\makecell[c]{\xmark}
\\
\hline \rule[0mm]{0cm}{.6cm}PL: Klausur über Stochastik I und II am Ende des Sommersemesters (Dauer: 2 bis 4 Stunden). \rule[-3mm]{0cm}{0cm}
 &
\makecell[c]{\xmark}
 &
\\
\hline \rule[0mm]{0cm}{.6cm}SL: Erreichen von mindestens 50\% der Punkte, die insgesamt durch die Bearbeitung der für die Übung ausgegebenen Übungsaufgaben erreicht werden können. \rule[-3mm]{0cm}{0cm}
 &
\makecell[c]{\xmark}
 &
\makecell[c]{\xmark}
\\
\end{tabularx}




\clearpage\hrule\vskip1pt\hrule
\section*{\Large Erweiterung der Analysis}
\addcontentsline{toc}{subsection}{Erweiterung der Analysis\ \textcolor{gray}{(\em Nadine Große)}}
\vskip-2ex
{\it Nadine Große, Assistenz: Jonah Reuß}
\hfill{D, 5.0 ECTS}\\
Vorlesung: Mi, 8--10 Uhr, HS Weismann-Haus, \href{https://www.openstreetmap.org/?mlat=48.001683&mlon=7.849361#map=19/48.001683/7.849361}{Albertstr. 21a}\\
Übung: 2-stündig, verschiedene Termine \\
\subsubsection*{\large
    Inhalt:
}
Mehrfachintegration: Jordan-Inhalt im $\mathbb R^n$, Satz von Fubini, Transformationssatz, Divergenz und Rotation von Vektorfeldern, Pfad- und Oberflächenintegrale im $\mathbb R^3$, Satz von Gauß, Satz von Stokes.

Funktionentheorie: Einführung in die Theorie holomorpher Funktionen, Cauchy’scher Integralsatz, Cauchy’sche Integralformel und Anwendungen.
\subsubsection*{\large
    Literatur:
}
\begin{itemize}
\item
K.~Königsberger: \emph{Analysis~2}, 5.~Auflage., Springer, 2004.
\item
W.~Walter: \emph{Analysis~2}, 5.~Auflage, Springer, 2002.
\item
E.~Freitag, R.~Busam: \emph{Funktionentheorie~I}, 4.~Auflage, Springer, 2006.
\item
R.~Remmert, G.~Schumacher: \emph{Funktionentheorie~1}. 5.~Auflage, Springer, 2002.
\end{itemize}
\subsubsection*{\large
    Vorkenntnisse:
}
Analysis~I und II, Lineare Algebra~I und II
\subsubsection*{\large
    Bemerkungen:
}
This course is only offered in German.
\subsubsection*{\large
    Verwendbarkeit, Studien- und Prüfungsleistungen:
}

\begin{tabularx}{\textwidth}{ p{.5\textwidth}
    |X
}
 &
\makecell[c]{\rotatebox[origin=l]{90}{\parbox{
            4
            cm}{\begin{flushleft}
                Erweiterung der Analysis (MEd18, MEH21, MEdual24) (5.0 ECTS)
            \end{flushleft} }}}
\\
& \Var{veranstaltung["verwendbarkeit"].columns.index(y)}
\\[2ex] \hline
\hline \rule[0mm]{0cm}{.6cm}PL: Klausur (Dauer: 1 bis 3 Stunden). \rule[-3mm]{0cm}{0cm}
 &
\makecell[c]{\xmark}
\\
\hline \rule[0mm]{0cm}{.6cm}SL: Erreichen von mindestens 50\% der Punkte, die insgesamt durch die Bearbeitung der für die Übung ausgegebenen Übungsaufgaben erreicht werden können. \rule[-3mm]{0cm}{0cm}
 &
\makecell[c]{\xmark}
\\
\end{tabularx}




\clearpage\hrule\vskip1pt\hrule
\section*{\Large \href{https://www.stochastik.uni-freiburg.de/de/lehre/ws-2024-2025/lecture-basics-in-applied-mathematics-ws-2024-2025/}{Basics in Applied Mathematics}}
\addcontentsline{toc}{subsection}{Basics in Applied Mathematics\ \textcolor{gray}{(\em Moritz Diehl, Patrick Dondl, Angelika Rohde)}}
\vskip-2ex
{\it Moritz Diehl, Patrick Dondl, Angelika Rohde, Assistenz: Ben Deitmar, Coffi Aristide Hounkpe}
\hfill{E, 12.0 ECTS}\\
Vorlesung: Di, Do, 8--10 Uhr, HS II, \href{https://www.openstreetmap.org/?mlat=48.002320&mlon=7.847924#map=19/48.002320/7.847924}{Albertstr. 23b}\\
Übung: 2-stündig, Termin wird noch festgelegt \\
Praktische Übung: 2-stündig, Termin wird noch festgelegt \\
% Webseite: \url{https://www.stochastik.uni-freiburg.de/de/lehre/ws-2024-2025/lecture-basics-in-applied-mathematics-ws-2024-2025/}
\subsubsection*{\large
    Inhalt:
}
Angaben folgen noch!
\subsubsection*{\large
    Bemerkungen:
}
Dieser Kurs wird auf Englisch angeboten.
\subsubsection*{\large
    Verwendbarkeit, Studien- und Prüfungsleistungen:
}

\begin{tabularx}{\textwidth}{ p{.5\textwidth}
}
\\
\\[2ex] \hline
\end{tabularx}




\clearpage
\phantomsection
\thispagestyle{empty}
\vspace*{\fill}
\begin{center}
    \Huge\bfseries 1b. Weiterführende vierstündige Vorlesungen
\end{center}
\addcontentsline{toc}{section}{\textbf{1b. Weiterführende vierstündige Vorlesungen}}
\addtocontents{toc}{\medskip\hrule\medskip}\vspace*{\fill}\vspace*{\fill}\clearpage
\vfill
\thispagestyle{empty}
\clearpage

\clearpage\hrule\vskip1pt\hrule
\section*{\Large \href{ https://home.mathematik.uni-freiburg.de/soergel/ws2425al.html}{Algebra und Zahlentheorie}}
\addcontentsline{toc}{subsection}{Algebra und Zahlentheorie\ \textcolor{gray}{(\em Wolfgang Soergel)}}
\vskip-2ex
{\it Wolfgang Soergel, Assistenz: Damian Sercombe}
\hfill{D, 9.0 ECTS}\\
Vorlesung: Di, Do, 10--12 Uhr, HS Weismann-Haus, \href{https://www.openstreetmap.org/?mlat=48.001683&mlon=7.849361#map=19/48.001683/7.849361}{Albertstr. 21a}\\
Übung: 2-stündig, verschiedene Termine \\
% Webseite: \url{ https://home.mathematik.uni-freiburg.de/soergel/ws2425al.html}
\subsubsection*{\large
    Inhalt:
}
Diese Vorlesung setzt die Lineare Algebra fort. Behandelt werden Gruppen, Ringe, Körper sowie Anwendungen in der Zahlentheorie und Geometrie. Höhepunkte der Vorlesung sind die Klassifikation endlicher Körper, die Unmöglichkeit der Winkeldreiteilung mit Zirkel und Lineal, die Nicht-Existenz von Lösungsformeln für allgemeine Gleichungen fünften Grades und das quadratische Reziprozitätsgesetz.
\subsubsection*{\large
    Literatur:
}
\begin{itemize}
\item Michael Artin: \emph{Algebra}, Birkhäuser 1998.
\item Siegfried Bosch: Algebra (8. Auf"|lage.), Springer Spektrum 2013.
\item Serge Lang: \emph{Algebra} (3. Auf"|lage.), Springer 2002.
\item Wolfgang Soergel: Skript \emph{Algebra und Zahlentheorie}
\end{itemize}
\subsubsection*{\large
    Vorkenntnisse:
}
Lineare Algebra~I und II
\subsubsection*{\large
    Bemerkungen:
}
This course is only offered in German.
\subsubsection*{\large
    Verwendbarkeit, Studien- und Prüfungsleistungen:
}

\begin{tabularx}{\textwidth}{ p{.5\textwidth}
    |X
    |X
    |X
    |X
    |X
}
 &
\makecell[c]{\rotatebox[origin=l]{90}{\parbox{
            8
            cm}{\begin{flushleft}
                Algebra und Zahlentheorie (2HfB21, MEH21) (9.0 ECTS) \newline Wahlpflichtmodul Mathematik (BSc21) (9.0 ECTS)
            \end{flushleft} }}}
 &
\makecell[c]{\rotatebox[origin=l]{90}{\parbox{
            8
            cm}{\begin{flushleft}
                Algebra und Zahlentheorie (MEdual24) (9.0 ECTS)
            \end{flushleft} }}}
 &
\makecell[c]{\rotatebox[origin=l]{90}{\parbox{
            8
            cm}{\begin{flushleft}
                Einführung in die Algebra und Zahlentheorie (MEB21) (5.0 ECTS)
            \end{flushleft} }}}
 &
\makecell[c]{\rotatebox[origin=l]{90}{\parbox{
            8
            cm}{\begin{flushleft}
                Reine Mathematik (MSc14) (11.0 ECTS)
            \end{flushleft} }}}
 &
\makecell[c]{\rotatebox[origin=l]{90}{\parbox{
            8
            cm}{\begin{flushleft}
                Wahlmodul (MSc14) (9.0 ECTS) \newline Wahlmodul (MScData24) (9.0 ECTS)
            \end{flushleft} }}}
\\
& \Var{veranstaltung["verwendbarkeit"].columns.index(y)}
& \Var{veranstaltung["verwendbarkeit"].columns.index(y)}
& \Var{veranstaltung["verwendbarkeit"].columns.index(y)}
& \Var{veranstaltung["verwendbarkeit"].columns.index(y)}
& \Var{veranstaltung["verwendbarkeit"].columns.index(y)}
\\[2ex] \hline
\hline \rule[0mm]{0cm}{.6cm}PL: Klausur (Dauer: 1 bis 3 Stunden). \rule[-3mm]{0cm}{0cm}
 &
\makecell[c]{\xmark}
 &
 &
 &
 &
\\
\hline \rule[0mm]{0cm}{.6cm}PL: Mündliche Prüfung (Dauer: max. 30 Minuten). \rule[-3mm]{0cm}{0cm}
 &
 &
\makecell[c]{\xmark}
 &
 &
\makecell[c]{\xmark}
 &
\\
\hline \rule[0mm]{0cm}{.6cm}PL: Mündliche Prüfung über den ersten Teil der Veranstaltung bis Weihnachten (Dauer: max. 30 Minuten) \rule[-3mm]{0cm}{0cm}
 &
 &
 &
\makecell[c]{\xmark}
 &
 &
\\
\hline \rule[0mm]{0cm}{.6cm}SL: Bestehen der Abschlussklausur (Dauer 1 bis 3 Stunden). \rule[-3mm]{0cm}{0cm}
 &
 &
 &
 &
\makecell[c]{\xmark}
 &
\makecell[c]{\xmark}
\\
\hline \rule[0mm]{0cm}{.6cm}SL: Erreichen von mindestens 50\% der Punkte, die insgesamt durch die Bearbeitung der für die Übung ausgegebenen Übungsaufgaben erreicht werden können. \rule[-3mm]{0cm}{0cm}
 &
\makecell[c]{\xmark}
 &
\makecell[c]{\xmark}
 &
\makecell[c]{\xmark}
 &
\makecell[c]{\xmark}
 &
\makecell[c]{\xmark}
\\
\end{tabularx}




\clearpage\hrule\vskip1pt\hrule
\section*{\Large Algebraische Zahlentheorie}
\addcontentsline{toc}{subsection}{Algebraische Zahlentheorie\ \textcolor{gray}{(\em Abhishek Oswal)}}
\vskip-2ex
{\it Abhishek Oswal, Assistenz: Andreas Demleitner}
\hfill{E, 9.0 ECTS}\\
Vorlesung: Di, Do, 12--14 Uhr, HS II, \href{https://www.openstreetmap.org/?mlat=48.002320&mlon=7.847924#map=19/48.002320/7.847924}{Albertstr. 23b}\\
Übung: 2-stündig, Termin wird noch festgelegt \\
\subsubsection*{\large
    Inhalt:
}
Short description of topics: Number fields, Prime decomposition in Dedekind domains, Ideal class groups, Unit groups, Dirichlet's unit theorem, local fields, valuations, decomposition and inertia groups, introduction to class field theory.  
\subsubsection*{\large
    Literatur:
}
Jürgen Neukirch: \emph{Algebraic Number Theory}, Springer, 1999.
\subsubsection*{\large
    Vorkenntnisse:
}
Algebra und Zahlentheorie
\subsubsection*{\large
    Bemerkungen:
}
Dieser Kurs wird auf Englisch angeboten.
\subsubsection*{\large
    Verwendbarkeit, Studien- und Prüfungsleistungen:
}

\begin{tabularx}{\textwidth}{ p{.5\textwidth}
    |X
    |X
    |X
}
 &
\makecell[c]{\rotatebox[origin=l]{90}{\parbox{
            4
            cm}{\begin{flushleft}
                Mathematik (MSc14) (11.0 ECTS) \newline Mathematische Vertiefung (MEd18, MEH21) (9.0 ECTS) \newline Reine Mathematik (MSc14) (11.0 ECTS) \newline Wahlpflichtmodul Mathematik (BSc21) (9.0 ECTS)
            \end{flushleft} }}}
 &
\makecell[c]{\rotatebox[origin=l]{90}{\parbox{
            4
            cm}{\begin{flushleft}
                Teil des Vertiefungsmoduls (MSc14) (10.5 ECTS)
            \end{flushleft} }}}
 &
\makecell[c]{\rotatebox[origin=l]{90}{\parbox{
            4
            cm}{\begin{flushleft}
                Wahlmodul (MSc14) (9.0 ECTS) \newline Wahlmodul (MScData24) (9.0 ECTS) \newline Wahlmodul (Option ''Individuelle Studiengestaltung'') (2HfB21) (9.0 ECTS)
            \end{flushleft} }}}
\\
& \Var{veranstaltung["verwendbarkeit"].columns.index(y)}
& \Var{veranstaltung["verwendbarkeit"].columns.index(y)}
& \Var{veranstaltung["verwendbarkeit"].columns.index(y)}
\\[2ex] \hline
\hline \rule[0mm]{0cm}{.6cm}PL: Mündliche Prüfung (Dauer: ca. 30 Minuten). \rule[-3mm]{0cm}{0cm}
 &
\makecell[c]{\xmark}
 &
 &
\\
\hline \rule[0mm]{0cm}{.6cm}PL: Mündliche Prüfung über alle Teile des Moduls (Dauer:  45 Minuten) \rule[-3mm]{0cm}{0cm}
 &
 &
\makecell[c]{\xmark}
 &
\\
\hline \rule[0mm]{0cm}{.6cm}SL: Bestehen eines mündlichen Abschlusstests. \rule[-3mm]{0cm}{0cm}
 &
 &
 &
\makecell[c]{\xmark}
\\
\hline \rule[0mm]{0cm}{.6cm}SL: Erreichen von mindestens 50\% der Punkte, die insgesamt durch die Bearbeitung der für die Übung ausgegebenen Übungsaufgaben erreicht werden können. \rule[-3mm]{0cm}{0cm}
 &
\makecell[c]{\xmark}
 &
\makecell[c]{\xmark}
 &
\makecell[c]{\xmark}
\\
\end{tabularx}




\clearpage\hrule\vskip1pt\hrule
\section*{\Large \href{https://aam.uni-freiburg.de/agdo/lehre/ws24/analysis3/index.html}{Analysis III}}
\addcontentsline{toc}{subsection}{Analysis III\ \textcolor{gray}{(\em Patrick Dondl)}}
\vskip-2ex
{\it Patrick Dondl, Assistenz: Oliver Suchan}
\hfill{D, 9.0 ECTS}\\
Vorlesung: Mo, 12--14 Uhr, HS Rundbau, \href{https://www.openstreetmap.org/?mlat=48.001613&mlon=7.849254#map=19/48.001613/7.849254}{Albertstr. 21}, Mi, 10--12 Uhr, HS Weismann-Haus, \href{https://www.openstreetmap.org/?mlat=48.001683&mlon=7.849361#map=19/48.001683/7.849361}{Albertstr. 21a}\\
Übung: 2-stündig, verschiedene Termine \\
% Webseite: \url{https://aam.uni-freiburg.de/agdo/lehre/ws24/analysis3/index.html}
\subsubsection*{\large
    Inhalt:
}
Lebesgue-Maß und Maßtheorie, Lebesgue-Integral auf Maßräumen und Satz von Fubini, Fourier-Reihen und Fourier-Transformation, Hilbert-Räume.
Differentialformen, ihre Integration und äußere Ableitung. Satz von Stokes und Satz von Gauß.
\subsubsection*{\large
    Vorkenntnisse:
}
Analysis I und II, Lineare Algebra I
\subsubsection*{\large
    Bemerkungen:
}
This course is only offered in German.
\subsubsection*{\large
    Verwendbarkeit, Studien- und Prüfungsleistungen:
}

\begin{tabularx}{\textwidth}{ p{.5\textwidth}
    |X
    |X
    |X
}
 &
\makecell[c]{\rotatebox[origin=l]{90}{\parbox{
            4
            cm}{\begin{flushleft}
                Analysis III (BSc21) (9.0 ECTS) \newline Elective in Data (MScData24) (9.0 ECTS)
            \end{flushleft} }}}
 &
\makecell[c]{\rotatebox[origin=l]{90}{\parbox{
            4
            cm}{\begin{flushleft}
                Mathematische Vertiefung (MEd18, MEH21) (9.0 ECTS)
            \end{flushleft} }}}
 &
\makecell[c]{\rotatebox[origin=l]{90}{\parbox{
            4
            cm}{\begin{flushleft}
                Wahlmodul (Option ''Individuelle Studiengestaltung'') (2HfB21) (9.0 ECTS)
            \end{flushleft} }}}
\\
& \Var{veranstaltung["verwendbarkeit"].columns.index(y)}
& \Var{veranstaltung["verwendbarkeit"].columns.index(y)}
& \Var{veranstaltung["verwendbarkeit"].columns.index(y)}
\\[2ex] \hline
\hline \rule[0mm]{0cm}{.6cm}PL: Klausur (Dauer: 1 bis 3 Stunden). \rule[-3mm]{0cm}{0cm}
 &
\makecell[c]{\xmark}
 &
 &
\\
\hline \rule[0mm]{0cm}{.6cm}PL: Mündliche Prüfung (Dauer: ca. 30 Minuten). \rule[-3mm]{0cm}{0cm}
 &
 &
\makecell[c]{\xmark}
 &
\\
\hline \rule[0mm]{0cm}{.6cm}SL: Bestehen der Abschlussklausur (Dauer 1 bis 3 Stunden). \rule[-3mm]{0cm}{0cm}
 &
 &
 &
\makecell[c]{\xmark}
\\
\hline \rule[0mm]{0cm}{.6cm}SL: Erreichen von mindestens 50\% der Punkte, die insgesamt durch die Bearbeitung der für die Übung ausgegebenen Übungsaufgaben erreicht werden können. \rule[-3mm]{0cm}{0cm}
 &
\makecell[c]{\xmark}
 &
\makecell[c]{\xmark}
 &
\makecell[c]{\xmark}
\\
\end{tabularx}




\clearpage\hrule\vskip1pt\hrule
\section*{\Large \href{https://home.mathematik.uni-freiburg.de/geometrie/lehre/ws2024/DG/}{Differentialgeometrie}}
\addcontentsline{toc}{subsection}{Differentialgeometrie\ \textcolor{gray}{(\em Sebastian Goette)}}
\vskip-2ex
{\it Sebastian Goette, Assistenz: Mikhael Tëmkin}
\hfill{D, 9.0 ECTS}\\
Vorlesung: Mo, Mi, 14--16 Uhr, HS II, \href{https://www.openstreetmap.org/?mlat=48.002320&mlon=7.847924#map=19/48.002320/7.847924}{Albertstr. 23b}\\
Übung: 2-stündig, Termin wird noch festgelegt \\
% Webseite: \url{https://home.mathematik.uni-freiburg.de/geometrie/lehre/ws2024/DG/}
\subsubsection*{\large
    Inhalt:
}
Die Differentialgeometrie, speziell die Riemannsche Geometrie, besch\"aftigt sich mit den geometrischen Eigenschaften gekr\"ummter R\"aume.
Solche R\"aume treten auch in anderen Bereichen der Mathematik und Physik auf, beispielsweise in der geometrischen Analysis, der theoretischen Mechanik
und der allgemeinen Relativit\"atstheorie.

Im ersten Teil der Vorlesung lernen wir Grundbegriffe der Differentialgeometrie (z.\ B. differenzierbare Mannigfaltigkeiten, Vektorb\"undel, Zusammenh\"ange und ihre Kr\"ummung) und der Riemannschen Geometrie (Riemannscher Kr\"ummungstensor, Geod\"atische, Jacobi-Felder etc.) kennen.

Im zweiten Teil betrachten wir das Zusammenspiel zwischen lokalen Eigenschaften Riemannscher Mannigfaltigkeiten wie der Kr\"ummung und globalen topologischen und geometrischen Eigenschaften wie Kompaktheit, Fundamentalgruppe, Durchmesser, Volumenwachstum und Gestalt geod\"atischer Dreiecke.
\subsubsection*{\large
    Literatur:
}
\begin{itemize}
\item{J. Cheeger, D. G. Ebin, {\em Comparison Theorems in Riemannian Geometry,\/} North-Holland, Amsterdam 1975.}
\item{S. Gallot, D. Hulin, J. Lafontaine, {\em Riemannian Geometry,\/} Springer, Berlin-Heidelberg-New York 1987.}
\item{P. Petersen, {\em Riemannian Geometry,\/} Grad. Texts Math.~171, Springer, New York, 2006.}
\end{itemize}
\subsubsection*{\large
    Vorkenntnisse:
}
Notwendig: Analysis~I–III, Lineare Algebra~I und II \\
Nützlich: Kurven und Flächen, Topologie
\subsubsection*{\large
    Bemerkungen:
}
Im Sommersemester 2025 wird voraussichtlich eine Vorlesung über Differentialgeometrie~II angeboten. 
\subsubsection*{\large
    Verwendbarkeit, Studien- und Prüfungsleistungen:
}

\begin{tabularx}{\textwidth}{ p{.5\textwidth}
    |X
    |X
    |X
}
 &
\makecell[c]{\rotatebox[origin=l]{90}{\parbox{
            4
            cm}{\begin{flushleft}
                Mathematik (MSc14) (11.0 ECTS) \newline Mathematische Vertiefung (MEd18, MEH21) (9.0 ECTS) \newline Reine Mathematik (MSc14) (11.0 ECTS) \newline Wahlpflichtmodul Mathematik (BSc21) (9.0 ECTS)
            \end{flushleft} }}}
 &
\makecell[c]{\rotatebox[origin=l]{90}{\parbox{
            4
            cm}{\begin{flushleft}
                Teil des Vertiefungsmoduls (MSc14) (10.5 ECTS)
            \end{flushleft} }}}
 &
\makecell[c]{\rotatebox[origin=l]{90}{\parbox{
            4
            cm}{\begin{flushleft}
                Wahlmodul (MSc14) (9.0 ECTS) \newline Wahlmodul (MScData24) (9.0 ECTS) \newline Wahlmodul (Option ''Individuelle Studiengestaltung'') (2HfB21) (9.0 ECTS)
            \end{flushleft} }}}
\\
& \Var{veranstaltung["verwendbarkeit"].columns.index(y)}
& \Var{veranstaltung["verwendbarkeit"].columns.index(y)}
& \Var{veranstaltung["verwendbarkeit"].columns.index(y)}
\\[2ex] \hline
\hline \rule[0mm]{0cm}{.6cm}PL: Mündliche Prüfung (Dauer: ca. 30 Minuten). \rule[-3mm]{0cm}{0cm}
 &
\makecell[c]{\xmark}
 &
 &
\\
\hline \rule[0mm]{0cm}{.6cm}PL: Mündliche Prüfung über alle Teile des Moduls (Dauer:  45 Minuten) \rule[-3mm]{0cm}{0cm}
 &
 &
\makecell[c]{\xmark}
 &
\\
\hline \rule[0mm]{0cm}{.6cm}SL: Bestehen eines mündlichen Abschlusstests. \rule[-3mm]{0cm}{0cm}
 &
 &
 &
\makecell[c]{\xmark}
\\
\hline \rule[0mm]{0cm}{.6cm}SL: Erreichen von mindestens 50\% der Punkte, die insgesamt durch die Bearbeitung der für die Übung ausgegebenen Übungsaufgaben erreicht werden können. \rule[-3mm]{0cm}{0cm}
 &
\makecell[c]{\xmark}
 &
\makecell[c]{\xmark}
 &
\makecell[c]{\xmark}
\\
\end{tabularx}




\clearpage\hrule\vskip1pt\hrule
\section*{\Large \href{https://home.mathematik.uni-freiburg.de/analysis/2024_WiSe_Lehre/2024_WiSe_PDE/}{Einführung in partielle Differentialgleichungen}}
\addcontentsline{toc}{subsection}{Einführung in partielle Differentialgleichungen\ \textcolor{gray}{(\em Guofang Wang)}}
\vskip-2ex
{\it Guofang Wang, Assistenz: Christine Schmidt}
\hfill{D, 9.0 ECTS}\\
Vorlesung: Mo, Mi, 12--14 Uhr, HS II, \href{https://www.openstreetmap.org/?mlat=48.002320&mlon=7.847924#map=19/48.002320/7.847924}{Albertstr. 23b}\\
Übung: 2-stündig, Termin wird noch festgelegt \\
% Webseite: \url{https://home.mathematik.uni-freiburg.de/analysis/2024_WiSe_Lehre/2024_WiSe_PDE/}
\subsubsection*{\large
    Inhalt:
}
Eine Vielzahl unterschiedlicher Probleme aus den Naturwissenschaften und der Geometrie führt auf partielle Differentialgleichungen. Mithin kann keine Rede von einer allumfassenden Theorie sein. Dennoch gibt es für lineare Gleichungen ein klares Bild, das sich an drei Prototypen orientiert: der Potentialgleichung $-\Delta u = f$, der Wärmeleitungsgleichung $u_t - \Delta u = f$ und der Wellengleichung $u_{tt} - \Delta u = f$, die wir in der Vorlesung untersuchen werden.
\subsubsection*{\large
    Literatur:
}
\begin{itemize}
\item
E. DiBenedetto: \href{https://link.springer.com/book/10.1007/978-0-8176-4552-6}{\emph{Partial differential equations}}, Birkhäuser, 2010. 
\item
L. C. Evans: \href{http://home.ustc.edu.cn/\~xushijie/pdf/textbooks/pde-evans.pdf}{\emph{Partial Differential Equations}} (Second Edition), Graduate Studies in Mathematics 19, AMS, 2010.
\item
Q. Han: \href{https://pdfcoffee.com/a-basic-course-in-partial-differential-equations-qing-han-pdf-free.html}{\emph{A Basic Course in Partial Differential Equations}}, Graduate Studies in Mathematics 120, AMS, 2011. 
\item
J. Jost: \href{http://www.redi-bw.de/start/unifr/EBooks-springer/10.1007/978-1-4614-4809-9}{\emph{Partial Differential Equations}} (Third Edition), Springer, 2013. 
\end{itemize}
\subsubsection*{\large
    Vorkenntnisse:
}
Notwendig: Analysis III \\
Nützlich: Funktionentheorie
\subsubsection*{\large
    Bemerkungen:
}
This course is only offered in German.
\subsubsection*{\large
    Verwendbarkeit, Studien- und Prüfungsleistungen:
}

\begin{tabularx}{\textwidth}{ p{.5\textwidth}
    |X
    |X
    |X
}
 &
\makecell[c]{\rotatebox[origin=l]{90}{\parbox{
            4
            cm}{\begin{flushleft}
                Mathematik (MSc14) (11.0 ECTS) \newline Mathematische Vertiefung (MEd18, MEH21) (9.0 ECTS) \newline Reine Mathematik (MSc14) (11.0 ECTS) \newline Wahlpflichtmodul Mathematik (BSc21) (9.0 ECTS)
            \end{flushleft} }}}
 &
\makecell[c]{\rotatebox[origin=l]{90}{\parbox{
            4
            cm}{\begin{flushleft}
                Teil des Vertiefungsmoduls (MSc14) (10.5 ECTS)
            \end{flushleft} }}}
 &
\makecell[c]{\rotatebox[origin=l]{90}{\parbox{
            4
            cm}{\begin{flushleft}
                Wahlmodul (MSc14) (9.0 ECTS) \newline Wahlmodul (MScData24) (9.0 ECTS) \newline Wahlmodul (Option ''Individuelle Studiengestaltung'') (2HfB21) (9.0 ECTS)
            \end{flushleft} }}}
\\
& \Var{veranstaltung["verwendbarkeit"].columns.index(y)}
& \Var{veranstaltung["verwendbarkeit"].columns.index(y)}
& \Var{veranstaltung["verwendbarkeit"].columns.index(y)}
\\[2ex] \hline
\hline \rule[0mm]{0cm}{.6cm}PL: Mündliche Prüfung (Dauer: ca. 30 Minuten). \rule[-3mm]{0cm}{0cm}
 &
\makecell[c]{\xmark}
 &
 &
\\
\hline \rule[0mm]{0cm}{.6cm}PL: Mündliche Prüfung über alle Teile des Moduls (Dauer:  45 Minuten) \rule[-3mm]{0cm}{0cm}
 &
 &
\makecell[c]{\xmark}
 &
\\
\hline \rule[0mm]{0cm}{.6cm}SL: Bestehen eines mündlichen Abschlusstests. \rule[-3mm]{0cm}{0cm}
 &
 &
 &
\makecell[c]{\xmark}
\\
\hline \rule[0mm]{0cm}{.6cm}SL: Erreichen von mindestens 50\% der Punkte, die insgesamt durch die Bearbeitung der für die Übung ausgegebenen Übungsaufgaben erreicht werden können. \rule[-3mm]{0cm}{0cm}
 &
\makecell[c]{\xmark}
 &
\makecell[c]{\xmark}
 &
\makecell[c]{\xmark}
\\
\end{tabularx}




\clearpage\hrule\vskip1pt\hrule
\section*{\Large Funktionentheorie}
\addcontentsline{toc}{subsection}{Funktionentheorie\ \textcolor{gray}{(\em David Criens)}}
\vskip-2ex
{\it David Criens, Assistenz: Eric Trébuchon}
\hfill{D, 9.0 ECTS}\\
Vorlesung: Di, Mi, 16--18 Uhr, HS II, \href{https://www.openstreetmap.org/?mlat=48.002320&mlon=7.847924#map=19/48.002320/7.847924}{Albertstr. 23b}\\
Übung: 2-stündig, Termin wird noch festgelegt \\
\subsubsection*{\large
    Inhalt:
}
Die Funktionentheorie beschäftigt sich mit Funktionen $f : \mathbb C \to \mathbb C$ , die komplexe Zahlen auf komplexe
Zahlen abbilden. Viele Konzepte der Analysis~I lassen sich direkt auf diesen Fall übertragen, z.\,B. die
Definition der Differenzierbarkeit. Man würde vielleicht erwarten, dass sich dadurch eine zur Analysis~I
analoge Theorie entwickelt, doch viel mehr ist wahr: Man erhält eine in vielerlei Hinsicht elegantere und
einfachere Theorie. Beispielsweise impliziert die komplexe Differenzierbarkeit auf einer offenen Menge, dass
eine Funktion sogar unendlich oft differenzierbar ist, und dies stimmt weiter mit Analytizität überein. Für
reelle Funktionen sind alle diese Begriffe unterschiedlich. Doch auch einige neue Ideen sind notwendig: Für
reelle Zahlen $a$, $b$ integriert man für
$$\int_a^b f(x) \mathrm dx$$
über die Elemente des Intervalls $[a, b]$ bzw. $[b, a]$. Sind $a$, $b$ jedoch komplexe Zahlen, ist nicht mehr so
klar, wie man ein solches Integral auf"|fassen soll. Man könnte z.\,B. in den komplexen Zahlen entlang der
Strecke, die $a, b \in \mathbb C$ verbindet, integrieren, oder aber entlang einer anderen Kurve, die von $a$ nach $b$ führt.
Führt dies zu einem wohldefinierten Integralbegriff oder hängt ein solches Kurvenintegral von der Wahl
der Kurve ab?
\subsubsection*{\large
    Vorkenntnisse:
}
Analysis I+II, Lineare Algebra I
\subsubsection*{\large
    Bemerkungen:
}
This course is only offered in German.
\subsubsection*{\large
    Verwendbarkeit, Studien- und Prüfungsleistungen:
}

\begin{tabularx}{\textwidth}{ p{.5\textwidth}
    |X
    |X
    |X
    |X
}
 &
\makecell[c]{\rotatebox[origin=l]{90}{\parbox{
            8
            cm}{\begin{flushleft}
                Mathematische Vertiefung (MEd18, MEH21) (9.0 ECTS)
            \end{flushleft} }}}
 &
\makecell[c]{\rotatebox[origin=l]{90}{\parbox{
            8
            cm}{\begin{flushleft}
                Reine Mathematik (MSc14) (11.0 ECTS)
            \end{flushleft} }}}
 &
\makecell[c]{\rotatebox[origin=l]{90}{\parbox{
            8
            cm}{\begin{flushleft}
                Wahlmodul (MSc14) (9.0 ECTS) \newline Wahlmodul (MScData24) (9.0 ECTS) \newline Wahlmodul (Option ''Individuelle Studiengestaltung'') (2HfB21) (9.0 ECTS)
            \end{flushleft} }}}
 &
\makecell[c]{\rotatebox[origin=l]{90}{\parbox{
            8
            cm}{\begin{flushleft}
                Wahlpflichtmodul Mathematik (BSc21) (9.0 ECTS)
            \end{flushleft} }}}
\\
& \Var{veranstaltung["verwendbarkeit"].columns.index(y)}
& \Var{veranstaltung["verwendbarkeit"].columns.index(y)}
& \Var{veranstaltung["verwendbarkeit"].columns.index(y)}
& \Var{veranstaltung["verwendbarkeit"].columns.index(y)}
\\[2ex] \hline
\hline \rule[0mm]{0cm}{.6cm}PL: Klausur (Dauer: 1 bis 3 Stunden) \rule[-3mm]{0cm}{0cm}
 &
 &
 &
 &
\makecell[c]{\xmark}
\\
\hline \rule[0mm]{0cm}{.6cm}PL: Mündliche Prüfung (Dauer: ca. 30 Minuten). \rule[-3mm]{0cm}{0cm}
 &
\makecell[c]{\xmark}
 &
\makecell[c]{\xmark}
 &
 &
\\
\hline \rule[0mm]{0cm}{.6cm}SL: Bestehen der Abschlussklausur (ein- bis dreistündig). \rule[-3mm]{0cm}{0cm}
 &
 &
\makecell[c]{\xmark}
 &
\makecell[c]{\xmark}
 &
\\
\hline \rule[0mm]{0cm}{.6cm}SL: Erreichen von mindestens 50\% der Punkte, die insgesamt durch die Bearbeitung der für die Übung ausgegebenen Übungsaufgaben erreicht werden können. \rule[-3mm]{0cm}{0cm}
 &
\makecell[c]{\xmark}
 &
\makecell[c]{\xmark}
 &
\makecell[c]{\xmark}
 &
\makecell[c]{\xmark}
\\
\end{tabularx}




\clearpage\hrule\vskip1pt\hrule
\section*{\Large \href{ https://aam.uni-freiburg.de/agba/lehre/ws24/tun0/index.html}{Einführung in Theorie und Numerik Partieller Differentialgleichungen}}
\addcontentsline{toc}{subsection}{Einführung in Theorie und Numerik Partieller Differentialgleichungen\ \textcolor{gray}{(\em Sören Bartels)}}
\vskip-2ex
{\it Sören Bartels, Assistenz: Vera Jackisch}
\hfill{E, 9.0 ECTS}\\
Vorlesung: Di, Do, 10--12 Uhr, SR 226, \href{https://www.openstreetmap.org/?mlat=48.003472&mlon=7.848195#map=19/48.003472/7.848195}{Hermann-Herder-Str. 10}\\
Übung: 2-stündig, Termin wird noch festgelegt \\
% Webseite: \url{ https://aam.uni-freiburg.de/agba/lehre/ws24/tun0/index.html}
\subsubsection*{\large
    Inhalt:
}
Ziel dieses Kurses ist es, eine Einführung in die Theorie der linearen partiellen Differentialgleichungen und deren Finite-Differenzen- sowie Finite-Elemente-Approximationen. Finite-Elemente-Methoden zur Approximation partieller Differentialgleichungen haben einen hohen Reifegrad erreicht und sind ein unverzichtbares Werkzeug in Wissenschaft und Technik. Wir geben eine Einführung in die Konstruktion, Analyse und Implementierung von Finite-Elemente-Methoden für verschiedene Modellprobleme. Wir behandeln elementare Eigenschaften von linearen partiellen Differentialgleichungen zusammen mit deren grundlegender numerischer Approximation, dem funktionalanalytischen Ansatz für den strengen Nachweis der Existenz von Lösungen sowie die Konstruktion und Analyse grundlegender Finite-Elemente-Methoden.
\subsubsection*{\large
    Literatur:
}
\begin{itemize}
\item  S. Bartels: Numerical Approximation of Partial Differential Equations, Springer 2016. 
\item  D. Braess: Finite Elemente, Springer 2007. 
\item  S. Brenner, R. Scott: Finite Elements, Springer 2008. 
\item  L. C. Evans: Partial Differential Equations, AMS 2010
\end{itemize}
\subsubsection*{\large
    Vorkenntnisse:
}
Required: Analysis~I and II, Linear Algebra~I and II as well as knowledge about higher-dimensional integration (e.g. from Analysis~III or Extensions of Analysis) \\
Recommended:  Numerics for differential equations, Functional analysis
\subsubsection*{\large
    Bemerkungen:
}
Dieser Kurs wird auf Englisch angeboten.
\subsubsection*{\large
    Verwendbarkeit, Studien- und Prüfungsleistungen:
}

\begin{tabularx}{\textwidth}{ p{.5\textwidth}
    |X
    |X
    |X
    |X
}
 &
\makecell[c]{\rotatebox[origin=l]{90}{\parbox{
            8
            cm}{\begin{flushleft}
                Advanced Lecture in Numerics (MScData24) (9.0 ECTS) \newline Elective in Data (MScData24) (9.0 ECTS) \newline Wahlpflichtmodul Mathematik (BSc21) (9.0 ECTS)
            \end{flushleft} }}}
 &
\makecell[c]{\rotatebox[origin=l]{90}{\parbox{
            8
            cm}{\begin{flushleft}
                Angewandte Mathematik (MSc14) (11.0 ECTS) \newline Mathematik (MSc14) (11.0 ECTS) \newline Mathematische Vertiefung (MEd18, MEH21) (9.0 ECTS)
            \end{flushleft} }}}
 &
\makecell[c]{\rotatebox[origin=l]{90}{\parbox{
            8
            cm}{\begin{flushleft}
                Teil des Vertiefungsmoduls (MSc14) (10.5 ECTS)
            \end{flushleft} }}}
 &
\makecell[c]{\rotatebox[origin=l]{90}{\parbox{
            8
            cm}{\begin{flushleft}
                Wahlmodul (MSc14) (9.0 ECTS) \newline Wahlmodul (Option ''Individuelle Studiengestaltung'') (2HfB21) (9.0 ECTS)
            \end{flushleft} }}}
\\
& \Var{veranstaltung["verwendbarkeit"].columns.index(y)}
& \Var{veranstaltung["verwendbarkeit"].columns.index(y)}
& \Var{veranstaltung["verwendbarkeit"].columns.index(y)}
& \Var{veranstaltung["verwendbarkeit"].columns.index(y)}
\\[2ex] \hline
\hline \rule[0mm]{0cm}{.6cm}PL: Klausur (Dauer: 1 bis 3 Stunden) \rule[-3mm]{0cm}{0cm}
 &
\makecell[c]{\xmark}
 &
 &
 &
\\
\hline \rule[0mm]{0cm}{.6cm}PL: Mündliche Prüfung (Dauer: ca. 30 Minuten). \rule[-3mm]{0cm}{0cm}
 &
 &
\makecell[c]{\xmark}
 &
 &
\\
\hline \rule[0mm]{0cm}{.6cm}PL: Mündliche Prüfung über alle Teile des Moduls (Dauer:  45 Minuten) \rule[-3mm]{0cm}{0cm}
 &
 &
 &
\makecell[c]{\xmark}
 &
\\
\hline \rule[0mm]{0cm}{.6cm}SL: Bestehen der Abschlussklausur (ein- bis dreistündig). \rule[-3mm]{0cm}{0cm}
 &
 &
 &
 &
\makecell[c]{\xmark}
\\
\hline \rule[0mm]{0cm}{.6cm}SL: Erreichen von mindestens 50\% der Punkte, die insgesamt durch die Bearbeitung der für die Übung ausgegebenen Übungsaufgaben erreicht werden können. \rule[-3mm]{0cm}{0cm}
 &
\makecell[c]{\xmark}
 &
\makecell[c]{\xmark}
 &
\makecell[c]{\xmark}
 &
\makecell[c]{\xmark}
\\
\end{tabularx}




\clearpage\hrule\vskip1pt\hrule
\section*{\Large \href{https://www.stochastik.uni-freiburg.de/de/lehre/ws-2024-2025/lecture-mathematical-statistics-ws-2024-2025}{Mathematische Statistik}}
\addcontentsline{toc}{subsection}{Mathematische Statistik\ \textcolor{gray}{(\em Ernst August v. Hammerstein)}}
\vskip-2ex
{\it Ernst August v. Hammerstein, Assistenz: Sebastian Stroppel}
\hfill{E, 9.0 ECTS}\\
Vorlesung: Mo, Mi, 14--16 Uhr, SR 404, \href{https://www.openstreetmap.org/?mlat=48.000637&mlon=7.846006#map=19/48.000636/7.846006}{Ernst-Zermelo-Str. 1}\\
Übung: 2-stündig, Termin wird noch festgelegt \\
% Webseite: \url{https://www.stochastik.uni-freiburg.de/de/lehre/ws-2024-2025/lecture-mathematical-statistics-ws-2024-2025}
\subsubsection*{\large
    Inhalt:
}
Die Vorlesung "`Mathematische Statistik"' baut auf Grundkenntnissen aus der Vorlesung "`Wahrscheinlichkeitstheorie"' auf. 
Das grundlegende Problem der Statistik ist, anhand einer Stichprobe von Beobachtungen möglichst präzise Aussagen über den datengenerierenden
Prozess bzw. die den Daten zugrundeliegenden Verteilungen zu machen. Hierzu werden in der Vorlesung die wichtigsten Methoden aus der statistischen Entscheidungstheorie wie Test- und Schätzverfahren eingeführt.

Stichworte hierzu sind u.a. Bayes-Schätzer und -Tests, Neyman-Pearson-Testtheorie, Maximum-Likelihood-Schätzer, UMVU-Schätzer, exponentielle Familien, lineare Modelle. Weitere Themen sind Ordnungsprinzipien zur Reduktion der Komplexität der Modelle (Suffizienz und Invarianz).

Statistische Methoden und Verfahren kommen nicht nur in den Naturwissenschaften und der Medizin, sondern in nahezu allen Bereichen zum Einsatz, in denen Daten erhoben und analysiert werden, so z. B. auch in den Wirtschaftswissenschaften (Ökonometrie) und Sozialwissenschaften (dort vor allem in der Psychologie). Im Rahmen dieser Vorlesung wird der Schwerpunkt aber weniger auf Anwendungen, sondern – wie der Name schon sagt – mehr auf der mathematisch fundierten Begründung der Verfahren liegen.
\subsubsection*{\large
    Literatur:
}
\begin{itemize}
\item C. Czado, T. Schmidt: \href{https://link.springer.com/book/10.1007/978-3-642-17261-8}{\emph{Mathematische Statistik}}, Springer, 2011.
\item E.L. Lehmann, J.P. Romano:\href{https://link.springer.com/book/10.1007/978-3-030-70578-7}{\emph{Testing Statistical Hypotheses (Fourth Edition)}}, Springer, 2022.
\item E.L. Lehmann, G. Casella: \href{https://link.springer.com/book/10.1007/b98854}{\emph{Theory of Point Estimation, Second Edition}}, Springer, 1998. 
\item  L. Rüschendorf: \href{https://link.springer.com/book/10.1007/978-3-642-41997-3}{\emph{Mathematische Statistik}}, Springer Spektrum, 2014. 
\item  M. J. Schervish: \href{https://link.springer.com/book/10.1007/978-1-4612-4250-5}{\emph{Theory of Statistics}}, Springer, 1995.
\item J. Shao:  \href{https://link.springer.com/book/10.1007/b97553}{\emph{Mathematical Statistics}}, Springer, 2003.
\item H. Witting: \emph{Mathematische Statistik I}, Teubner, 1985.
\end{itemize}
\subsubsection*{\large
    Vorkenntnisse:
}
Wahrscheinlichkeitstheorie (insbesondere Maßtheorie sowie bedingte Wahrscheinlichkeiten und Erwartungen)
\subsubsection*{\large
    Bemerkungen:
}
Dieser Kurs wird auf Englisch angeboten.
\subsubsection*{\large
    Verwendbarkeit, Studien- und Prüfungsleistungen:
}

\begin{tabularx}{\textwidth}{ p{.5\textwidth}
    |X
    |X
    |X
}
 &
\makecell[c]{\rotatebox[origin=l]{90}{\parbox{
            4
            cm}{\begin{flushleft}
                Advanced Lecture in Stochastics (MScData24) (11.0 ECTS) \newline Angewandte Mathematik (MSc14) (11.0 ECTS) \newline Elective in Data (MScData24) (11.0 ECTS) \newline Mathematik (MSc14) (11.0 ECTS) \newline Mathematische Vertiefung (MEd18, MEH21) (9.0 ECTS) \newline Wahlpflichtmodul Mathematik (BSc21) (9.0 ECTS)
            \end{flushleft} }}}
 &
\makecell[c]{\rotatebox[origin=l]{90}{\parbox{
            4
            cm}{\begin{flushleft}
                Teil des Vertiefungsmoduls (MSc14) (10.5 ECTS)
            \end{flushleft} }}}
 &
\makecell[c]{\rotatebox[origin=l]{90}{\parbox{
            4
            cm}{\begin{flushleft}
                Wahlmodul (MSc14) (9.0 ECTS) \newline Wahlmodul (Option ''Individuelle Studiengestaltung'') (2HfB21) (9.0 ECTS)
            \end{flushleft} }}}
\\
& \Var{veranstaltung["verwendbarkeit"].columns.index(y)}
& \Var{veranstaltung["verwendbarkeit"].columns.index(y)}
& \Var{veranstaltung["verwendbarkeit"].columns.index(y)}
\\[2ex] \hline
\hline \rule[0mm]{0cm}{.6cm}PL: Mündliche Prüfung (Dauer: ca. 30 Minuten). \rule[-3mm]{0cm}{0cm}
 &
\makecell[c]{\xmark}
 &
 &
\\
\hline \rule[0mm]{0cm}{.6cm}PL: Mündliche Prüfung über alle Teile des Moduls (Dauer:  45 Minuten) \rule[-3mm]{0cm}{0cm}
 &
 &
\makecell[c]{\xmark}
 &
\\
\hline \rule[0mm]{0cm}{.6cm}SL: Bestehen eines mündlichen Abschlusstests. \rule[-3mm]{0cm}{0cm}
 &
 &
 &
\makecell[c]{\xmark}
\\
\hline \rule[0mm]{0cm}{.6cm}SL: Erreichen von mindestens 50\% der Punkte, die insgesamt durch die Bearbeitung der für die Übung ausgegebenen Übungsaufgaben erreicht werden können. \rule[-3mm]{0cm}{0cm}
 &
\makecell[c]{\xmark}
 &
\makecell[c]{\xmark}
 &
\makecell[c]{\xmark}
\\
\end{tabularx}




\clearpage\hrule\vskip1pt\hrule
\section*{\Large \href{https://pfaffelh.github.io/hp/2024WS_stochastic_processes.html}{Wahrscheinlichkeitstheorie II (Stochastische Prozesse)}}
\addcontentsline{toc}{subsection}{Wahrscheinlichkeitstheorie II (Stochastische Prozesse)\ \textcolor{gray}{(\em Peter Pfaffelhuber)}}
\vskip-2ex
{\it Peter Pfaffelhuber, Assistenz: Samuel Adeosun}
\hfill{E, 9.0 ECTS}\\
Vorlesung: Mo, 10--12 Uhr, HS II, \href{https://www.openstreetmap.org/?mlat=48.002320&mlon=7.847924#map=19/48.002320/7.847924}{Albertstr. 23b}\\
Übung (2-stündig): Mi, 12--14 Uhr, SR 127, \href{https://www.openstreetmap.org/?mlat=48.000637&mlon=7.846006#map=19/48.000636/7.846006}{Ernst-Zermelo-Str. 1}\\
Vorlesung (4-stündig): asynchrone Videos \\
% Webseite: \url{https://pfaffelh.github.io/hp/2024WS_stochastic_processes.html}
\subsubsection*{\large
    Inhalt:
}
Ein stochastischer Prozess $(X_t)_{t\in I}$ ist nichts weiter als eine Familie von Zufallsvariablen, wobei etwa $I = [0,\infty)$ eine kontinuierliche Zeitmenge ist. Einfache Beispiele sind Irrfahrten, Markov-Ketten, die Brown’sche Bewegung oder davon abgeleitete Prozesse. Letztere spielen vor allem in der Modellierung von finanzmathematischen oder naturwissenschaftlichen Fragestellungen eine große Rolle. Wir werden zunächst Martingale behandeln, die in allgemeiner Form faire Spiele beschreiben. Nach der Konstruktion des Poisson-Prozesses und der Brown’sche Bewegung konstruieren, werden wir uns auf Eigenschaften der Brown'schen Bewegung konzentriieren. Infinitesimale Charakteristiken eines Markov-Prozesses werden durch Generatoren beschrieben, was eine Verbindung zur Theorie von partiellen Differentialgleichungen ermöglicht. Abschließend kommt mit dem Ergodensatz fur stationäre stochastische Prozesse eine Verallgemeinerung des Gesetzes der großen Zahlen zur Sprache. Weiter werden Einblicke in ein paar Anwendungsgebiete, etwa Biomathematik oder zufällige Graphen gegeben. 
\subsubsection*{\large
    Literatur:
}
\begin{itemize}
\item
 O. Kallenberg: \href{https://link.springer.com/book/10.1007/978-3-030-61871-1}{\emph{Foundations of Modern Probability}} (Third Edition), Springer, 2021.
\item
 A. Klenke: \href{https://link.springer.com/book/10.1007/978-3-662-62089-2}{\emph{Wahrscheinlichkeitstheorie}} (4. Auf"|lage), Springer, 2020. 
\item 
D. Williams: \href{https://edisciplinas.usp.br/pluginfile.php/343758/mod_folder/content/0/Probability With Martingales(Williams).pdf}{\emph{Probability with Martingales}}, Cambridge University Press, 1991. 
\end{itemize}
\subsubsection*{\large
    Vorkenntnisse:
}
Wahrscheinlichkeitstheorie~I
\subsubsection*{\large
    Bemerkungen:
}
Die Vorlesung schließt direkt an die Vorlesung {\em Wahrscheinlichkeitstheorie} aus dem Sommersemester 2024 an. Im Sommersemester 2025 wird diese Veranstaltung durch die Vorlesung {\em Wahrscheinlichkeitstheorie~III (Stochastische Analysis)} fortgeführt.
\subsubsection*{\large
    Verwendbarkeit, Studien- und Prüfungsleistungen:
}

\begin{tabularx}{\textwidth}{ p{.5\textwidth}
    |X
    |X
    |X
}
 &
\makecell[c]{\rotatebox[origin=l]{90}{\parbox{
            4
            cm}{\begin{flushleft}
                Advanced Lecture in Stochastics (MScData24) (11.0 ECTS) \newline Angewandte Mathematik (MSc14) (11.0 ECTS) \newline Elective in Data (MScData24) (11.0 ECTS) \newline Mathematik (MSc14) (11.0 ECTS) \newline Mathematische Vertiefung (MEd18, MEH21) (9.0 ECTS) \newline Wahlpflichtmodul Mathematik (BSc21) (9.0 ECTS)
            \end{flushleft} }}}
 &
\makecell[c]{\rotatebox[origin=l]{90}{\parbox{
            4
            cm}{\begin{flushleft}
                Teil des Vertiefungsmoduls (MSc14) (10.5 ECTS)
            \end{flushleft} }}}
 &
\makecell[c]{\rotatebox[origin=l]{90}{\parbox{
            4
            cm}{\begin{flushleft}
                Wahlmodul (MSc14) (9.0 ECTS) \newline Wahlmodul (Option ''Individuelle Studiengestaltung'') (2HfB21) (9.0 ECTS)
            \end{flushleft} }}}
\\
& \Var{veranstaltung["verwendbarkeit"].columns.index(y)}
& \Var{veranstaltung["verwendbarkeit"].columns.index(y)}
& \Var{veranstaltung["verwendbarkeit"].columns.index(y)}
\\[2ex] \hline
\hline \rule[0mm]{0cm}{.6cm}PL: Mündliche Prüfung (Dauer: ca. 30 Minuten). \rule[-3mm]{0cm}{0cm}
 &
\makecell[c]{\xmark}
 &
 &
\\
\hline \rule[0mm]{0cm}{.6cm}PL: Mündliche Prüfung über alle Teile des Moduls (Dauer:  45 Minuten) \rule[-3mm]{0cm}{0cm}
 &
 &
\makecell[c]{\xmark}
 &
\\
\hline \rule[0mm]{0cm}{.6cm}SL: Bestehen eines mündlichen Abschlusstests. \rule[-3mm]{0cm}{0cm}
 &
 &
 &
\makecell[c]{\xmark}
\\
\hline \rule[0mm]{0cm}{.6cm}SL: Erreichen von mindestens 50\% der Punkte, die insgesamt durch die Bearbeitung der für die Übung ausgegebenen Übungsaufgaben erreicht werden können. \rule[-3mm]{0cm}{0cm}
 &
\makecell[c]{\xmark}
 &
\makecell[c]{\xmark}
 &
\makecell[c]{\xmark}
\\
\end{tabularx}




\clearpage\hrule\vskip1pt\hrule
\section*{\Large Wahrscheinlichkeitstheorie III (Stochastische Integration und Finanzmathematik)}
\addcontentsline{toc}{subsection}{Wahrscheinlichkeitstheorie III (Stochastische Integration und Finanzmathematik)\ \textcolor{gray}{(\em Thorsten Schmidt)}}
\vskip-2ex
{\it Thorsten Schmidt, Assistenz: Moritz Ritter}
\hfill{E, 9.0 ECTS}\\
Vorlesung: Mo, Mi, 12--14 Uhr, SR 404, \href{https://www.openstreetmap.org/?mlat=48.000637&mlon=7.846006#map=19/48.000636/7.846006}{Ernst-Zermelo-Str. 1}\\
Übung: 2-stündig, Termin wird noch festgelegt \\
\subsubsection*{\large
    Inhalt:
}
Diese Vorlesung bildet den Höhepunkt unserer Reihe zur Wahrscheinlichkeitstheorie und erreicht das ultimative Ziel dieser Reihe: Die Kombination von stochastischer Analysis und Finanzmathematik, ein Gebiet, das seit den 1990er Jahren eine erstaunliche Fülle von faszinierenden Ergebnissen hervorgebracht hat. Der Kern ist sicherlich die Anwendung der Semi-Martingale-Theorie auf die Finanzmärkte, die in dem fundamentalen Theorem der Preisbildung von Vermögenswerten kummulieren. Dieses Ergebnis wird überall auf den Finanzmärkten verwendet. Danach befassen wir uns mit modernen Formen der stochastischen Analysis, die neuronale SDEs, Signaturmethoden, Unsicherheits- und Terminstrukturmodelle. Die Vorlesung schließt mit einer Untersuchung der neuesten Anwendungen von maschinellem Lernen auf den Finanzmärkten und dem wechselseitigen Einfluss der stochastischen Analyse auf maschinelles Lernen ab.
\subsubsection*{\large
    Literatur:
}
Relevante Literatur wird in der Vorlesung bekannt gegeben.
\subsubsection*{\large
    Vorkenntnisse:
}
Wahrscheinlichkeitstheorie II (Stochastische Prozesse)
\subsubsection*{\large
    Bemerkungen:
}
Diese Vorlesung wird auf Englisch angeboten
\subsubsection*{\large
    Verwendbarkeit, Studien- und Prüfungsleistungen:
}

\begin{tabularx}{\textwidth}{ p{.5\textwidth}
    |X
    |X
    |X
}
 &
\makecell[c]{\rotatebox[origin=l]{90}{\parbox{
            4
            cm}{\begin{flushleft}
                Advanced Lecture in Stochastics (MScData24) (11.0 ECTS) \newline Angewandte Mathematik (MSc14) (11.0 ECTS) \newline Elective in Data (MScData24) (11.0 ECTS) \newline Mathematik (MSc14) (11.0 ECTS) \newline Mathematische Vertiefung (MEd18, MEH21) (9.0 ECTS) \newline Wahlpflichtmodul Mathematik (BSc21) (9.0 ECTS)
            \end{flushleft} }}}
 &
\makecell[c]{\rotatebox[origin=l]{90}{\parbox{
            4
            cm}{\begin{flushleft}
                Teil des Vertiefungsmoduls (MSc14) (10.5 ECTS)
            \end{flushleft} }}}
 &
\makecell[c]{\rotatebox[origin=l]{90}{\parbox{
            4
            cm}{\begin{flushleft}
                Wahlmodul (MSc14) (9.0 ECTS) \newline Wahlmodul (Option ''Individuelle Studiengestaltung'') (2HfB21) (9.0 ECTS)
            \end{flushleft} }}}
\\
& \Var{veranstaltung["verwendbarkeit"].columns.index(y)}
& \Var{veranstaltung["verwendbarkeit"].columns.index(y)}
& \Var{veranstaltung["verwendbarkeit"].columns.index(y)}
\\[2ex] \hline
\hline \rule[0mm]{0cm}{.6cm}PL: Mündliche Prüfung (Dauer: ca. 30 Minuten). \rule[-3mm]{0cm}{0cm}
 &
\makecell[c]{\xmark}
 &
 &
\\
\hline \rule[0mm]{0cm}{.6cm}PL: Mündliche Prüfung über alle Teile des Moduls (Dauer:  45 Minuten) \rule[-3mm]{0cm}{0cm}
 &
 &
\makecell[c]{\xmark}
 &
\\
\hline \rule[0mm]{0cm}{.6cm}SL: Bestehen eines mündlichen Abschlusstests. \rule[-3mm]{0cm}{0cm}
 &
 &
 &
\makecell[c]{\xmark}
\\
\hline \rule[0mm]{0cm}{.6cm}SL: Erreichen von mindestens 50\% der Punkte, die insgesamt durch die Bearbeitung der für die Übung ausgegebenen Übungsaufgaben erreicht werden können. \rule[-3mm]{0cm}{0cm}
 &
\makecell[c]{\xmark}
 &
\makecell[c]{\xmark}
 &
\makecell[c]{\xmark}
\\
\end{tabularx}




\clearpage\hrule\vskip1pt\hrule
\section*{\Large \href{https://home.mathematik.uni-freiburg.de/arithgeom/lehre/ws24/semialg.html}{Semi-algebraische Geometrie}}
\addcontentsline{toc}{subsection}{Semi-algebraische Geometrie\ \textcolor{gray}{(\em Annette Huber-Klawitter, Amador Martín Pizarro)}}
\vskip-2ex
{\it Annette Huber-Klawitter, Amador Martín Pizarro, Assistenz: Christoph Brackenhofer}
\hfill{D, 9.0 ECTS}\\
Vorlesung: Di, Do, 10--12 Uhr, HS II, \href{https://www.openstreetmap.org/?mlat=48.002320&mlon=7.847924#map=19/48.002320/7.847924}{Albertstr. 23b}\\
Übung: 2-stündig, Termin wird noch festgelegt \\
% Webseite: \url{https://home.mathematik.uni-freiburg.de/arithgeom/lehre/ws24/semialg.html}
\subsubsection*{\large
    Inhalt:
}
In der semi-algebraischen Geometrie geht es um Eigenschaften von Teilmengen von $\textbf{R}^n$, die durch Ungleichungen der Form
\[ f(x_1,\dots,x_n)\geq 0\]
für Polynome $f\in\textbf{R}[X_1,\dots,X_n]$ definiert werden. 

Die Theorie hat sehr unterschiedliche Gesichter. Einerseits kann sie als eine Version von algebraischer Geometrie über $\mathbf{R}$ (oder noch allgemeiner über sogenannten reell abgeschlossenen Körpern) gesehen werden. Andererseits sind die Eigenschaften dieser Körper ein zentrales Hilfsmittel für den modelltheoretischen Beweis des Satzes von Tarski-Seidenberg der Quantorenelimination in reell abgeschlossenen Körpern. Geometrisch wird dieser als Projektionssatz interpretiert.

Aus diesem Satz folgt leicht ein Beweis des Hilbert’schen 17. Problems, welches 1926 von Artin bewiesen wurde. 

\textit{Ist jedes reelle Polynom $P \in \mathbf{R}[x_1 ,\dots , x_n ]$, welches an jedem n-Tupel aus $\mathbf{R}^n$ einen nicht-negativen Wert annnimmt, eine Summe von Quadraten rationaler Funktionen (d.h. Quotienten von Polynomen)?}

In der Vorlesung wollen wir beide Aspekte kennenlernen. Nötige Hilfsmittel aus der kommutativen Algebra oder Modelltheorie werden entsprechend den Vorkenntnissen der Hörer:innen besprochen. 
\subsubsection*{\large
    Literatur:
}
\begin{itemize}
\item A.~Prestel: Vorlesungsskript \href{http://www.math.uni-konstanz.de/\~prestel/raskript.pdf}{\emph{Reelle Algebra}}.
\item
L.~van den Dries: \emph{Tame topology and o-minimal structures}, London Mathematical Society Lecture Note Series, Cambridge University Press, 1998. 
\item
Jacek Bochnak, Michel Coste \& Marie-Françoise Roy: \emph{Real Algebra}, Ergebnisse der Mathematik und ihrer Grenzgebiete 36, Springer Verlag, 1998.
\end{itemize}
\subsubsection*{\large
    Vorkenntnisse:
}
Notwendig: Algebra und Zahlentheorie \\
Nützlich: Kommutative Algebra und Einführung in die algebraische Geometrie, Modelltheorie
\subsubsection*{\large
    Bemerkungen:
}
This course is only offered in German.
\subsubsection*{\large
    Verwendbarkeit, Studien- und Prüfungsleistungen:
}

\begin{tabularx}{\textwidth}{ p{.5\textwidth}
    |X
    |X
    |X
}
 &
\makecell[c]{\rotatebox[origin=l]{90}{\parbox{
            4
            cm}{\begin{flushleft}
                Mathematik (MSc14) (11.0 ECTS) \newline Mathematische Vertiefung (MEd18, MEH21) (9.0 ECTS) \newline Reine Mathematik (MSc14) (11.0 ECTS) \newline Wahlpflichtmodul Mathematik (BSc21) (9.0 ECTS)
            \end{flushleft} }}}
 &
\makecell[c]{\rotatebox[origin=l]{90}{\parbox{
            4
            cm}{\begin{flushleft}
                Teil des Vertiefungsmoduls (MSc14) (10.5 ECTS)
            \end{flushleft} }}}
 &
\makecell[c]{\rotatebox[origin=l]{90}{\parbox{
            4
            cm}{\begin{flushleft}
                Wahlmodul (MSc14) (9.0 ECTS) \newline Wahlmodul (MScData24) (9.0 ECTS) \newline Wahlmodul (Option ''Individuelle Studiengestaltung'') (2HfB21) (9.0 ECTS)
            \end{flushleft} }}}
\\
& \Var{veranstaltung["verwendbarkeit"].columns.index(y)}
& \Var{veranstaltung["verwendbarkeit"].columns.index(y)}
& \Var{veranstaltung["verwendbarkeit"].columns.index(y)}
\\[2ex] \hline
\hline \rule[0mm]{0cm}{.6cm}PL: Mündliche Prüfung (Dauer: ca. 30 Minuten). \rule[-3mm]{0cm}{0cm}
 &
\makecell[c]{\xmark}
 &
 &
\\
\hline \rule[0mm]{0cm}{.6cm}PL: Mündliche Prüfung über alle Teile des Moduls (Dauer:  45 Minuten) \rule[-3mm]{0cm}{0cm}
 &
 &
\makecell[c]{\xmark}
 &
\\
\hline \rule[0mm]{0cm}{.6cm}SL: Bestehen eines mündlichen Abschlusstests. \rule[-3mm]{0cm}{0cm}
 &
 &
 &
\makecell[c]{\xmark}
\\
\hline \rule[0mm]{0cm}{.6cm}SL: Erreichen von mindestens 50\% der Punkte, die insgesamt durch die Bearbeitung der für die Übung ausgegebenen Übungsaufgaben erreicht werden können. \rule[-3mm]{0cm}{0cm}
 &
\makecell[c]{\xmark}
 &
\makecell[c]{\xmark}
 &
\makecell[c]{\xmark}
\\
\end{tabularx}




\clearpage\hrule\vskip1pt\hrule
\section*{\Large Mengenlehre: Unabhängigkeitsbeweise}
\addcontentsline{toc}{subsection}{Mengenlehre: Unabhängigkeitsbeweise\ \textcolor{gray}{(\em Maxwell Levine)}}
\vskip-2ex
{\it Maxwell Levine, Assistenz: Hannes Jakob}
\hfill{E, 9.0 ECTS}\\
Vorlesung: Di, Do, 12--14 Uhr, SR 404, \href{https://www.openstreetmap.org/?mlat=48.000637&mlon=7.846006#map=19/48.000636/7.846006}{Ernst-Zermelo-Str. 1}\\
Übung: 2-stündig, Termin wird noch festgelegt \\
\subsubsection*{\large
    Inhalt:
}
How does one prove that something cannot be proved? More precisely, how does one prove that a particular statement does not follow from a particular collection of axioms?

These questions are often asked with respect to the axioms most commonly used by mathematicians: the axioms of Zermelo-Fraenkel set theory, or ZFC for short. In this course, we will develop the conceptual tools needed to understand independence proofs with respect to ZFC. On the way we will develop the theory of ordinal and cardinal numbers, the basics of inner model theory, and the method of forcing. In particular, we will show that Cantor's continuum hypothesis, the statement that $2^{\aleph_0}=\aleph_1$, is independent of ZFC. 

\subsubsection*{\large
    Literatur:
}
\begin{itemize}
\item Thomas Jech: \emph{Set Theory}. The Third Millenium Edition, Springer, 2001. 
\item Kenneth Kunen: \emph{Set Theory: An Introduction to Independence Proofs}. North-Holland Pub. Co, 1980.
\end{itemize}

\subsubsection*{\large
    Vorkenntnisse:
}
Mathematische Logik
\subsubsection*{\large
    Bemerkungen:
}
Dieser Kurs wird auf Englisch angeboten.
\subsubsection*{\large
    Verwendbarkeit, Studien- und Prüfungsleistungen:
}

\begin{tabularx}{\textwidth}{ p{.5\textwidth}
    |X
    |X
    |X
}
 &
\makecell[c]{\rotatebox[origin=l]{90}{\parbox{
            4
            cm}{\begin{flushleft}
                Mathematik (MSc14) (11.0 ECTS) \newline Mathematische Vertiefung (MEd18, MEH21) (9.0 ECTS) \newline Reine Mathematik (MSc14) (11.0 ECTS) \newline Wahlpflichtmodul Mathematik (BSc21) (9.0 ECTS)
            \end{flushleft} }}}
 &
\makecell[c]{\rotatebox[origin=l]{90}{\parbox{
            4
            cm}{\begin{flushleft}
                Teil des Vertiefungsmoduls (MSc14) (10.5 ECTS)
            \end{flushleft} }}}
 &
\makecell[c]{\rotatebox[origin=l]{90}{\parbox{
            4
            cm}{\begin{flushleft}
                Wahlmodul (MSc14) (9.0 ECTS) \newline Wahlmodul (MScData24) (9.0 ECTS) \newline Wahlmodul (Option ''Individuelle Studiengestaltung'') (2HfB21) (9.0 ECTS)
            \end{flushleft} }}}
\\
& \Var{veranstaltung["verwendbarkeit"].columns.index(y)}
& \Var{veranstaltung["verwendbarkeit"].columns.index(y)}
& \Var{veranstaltung["verwendbarkeit"].columns.index(y)}
\\[2ex] \hline
\hline \rule[0mm]{0cm}{.6cm}PL: Mündliche Prüfung (Dauer: ca. 30 Minuten). \rule[-3mm]{0cm}{0cm}
 &
\makecell[c]{\xmark}
 &
 &
\\
\hline \rule[0mm]{0cm}{.6cm}PL: Mündliche Prüfung über alle Teile des Moduls (Dauer:  45 Minuten) \rule[-3mm]{0cm}{0cm}
 &
 &
\makecell[c]{\xmark}
 &
\\
\hline \rule[0mm]{0cm}{.6cm}SL: Bestehen eines mündlichen Abschlusstests. \rule[-3mm]{0cm}{0cm}
 &
 &
 &
\makecell[c]{\xmark}
\\
\hline \rule[0mm]{0cm}{.6cm}SL: Erreichen von mindestens 50\% der Punkte, die insgesamt durch die Bearbeitung der für die Übung ausgegebenen Übungsaufgaben erreicht werden können. \rule[-3mm]{0cm}{0cm}
 &
\makecell[c]{\xmark}
 &
\makecell[c]{\xmark}
 &
\makecell[c]{\xmark}
\\
\end{tabularx}




\clearpage\hrule\vskip1pt\hrule
\section*{\Large Theorie und Numerik for Partieller Differentialgleichungen – Nichtlineare Probleme}
\addcontentsline{toc}{subsection}{Theorie und Numerik for Partieller Differentialgleichungen – Nichtlineare Probleme\ \textcolor{gray}{(\em Sören Bartels, Patrick Dondl)}}
\vskip-2ex
{\it Sören Bartels, Patrick Dondl}
\hfill{E, 9.0 ECTS}\\
Vorlesung: 4-stündig \newline als Lesekurs, Termine nach Vereinbarung\\
Übung: 2-stündig, Termin wird noch festgelegt \\
\subsubsection*{\large
    Inhalt:
}
Die Vorlesung behandelt die Entwicklung und Analyse von numerischen Methoden für die Approximation bestimmter nichtlinearer partieller Differentialgleichungen. Zu den betrachteten Modellproblemen gehören harmonische Abbildungen in Sphären, total-variable regulierte Minimierungsprobleme und nichtlineare Krümmungsmodelle. Für jedes der Probleme wird eine geeignete Finite-Elemente-Diskretisierung entwickelt, ihre Konvergenz wird analysiert und iterative Lösungsverfahren werden entwickelt. Die Vorlesung wird durch theoretische und praktische Übungen ergänzt, in denen die Ergebnisse vertieft und experimentell überprüft werden.
\subsubsection*{\large
    Literatur:
}
\begin{itemize}
\item
S. Bartels: \emph{Numerical methods for nonlinear partial differential equations}, Springer, 2015.
\item
M. Dobrowolski: \emph{Angewandte Funktionalanalysis}, Springer, 2010.
\item
L.C. Evans: \emph{Partial Differential Equations}, 2nd Edition, 2010. 
\end{itemize}
\subsubsection*{\large
    Vorkenntnisse:
}
Einführung in Theorie und Numerik partieller Differetialgleichungen oder Einführung in partielle Differetialgleichungen
\subsubsection*{\large
    Bemerkungen:
}
Die Vorlesung findet in Form eines Lesekurses statt. \\ Diese Vorlesung findet auf Englisch statt.
\subsubsection*{\large
    Verwendbarkeit, Studien- und Prüfungsleistungen:
}

\begin{tabularx}{\textwidth}{ p{.5\textwidth}
    |X
    |X
    |X
}
 &
\makecell[c]{\rotatebox[origin=l]{90}{\parbox{
            4
            cm}{\begin{flushleft}
                Advanced Lecture in Numerics (MScData24) (11.0 ECTS) \newline Angewandte Mathematik (MSc14) (11.0 ECTS) \newline Elective in Data (MScData24) (11.0 ECTS) \newline Mathematik (MSc14) (11.0 ECTS) \newline Mathematische Vertiefung (MEd18, MEH21) (9.0 ECTS) \newline Wahlpflichtmodul Mathematik (BSc21) (9.0 ECTS)
            \end{flushleft} }}}
 &
\makecell[c]{\rotatebox[origin=l]{90}{\parbox{
            4
            cm}{\begin{flushleft}
                Teil des Vertiefungsmoduls (MSc14) (10.5 ECTS)
            \end{flushleft} }}}
 &
\makecell[c]{\rotatebox[origin=l]{90}{\parbox{
            4
            cm}{\begin{flushleft}
                Wahlmodul (MSc14) (9.0 ECTS) \newline Wahlmodul (Option ''Individuelle Studiengestaltung'') (2HfB21) (9.0 ECTS)
            \end{flushleft} }}}
\\
& \Var{veranstaltung["verwendbarkeit"].columns.index(y)}
& \Var{veranstaltung["verwendbarkeit"].columns.index(y)}
& \Var{veranstaltung["verwendbarkeit"].columns.index(y)}
\\[2ex] \hline
\hline \rule[0mm]{0cm}{.6cm}PL: Mündliche Prüfung (Dauer: ca. 30 Minuten). \rule[-3mm]{0cm}{0cm}
 &
\makecell[c]{\xmark}
 &
 &
\\
\hline \rule[0mm]{0cm}{.6cm}PL: Mündliche Prüfung über alle Teile des Moduls (Dauer:  45 Minuten) \rule[-3mm]{0cm}{0cm}
 &
 &
\makecell[c]{\xmark}
 &
\\
\hline \rule[0mm]{0cm}{.6cm}SL: Bestehen eines mündlichen Abschlusstests. \rule[-3mm]{0cm}{0cm}
 &
 &
 &
\makecell[c]{\xmark}
\\
\hline \rule[0mm]{0cm}{.6cm}SL: Erreichen von mindestens 50\% der Punkte, die insgesamt durch die Bearbeitung der für die Übung ausgegebenen Übungsaufgaben erreicht werden können. \rule[-3mm]{0cm}{0cm}
 &
\makecell[c]{\xmark}
 &
\makecell[c]{\xmark}
 &
\makecell[c]{\xmark}
\\
\end{tabularx}




\clearpage\hrule\vskip1pt\hrule
\section*{\Large Lesekurse „Wissenschaftliches Arbeiten“ }
\addcontentsline{toc}{subsection}{Lesekurse „Wissenschaftliches Arbeiten“ \ \textcolor{gray}{(\em Alle Dozent:inn:en der Mathematik)}}
\vskip-2ex
{\it Alle Dozent:inn:en der Mathematik}
\hfill{D/E, 9.0 ECTS}\\
Termine nach Vereinbarung \\
\subsubsection*{\large
    Inhalt:
}
In einem Lesekurs wird der Stoff einer vierstündigen Vorlesung im betreuten Selbststudium erarbeitet. In seltenen Fällen kann dies im Rahmen einer Veranstaltung stattfinden; üblicherweise werden die Lesekurse aber nicht im Vorlesungsverzeichnis angekündigt. Bei Interesse nehmen Sie vor Vorlesungsbeginn Kontakt mit einer Professorin/einem Professor bzw. einer Privatdozentin/einem Privatdozenten auf; in der Regel wird es sich um die Betreuerin/den Betreuer der Master-Arbeit handeln, da der Lesekurs im Idealfall als Vorbereitung auf die Master-Arbeit dient (im M.Sc. wie im M.Ed.).

Der Inhalt des Lesekurses, die näheren Umstände sowie die Konkretisierung der zu erbringenden Studienleistungen werden zu Beginn der Vorlesungszeit von der Betreuerin/dem Betreuer festgelegt. Die Arbeitsbelastung sollte der einer vierstündigen Vorlesung mit Übungen entsprechen.
\subsubsection*{\large
    Verwendbarkeit, Studien- und Prüfungsleistungen:
}

\begin{tabularx}{\textwidth}{ p{.5\textwidth}
    |X
    |X
    |X
}
 &
\makecell[c]{\rotatebox[origin=l]{90}{\parbox{
            4
            cm}{\begin{flushleft}
                Mathematik (MSc14) (11.0 ECTS) \newline Wissenschaftliches Arbeiten (MEd18, MEH21) (9.0 ECTS)
            \end{flushleft} }}}
 &
\makecell[c]{\rotatebox[origin=l]{90}{\parbox{
            4
            cm}{\begin{flushleft}
                Teil des Vertiefungsmoduls (MSc14) (10.5 ECTS)
            \end{flushleft} }}}
 &
\makecell[c]{\rotatebox[origin=l]{90}{\parbox{
            4
            cm}{\begin{flushleft}
                Wahlmodul (MSc14) (9.0 ECTS)
            \end{flushleft} }}}
\\
& \Var{veranstaltung["verwendbarkeit"].columns.index(y)}
& \Var{veranstaltung["verwendbarkeit"].columns.index(y)}
& \Var{veranstaltung["verwendbarkeit"].columns.index(y)}
\\[2ex] \hline
\hline \rule[0mm]{0cm}{.6cm}PL: Mündliche Prüfung (Dauer: ca. 30 Minuten). \rule[-3mm]{0cm}{0cm}
 &
\makecell[c]{\xmark}
 &
 &
\\
\hline \rule[0mm]{0cm}{.6cm}PL: Mündliche Prüfung über alle Teile des Moduls (Dauer:  45 Minuten) \rule[-3mm]{0cm}{0cm}
 &
 &
\makecell[c]{\xmark}
 &
\\
\hline \rule[0mm]{0cm}{.6cm}SL: Selbständige Lektüre der von dem Betreuer/der Betreuerin vorgegebenen Skripte, Artikel oder Buchkapitel und ggf. Bearbeitung von begleitenden Übungsaufgaben.
Regelmäßiger Bericht über den Fortschritt des Selbststudiums mit der Formulierung von Fragen zu nicht verstandenen Punkten.
Bis zu zweimaliges Vortragen vor der Arbeitsgruppe über den bisher erarbeiten Stoff, ggf. im Rahmen eines Seminars, Projekt- oder Oberseminars.
Falls das Wissenschaftliche Arbeiten im Rahmen einer Lehrveranstaltung (z.B. Seminar oder Projektseminar) stattfindet: regelmäßige Teilnahme an dieser Veranstaltung. \rule[-3mm]{0cm}{0cm}
 &
\makecell[c]{\xmark}
 &
\makecell[c]{\xmark}
 &
\makecell[c]{\xmark}
\\
\end{tabularx}




\clearpage
\phantomsection
\thispagestyle{empty}
\vspace*{\fill}
\begin{center}
    \Huge\bfseries 1c. Weiterführende zweistündige Vorlesungen
\end{center}
\addcontentsline{toc}{section}{\textbf{1c. Weiterführende zweistündige Vorlesungen}}
\addtocontents{toc}{\medskip\hrule\medskip}\vspace*{\fill}\vspace*{\fill}\clearpage
\vfill
\thispagestyle{empty}
\clearpage

\clearpage\hrule\vskip1pt\hrule
\section*{\Large \href{ https://sites.google.com/view/xuwenzhang/teaching/functions-of-bounded-variation-sets-of-finite-perimeter}{Functions of Bounded Variation and Sets of Finite Perimeter}}
\addcontentsline{toc}{subsection}{Functions of Bounded Variation and Sets of Finite Perimeter\ \textcolor{gray}{(\em Xuwen Zhang)}}
\vskip-2ex
{\it Xuwen Zhang}
\hfill{E, 6.0 ECTS}\\
Vorlesung: Mo, 14--16 Uhr, SR 127, \href{https://www.openstreetmap.org/?mlat=48.000637&mlon=7.846006#map=19/48.000636/7.846006}{Ernst-Zermelo-Str. 1}\\
Übung: 2-stündig, Termin wird noch festgelegt \\
% Webseite: \url{ https://sites.google.com/view/xuwenzhang/teaching/functions-of-bounded-variation-sets-of-finite-perimeter}
\subsubsection*{\large
    Inhalt:
}
We will study functions of bounded variation, which are functions whose weak first
partial derivatives are Radon measures. This is essentially the weakest definition of a function to be
differentiable in the measure-theoretic sense. After discussing the basic properties of them, we move on
to the study of sets of finite perimeter, which are Lebesgue measurable sets in the Euclidean space whose
indicator functions are BV functions. Sets of finite perimeter are fundamental in the modern Calculus of
Variations as they generalize in a natural measure-theoretic way the notion of sets with regular boundaries
and possess nice compactness, thus appearing in many Geometric Variational problems. If time permits,
we will discuss the (capillary) sessile drop problem as one important application.
\subsubsection*{\large
    Literatur:
}
• Evans, Lawrence C. and Gariepy, Ronald F. Measure theory and fine properties of functions.
CRC Press, Boca Raton, FL, 2015.
• Maggi, Francesco. Sets of finite perimeter and geometric variational problems: an introduction
to Geometric Measure Theory. No. 135. Cambridge University Press, 2012.
\subsubsection*{\large
    Vorkenntnisse:
}
Grundkenntnisse in Maßtheorien und Analysis.
\subsubsection*{\large
    Verwendbarkeit, Studien- und Prüfungsleistungen:
}

\begin{tabularx}{\textwidth}{ p{.5\textwidth}
    |X
    |X
    |X
    |X
}
 &
\makecell[c]{\rotatebox[origin=l]{90}{\parbox{
            8
            cm}{\begin{flushleft}
                Teil des Moduls ''Mathematik'' (MSc14) (5.5 ECTS) \newline Teil des Moduls ''Reine Mathematik'' (MSc14) (5.5 ECTS)
            \end{flushleft} }}}
 &
\makecell[c]{\rotatebox[origin=l]{90}{\parbox{
            8
            cm}{\begin{flushleft}
                Teil des Vertiefungsmoduls (MSc14) (5.25 ECTS)
            \end{flushleft} }}}
 &
\makecell[c]{\rotatebox[origin=l]{90}{\parbox{
            8
            cm}{\begin{flushleft}
                Wahlmodul (MSc14) (6.0 ECTS) \newline Wahlmodul (MScData24) (6.0 ECTS) \newline Wahlmodul (Option ''Individuelle Studiengestaltung'') (2HfB21) (6.0 ECTS)
            \end{flushleft} }}}
 &
\makecell[c]{\rotatebox[origin=l]{90}{\parbox{
            8
            cm}{\begin{flushleft}
                Wahlpflichtmodul Mathematik (BSc21) (6.0 ECTS)
            \end{flushleft} }}}
\\
& \Var{veranstaltung["verwendbarkeit"].columns.index(y)}
& \Var{veranstaltung["verwendbarkeit"].columns.index(y)}
& \Var{veranstaltung["verwendbarkeit"].columns.index(y)}
& \Var{veranstaltung["verwendbarkeit"].columns.index(y)}
\\[2ex] \hline
\hline \rule[0mm]{0cm}{.6cm}PL: Mündliche Prüfung (Dauer: ca. 30 Minuten). \rule[-3mm]{0cm}{0cm}
 &
 &
 &
 &
\makecell[c]{\xmark}
\\
\hline \rule[0mm]{0cm}{.6cm}PL: Mündliche Prüfung über alle Teile des Moduls (Dauer:  45 Minuten) \rule[-3mm]{0cm}{0cm}
 &
 &
\makecell[c]{\xmark}
 &
 &
\\
\hline \rule[0mm]{0cm}{.6cm}PL: Mündliche Prüfung über alle Teile des Moduls (Dauer: 30 Minuten) \rule[-3mm]{0cm}{0cm}
 &
\makecell[c]{\xmark}
 &
 &
 &
\\
\hline \rule[0mm]{0cm}{.6cm}SL: Bestehen eines mündlichen Abschlusstests. \rule[-3mm]{0cm}{0cm}
 &
 &
 &
\makecell[c]{\xmark}
 &
\\
\hline \rule[0mm]{0cm}{.6cm}SL: Erreichen von mindestens 50\% der Punkte, die insgesamt durch die Bearbeitung der für die Übung ausgegebenen Übungsaufgaben erreicht werden können. \rule[-3mm]{0cm}{0cm}
 &
\makecell[c]{\xmark}
 &
\makecell[c]{\xmark}
 &
\makecell[c]{\xmark}
 &
\makecell[c]{\xmark}
\\
\end{tabularx}




\clearpage\hrule\vskip1pt\hrule
\section*{\Large \href{https://www.finance.uni-freiburg.de/studium-und-lehre/ws2425/fao2425}{Futures and Options}}
\addcontentsline{toc}{subsection}{Futures and Options\ \textcolor{gray}{(\em Eva Lütkebohmert-Holtz)}}
\vskip-2ex
{\it Eva Lütkebohmert-Holtz, Assistenz: Hongyi Shen}
\hfill{E, 6.0 ECTS}\\
Vorlesung: Mo, 10--12 Uhr, HS 1098, \href{https://www.openstreetmap.org/?mlat=47.993706&mlon=7.846060#map=19/47.993706/7.846060}{KG I}\\
Übung: Do, 10--12 Uhr, HS 1098, \href{https://www.openstreetmap.org/?mlat=47.993706&mlon=7.846060#map=19/47.993706/7.846060}{KG I}\\
% Webseite: \url{https://www.finance.uni-freiburg.de/studium-und-lehre/ws2425/fao2425}
\subsubsection*{\large
    Inhalt:
}
Dieser Kurs bietet eine Einführung in die Finanzmärkte und -produkte. Neben Futures und Standard-Put- und Call-Optionen europäischer und amerikanischer Art werden auch zinssensitive Instrumente wie z.B. Swaps behandelt.
Für die Bewertung von Finanzderivaten führen wir zunächst Finanzmodelle in diskreter Zeit ein, wie das Cox-Ross-Rubinstein-Modell vor und erläutern die Grundprinzipien der risikoneutralen Bewertung. Schließlich diskutieren wir das berühmte Black-Scholes-Modell, das ein zeitkontinuierliches Modell für die Optionsbewertung darstellt.
\subsubsection*{\large
    Literatur:
}
\begin{itemize}
\item
D. M. Chance, R. Brooks: \emph{An Introduction to Derivatives and Risk Management} (10th edition), Cengage, 2016. 
\item
J. C. Hull: \emph{Options, Futures, and other Derivatives} (11th global edition), Pearson, 2021.
\item 
S. E. Shreve: \href{https://link.springer.com/book/10.1007/978-0-387-22527-2}{\emph{Stochastic Calculus for Finance I: The Binomial Asset Pricing Model}}, Springer, 2004. 
\item 
R. A. Strong: \emph{Derivatives. An Introduction} (Second edition), South-Western, 2004.
\end{itemize}
\subsubsection*{\large
    Vorkenntnisse:
}
Stochastik~I
\subsubsection*{\large
    Bemerkungen:
}
Diese Veranstaltung wird für das erste Jahr des {\em Finance profile} des M.Sc. Economics angeboten, sowie für Studierende im B.Sc.Mathematik, M.Sc. Mathematik, M.Sc. Mathematics in Data and Technology und M.Sc. Volkswirtschaftslehre. In der Spezialisierung in Finanzmathematik im M.Sc. Mathematik kann die Veranstaltung auch als wirtschaftswissenschaftliches Spezialisierungsmodul gelten. Studierenden im B.Sc. Mathematik mit Interesse an der Spezialisierung in Finanzmathematik wird daher empfohlen, die Veranstaltung für den M.Sc. aufzuheben.\\ Dieser Kurs wird auf Englisch angeboten.
\subsubsection*{\large
    Verwendbarkeit, Studien- und Prüfungsleistungen:
}

\begin{tabularx}{\textwidth}{ p{.5\textwidth}
    |X
    |X
    |X
    |X
}
 &
\makecell[c]{\rotatebox[origin=l]{90}{\parbox{
            8
            cm}{\begin{flushleft}
                Elective in Data (MScData24) (6.0 ECTS) \newline Wahlpflichtmodul Mathematik (BSc21) (6.0 ECTS)
            \end{flushleft} }}}
 &
\makecell[c]{\rotatebox[origin=l]{90}{\parbox{
            8
            cm}{\begin{flushleft}
                Mathematische Ergänzung (MEd18) (6.0 ECTS) \newline Wahlmodul (MSc14) (6.0 ECTS) \newline Wahlmodul (Option ''Individuelle Studiengestaltung'') (2HfB21) (6.0 ECTS)
            \end{flushleft} }}}
 &
\makecell[c]{\rotatebox[origin=l]{90}{\parbox{
            8
            cm}{\begin{flushleft}
                Teil des Moduls ''Angewandte Mathematik'' (MSc14) (5.5 ECTS) \newline Teil des Moduls ''Mathematik'' (MSc14) (5.5 ECTS)
            \end{flushleft} }}}
 &
\makecell[c]{\rotatebox[origin=l]{90}{\parbox{
            8
            cm}{\begin{flushleft}
                Teil des Vertiefungsmoduls (MSc14) (5.25 ECTS)
            \end{flushleft} }}}
\\
& \Var{veranstaltung["verwendbarkeit"].columns.index(y)}
& \Var{veranstaltung["verwendbarkeit"].columns.index(y)}
& \Var{veranstaltung["verwendbarkeit"].columns.index(y)}
& \Var{veranstaltung["verwendbarkeit"].columns.index(y)}
\\[2ex] \hline
\hline \rule[0mm]{0cm}{.6cm}PL: Klausur (Dauer: 1 bis 3 Stunden). \rule[-3mm]{0cm}{0cm}
 &
\makecell[c]{\xmark}
 &
 &
 &
\\
\hline \rule[0mm]{0cm}{.6cm}PL: Mündliche Prüfung über alle Teile des Moduls (Dauer:  45 Minuten) \rule[-3mm]{0cm}{0cm}
 &
 &
 &
 &
\makecell[c]{\xmark}
\\
\hline \rule[0mm]{0cm}{.6cm}PL: Mündliche Prüfung über alle Teile des Moduls (Dauer: 30 Minuten) \rule[-3mm]{0cm}{0cm}
 &
 &
 &
\makecell[c]{\xmark}
 &
\\
\hline \rule[0mm]{0cm}{.6cm}SL: Bestehen der Abschlussklausur (Dauer 1 bis 3 Stunden). \rule[-3mm]{0cm}{0cm}
 &
 &
\makecell[c]{\xmark}
 &
 &
\\
\end{tabularx}




\clearpage\hrule\vskip1pt\hrule
\section*{\Large \href{https://sites.google.com/view/maximilianstegemeyer/teaching/lie-gruppen-und-symmetrische-r\%C3\%A4ume-ws-2425}{Lie-Gruppen und symmetrische Räume}}
\addcontentsline{toc}{subsection}{Lie-Gruppen und symmetrische Räume\ \textcolor{gray}{(\em Maximilian Stegemeyer)}}
\vskip-2ex
{\it Maximilian Stegemeyer}
\hfill{D, 6.0 ECTS}\\
Vorlesung: Do, 14--16 Uhr, SR 404, \href{https://www.openstreetmap.org/?mlat=48.000637&mlon=7.846006#map=19/48.000636/7.846006}{Ernst-Zermelo-Str. 1}\\
Übung: 2-stündig, Termin wird noch festgelegt \\
% Webseite: \url{https://sites.google.com/view/maximilianstegemeyer/teaching/lie-gruppen-und-symmetrische-r\%C3\%A4ume-ws-2425}
\subsubsection*{\large
    Inhalt:
}
In der Geometrie und Topologie spielen Lie-Gruppen und Wirkungen von Lie-Gruppen eine zentrale Rolle. Mit ihnen lassen sich kontinuierliche Symmetrien beschreiben, eins der wichtigsten Konzepte der Mathematik und der Physik. Das Ausnutzen von Symmetrien, z.B. bei der Beschreibung homogener Räume erleichtert bei vielen konkreten Problemen die Lösung und gibt oft einen tieferen Einblick in die untersuchten Strukturen.  Zudem ist die Geometrie und Topologie von Lie-Gruppen und homogenen Räumen selbst von großem Interesse.

In dieser Vorlesung werden wir zunächst die grundlegende Theorie von Lie-Gruppen und Lie-Algebren einführen, insbesondere mit Einblicken in die Strukturtheorie von Lie-Algebren. Im zweiten Teil werden wir dann homogene Räume betrachten mit einem besonderen Fokus auf Riemannsche symmetrische Räume. Letztere sind eine wichtige Beispielklasse Riemannscher Mannigfaltigkeiten. Ein besonderer Fokus wird neben den Lie-theoretischen Aspekten immer auch auf den homogenen Riemannschen Metriken der jeweiligen Räume liegen.
\subsubsection*{\large
    Literatur:
}
\begin{itemize}
\item
S.~Helgason. \emph{Differential geometry and symmetric spaces}. American Mathematical Soc., 2001.
\item
J.M.~Lee: \emph{Smooth manifolds}. Springer New York, 2012.
\item
B.~O'Neill: \emph{Semi-Riemannian geometry with applications to relativity}. Academic press, 1983.
\item
W.~Ziller: \emph{Lie Groups. Representation Theory and Symmetric Spaces}. Lecture Notes, 2010.
\end{itemize}

\subsubsection*{\large
    Vorkenntnisse:
}
Differentialgeometrie~I
\subsubsection*{\large
    Bemerkungen:
}
This course is only offered in German.
\subsubsection*{\large
    Verwendbarkeit, Studien- und Prüfungsleistungen:
}

\begin{tabularx}{\textwidth}{ p{.5\textwidth}
    |X
    |X
    |X
    |X
}
 &
\makecell[c]{\rotatebox[origin=l]{90}{\parbox{
            8
            cm}{\begin{flushleft}
                Teil des Moduls ''Mathematik'' (MSc14) (5.5 ECTS) \newline Teil des Moduls ''Reine Mathematik'' (MSc14) (5.5 ECTS)
            \end{flushleft} }}}
 &
\makecell[c]{\rotatebox[origin=l]{90}{\parbox{
            8
            cm}{\begin{flushleft}
                Teil des Vertiefungsmoduls (MSc14) (5.25 ECTS)
            \end{flushleft} }}}
 &
\makecell[c]{\rotatebox[origin=l]{90}{\parbox{
            8
            cm}{\begin{flushleft}
                Wahlmodul (MSc14) (6.0 ECTS) \newline Wahlmodul (MScData24) (6.0 ECTS) \newline Wahlmodul (Option ''Individuelle Studiengestaltung'') (2HfB21) (6.0 ECTS)
            \end{flushleft} }}}
 &
\makecell[c]{\rotatebox[origin=l]{90}{\parbox{
            8
            cm}{\begin{flushleft}
                Wahlpflichtmodul Mathematik (BSc21) (6.0 ECTS)
            \end{flushleft} }}}
\\
& \Var{veranstaltung["verwendbarkeit"].columns.index(y)}
& \Var{veranstaltung["verwendbarkeit"].columns.index(y)}
& \Var{veranstaltung["verwendbarkeit"].columns.index(y)}
& \Var{veranstaltung["verwendbarkeit"].columns.index(y)}
\\[2ex] \hline
\hline \rule[0mm]{0cm}{.6cm}PL: Mündliche Prüfung (Dauer: ca. 30 Minuten). \rule[-3mm]{0cm}{0cm}
 &
 &
 &
 &
\makecell[c]{\xmark}
\\
\hline \rule[0mm]{0cm}{.6cm}PL: Mündliche Prüfung über alle Teile des Moduls (Dauer:  45 Minuten) \rule[-3mm]{0cm}{0cm}
 &
 &
\makecell[c]{\xmark}
 &
 &
\\
\hline \rule[0mm]{0cm}{.6cm}PL: Mündliche Prüfung über alle Teile des Moduls (Dauer: 30 Minuten) \rule[-3mm]{0cm}{0cm}
 &
\makecell[c]{\xmark}
 &
 &
 &
\\
\hline \rule[0mm]{0cm}{.6cm}SL: Bestehen eines mündlichen Abschlusstests. \rule[-3mm]{0cm}{0cm}
 &
 &
 &
\makecell[c]{\xmark}
 &
\\
\hline \rule[0mm]{0cm}{.6cm}SL: Erreichen von mindestens 50\% der Punkte, die insgesamt durch die Bearbeitung der für die Übung ausgegebenen Übungsaufgaben erreicht werden können. \rule[-3mm]{0cm}{0cm}
 &
\makecell[c]{\xmark}
 &
\makecell[c]{\xmark}
 &
\makecell[c]{\xmark}
 &
\makecell[c]{\xmark}
\\
\end{tabularx}




\clearpage\hrule\vskip1pt\hrule
\section*{\Large Markov-Ketten}
\addcontentsline{toc}{subsection}{Markov-Ketten\ \textcolor{gray}{(\em David Criens)}}
\vskip-2ex
{\it David Criens, Assistenz: Dario Kieffer}
\hfill{E, 6.0 ECTS}\\
Vorlesung: Do, 12--14 Uhr, SR 226, \href{https://www.openstreetmap.org/?mlat=48.003472&mlon=7.848195#map=19/48.003472/7.848195}{Hermann-Herder-Str. 10}\\
Übung: 2-stündig, Termin wird noch festgelegt \\
\subsubsection*{\large
    Inhalt:
}
Die Klasse der Markov-Ketten ist eine wichtige Klasse von (zeitdiskreten) stochastischen Prozessen, die häufig verwendet werden, um zum Beispiel die Ausbreitung von Infektionen, Warteschlangensysteme oder Wechsel von Wirtschaftsszenarien zu modellieren. Ihr Hauptmerkmal ist die Markov-Eigenschaft, was in etwa bedeutet, dass die Zukunft von der Vergangenheit nur durch den aktuellen Zustand abhängt. In dieser Vorlesung wird die mathematischen Grundlagen der Theorie der Markov-Ketten vorgestellt. Insbesondere diskutieren wir über Pfadeigenschaften, wie Rekurrenz, Transienz, Zustandsklassifikationen sowie die Konvergenz zu einem Gleichgewicht. Wir untersuchen auch Erweiterungen auf kontinuierliche Zeit. Auf dem Weg dorthin diskutieren wir Anwendungen in der Biologie, in Warteschlangensystemen und  im Ressourcenmanagement. Wenn es die Zeit erlaubt, werfen wir auch einen Blick auf Markov-Ketten mit zufälligen Übergangswahrscheinlichkeiten, sogenannten Irrfahrten in zufälliger Umgebung, ein verbreitetes Modell für Zufällige Medien. 
\subsubsection*{\large
    Literatur:
}
J. R. Norris: \emph{Markov Chains}, Cambridge University Press, 1997
\subsubsection*{\large
    Vorkenntnisse:
}
Notwendig: Stochastik~I \\
Nützlich: Analysis~III,  Wahrscheinlichkeitstheorie~I
\subsubsection*{\large
    Bemerkungen:
}
Dieser Kurs wird auf englisch angeboten.
\subsubsection*{\large
    Verwendbarkeit, Studien- und Prüfungsleistungen:
}

\begin{tabularx}{\textwidth}{ p{.5\textwidth}
    |X
    |X
    |X
    |X
}
 &
\makecell[c]{\rotatebox[origin=l]{90}{\parbox{
            8
            cm}{\begin{flushleft}
                Elective in Data (MScData24) (6.0 ECTS) \newline Wahlpflichtmodul Mathematik (BSc21) (6.0 ECTS)
            \end{flushleft} }}}
 &
\makecell[c]{\rotatebox[origin=l]{90}{\parbox{
            8
            cm}{\begin{flushleft}
                Mathematische Ergänzung (MEd18) (3.0 ECTS) \newline Wahlmodul (MSc14) (6.0 ECTS) \newline Wahlmodul (Option ''Individuelle Studiengestaltung'') (2HfB21) (6.0 ECTS)
            \end{flushleft} }}}
 &
\makecell[c]{\rotatebox[origin=l]{90}{\parbox{
            8
            cm}{\begin{flushleft}
                Teil des Moduls ''Angewandte Mathematik'' (MSc14) (5.5 ECTS) \newline Teil des Moduls ''Mathematik'' (MSc14) (5.5 ECTS)
            \end{flushleft} }}}
 &
\makecell[c]{\rotatebox[origin=l]{90}{\parbox{
            8
            cm}{\begin{flushleft}
                Teil des Vertiefungsmoduls (MSc14) (5.25 ECTS)
            \end{flushleft} }}}
\\
& \Var{veranstaltung["verwendbarkeit"].columns.index(y)}
& \Var{veranstaltung["verwendbarkeit"].columns.index(y)}
& \Var{veranstaltung["verwendbarkeit"].columns.index(y)}
& \Var{veranstaltung["verwendbarkeit"].columns.index(y)}
\\[2ex] \hline
\hline \rule[0mm]{0cm}{.6cm}PL: Mündliche Prüfung (Dauer: ca. 30 Minuten). \rule[-3mm]{0cm}{0cm}
 &
\makecell[c]{\xmark}
 &
 &
 &
\\
\hline \rule[0mm]{0cm}{.6cm}PL: Mündliche Prüfung über alle Teile des Moduls (Dauer:  45 Minuten) \rule[-3mm]{0cm}{0cm}
 &
 &
 &
 &
\makecell[c]{\xmark}
\\
\hline \rule[0mm]{0cm}{.6cm}PL: Mündliche Prüfung über alle Teile des Moduls (Dauer: 30 Minuten) \rule[-3mm]{0cm}{0cm}
 &
 &
 &
\makecell[c]{\xmark}
 &
\\
\hline \rule[0mm]{0cm}{.6cm}SL: Bestehen eines mündlichen Abschlusstests. \rule[-3mm]{0cm}{0cm}
 &
 &
\makecell[c]{\xmark}
 &
 &
\\
\hline \rule[0mm]{0cm}{.6cm}SL: Erreichen von mindestens 50\% der Punkte, die insgesamt durch die Bearbeitung der für die Übung ausgegebenen Übungsaufgaben erreicht werden können. \rule[-3mm]{0cm}{0cm}
 &
\makecell[c]{\xmark}
 &
\makecell[c]{\xmark}
 &
\makecell[c]{\xmark}
 &
\makecell[c]{\xmark}
\\
\end{tabularx}




\clearpage\hrule\vskip1pt\hrule
\section*{\Large \href{https://pfaffelh.github.io/hp/2024WS_measure_theory.html}{Maßtheorie}}
\addcontentsline{toc}{subsection}{Maßtheorie\ \textcolor{gray}{(\em Peter Pfaffelhuber)}}
\vskip-2ex
{\it Peter Pfaffelhuber, Assistenz: Samuel Adeosun}
\hfill{E, 6.0 ECTS}\\
Tutorat: 2-stündig: Mi, 10--12 Uhr, HS II, \href{https://www.openstreetmap.org/?mlat=48.002320&mlon=7.847924#map=19/48.002320/7.847924}{Albertstr. 23b}\\
Vorlesung (2-stündig): asynchrone Videos \\
% Webseite: \url{https://pfaffelh.github.io/hp/2024WS_measure_theory.html}
\subsubsection*{\large
    Inhalt:
}
Die Maßtheorie ist die Grundlage der fortgeschrittenen Wahrscheinlichkeitstheorie. In diesem Kurs bauen wir auf den Kenntnissen der Analysis auf und liefern alle notwendigen Ergebnisse für spätere Kurse in Statistik, probabilistischem maschinellem Lernen und stochastischen Prozessen. Der Kurs beinhaltet Mengensysteme, Konstruktionen von Maßen über äußere Maße, das Integral und Produktmaße.
\subsubsection*{\large
    Literatur:
}
\begin{itemize}
 \item H. Bauer. \emph{Measure and Integration Theory}. deGruyter, 2001. 
\item V. Bogatchev. \emph{Measure Theory}. Springer, 2007.
\item O. Kallenberg. \emph{Foundations of Modern Probability Theory}. Springer, 2021.
\end{itemize}
\subsubsection*{\large
    Vorkenntnisse:
}
Grundlagenvorlesung in Analysis und Verständnis mathematischer Beweise.
\subsubsection*{\large
    Bemerkungen:
}
Dies ist ein Selbstlernkurs, für den (korrigierte) Übungsblätter angeboten werden.\\
Dieser Kurs wird auf englisch angeboten.
\subsubsection*{\large
    Verwendbarkeit, Studien- und Prüfungsleistungen:
}

\begin{tabularx}{\textwidth}{ p{.5\textwidth}
    |X
}
 &
\makecell[c]{\rotatebox[origin=l]{90}{\parbox{
            4
            cm}{\begin{flushleft}
                Elective in Data (MScData24) (6.0 ECTS)
            \end{flushleft} }}}
\\
& \Var{veranstaltung["verwendbarkeit"].columns.index(y)}
\\[2ex] \hline
\hline \rule[0mm]{0cm}{.6cm}PL: Mündliche Prüfung (Dauer: ca. 30 Minuten). \rule[-3mm]{0cm}{0cm}
 &
\makecell[c]{\xmark}
\\
\hline \rule[0mm]{0cm}{.6cm}SL: Erreichen von mindestens 50\% der Punkte, die insgesamt durch die Bearbeitung der für die Übung ausgegebenen Übungsaufgaben erreicht werden können. \rule[-3mm]{0cm}{0cm}
 &
\makecell[c]{\xmark}
\\
\end{tabularx}




\clearpage\hrule\vskip1pt\hrule
\section*{\Large \href{https://aam.uni-freiburg.de/agsa/lehre/ws24/numsde/index.html}{Numerische Approximation stochastischer Differentialgleichungen}}
\addcontentsline{toc}{subsection}{Numerische Approximation stochastischer Differentialgleichungen\ \textcolor{gray}{(\em Diyora Salimova)}}
\vskip-2ex
{\it Diyora Salimova, Assistenz:  Ilkhom Mukhammadiev}
\hfill{E, 6.0 ECTS}\\
Vorlesung: Di, Fr, 12--14 Uhr, SR 226, \href{https://www.openstreetmap.org/?mlat=48.003472&mlon=7.848195#map=19/48.003472/7.848195}{Hermann-Herder-Str. 10}\\
Übung: 2-stündig, Termin wird noch festgelegt \\
Praktische Übung: 2-stündig, Termin wird noch festgelegt \\
% Webseite: \url{https://aam.uni-freiburg.de/agsa/lehre/ws24/numsde/index.html}
\subsubsection*{\large
    Inhalt:
}
Ziel dieses Kurses ist es, die Studierenden in die Lage zu versetzen, Simulationen und deren mathematische Analyse für stochastische Modelle aus Anwendungen wie der Finanzmathematik und der Physik durchzuführen. Zu diesem Zweck vermittelt der Kurs ein solides Wissen über stochastische Differentialgleichungen (SDEs) und deren Lösungen. Darüber hinaus werden verschiedene numerische Methoden für SDEs, ihre zugrunde liegenden Ideen, Konvergenzeigenschaften und Implementierungsprobleme untersucht.
\subsubsection*{\large
    Literatur:
}
\begin{itemize}
\item
P. E. Kloeden and E. Platen: \emph{Numerical Solution of Stochastic Differential Equations.} Springer-Verlag, Berlin, 1992. 
\item
Bernt Oksendal: \emph{Stochastic Differential Equations}, Springer, 2010.
\end{itemize}
\subsubsection*{\large
    Vorkenntnisse:
}
Stochastik, Maßtheorie, Numerik und MATLAB-Programmierung.
\subsubsection*{\large
    Verwendbarkeit, Studien- und Prüfungsleistungen:
}

\begin{tabularx}{\textwidth}{ p{.5\textwidth}
    |X
    |X
    |X
    |X
}
 &
\makecell[c]{\rotatebox[origin=l]{90}{\parbox{
            8
            cm}{\begin{flushleft}
                Elective in Data (MScData24) (6.0 ECTS) \newline Wahlpflichtmodul Mathematik (BSc21) (6.0 ECTS)
            \end{flushleft} }}}
 &
\makecell[c]{\rotatebox[origin=l]{90}{\parbox{
            8
            cm}{\begin{flushleft}
                Mathematische Ergänzung (MEd18) (3.0 ECTS) \newline Wahlmodul (MSc14) (6.0 ECTS) \newline Wahlmodul (Option ''Individuelle Studiengestaltung'') (2HfB21) (6.0 ECTS)
            \end{flushleft} }}}
 &
\makecell[c]{\rotatebox[origin=l]{90}{\parbox{
            8
            cm}{\begin{flushleft}
                Teil des Moduls ''Angewandte Mathematik'' (MSc14) (5.5 ECTS) \newline Teil des Moduls ''Mathematik'' (MSc14) (5.5 ECTS)
            \end{flushleft} }}}
 &
\makecell[c]{\rotatebox[origin=l]{90}{\parbox{
            8
            cm}{\begin{flushleft}
                Teil des Vertiefungsmoduls (MSc14) (5.25 ECTS)
            \end{flushleft} }}}
\\
& \Var{veranstaltung["verwendbarkeit"].columns.index(y)}
& \Var{veranstaltung["verwendbarkeit"].columns.index(y)}
& \Var{veranstaltung["verwendbarkeit"].columns.index(y)}
& \Var{veranstaltung["verwendbarkeit"].columns.index(y)}
\\[2ex] \hline
\hline \rule[0mm]{0cm}{.6cm}PL: Mündliche Prüfung (Dauer: ca. 30 Minuten). \rule[-3mm]{0cm}{0cm}
 &
\makecell[c]{\xmark}
 &
 &
 &
\\
\hline \rule[0mm]{0cm}{.6cm}PL: Mündliche Prüfung über alle Teile des Moduls (Dauer:  45 Minuten) \rule[-3mm]{0cm}{0cm}
 &
 &
 &
 &
\makecell[c]{\xmark}
\\
\hline \rule[0mm]{0cm}{.6cm}PL: Mündliche Prüfung über alle Teile des Moduls (Dauer: 30 Minuten) \rule[-3mm]{0cm}{0cm}
 &
 &
 &
\makecell[c]{\xmark}
 &
\\
\hline \rule[0mm]{0cm}{.6cm}SL: Bestehen eines mündlichen Abschlusstests. \rule[-3mm]{0cm}{0cm}
 &
 &
\makecell[c]{\xmark}
 &
 &
\\
\hline \rule[0mm]{0cm}{.6cm}SL: Erreichen von mindestens 50\% der Punkte, die insgesamt durch die Bearbeitung der für die Übung ausgegebenen Übungsaufgaben erreicht werden können. \rule[-3mm]{0cm}{0cm}
 &
\makecell[c]{\xmark}
 &
\makecell[c]{\xmark}
 &
\makecell[c]{\xmark}
 &
\makecell[c]{\xmark}
\\
\end{tabularx}




\clearpage\hrule\vskip1pt\hrule
\section*{\Large \href{https://www.syscop.de/teaching/ws2024/numerical-optimal-control}{Numerical Optimal Control}}
\addcontentsline{toc}{subsection}{Numerical Optimal Control\ \textcolor{gray}{(\em Moritz Diehl)}}
\vskip-2ex
{\it Moritz Diehl, Assistenz: Florian Messerer}
\hfill{E, 6.0 ECTS}\\
Übung / flipped classroom: Di, 14--16 Uhr, HS II, \href{https://www.openstreetmap.org/?mlat=48.002320&mlon=7.847924#map=19/48.002320/7.847924}{Albertstr. 23b}\\
% Webseite: \url{https://www.syscop.de/teaching/ws2024/numerical-optimal-control}
\subsubsection*{\large
    Inhalt:
}
Ziel des Kurses ist es, eine Einführung in numerische Methoden zu geben für die Lösung optimaler Kontrollprobleme in Wissenschaft und Technik. Der Schwerpunkt liegt sowohl auf zeitdiskreter als auch auf zeitkontinuierlicher optimaler Steuerung in kontinuierlichen Zustandsräumen. Der Kurs richtet sich an ein gemischtes Publikum von Studenten der Mathematik, Ingenieurwesen und Informatik.

Der Kurs deckt die folgenden Themen ab:
\begin{itemize}
\item Einführung in dynamische Systeme und Optimierung
\item Newtonverfahren und numerische Optimierung
\item Algorithmische Differenzierung
\item Zeitdiskrete Optimale Steuerung
\item Dynamische Programmierung
\item Optimale Steuerung in kontinuierlicher Zeit
\item Numerische Simulationsmethoden
\item Hamilton-Jacobi-Bellmann-Gleichung
\item Pontryagin und der indirekte Ansatz
\item Direkte Optimale Steuerung
\item Echtzeit-Optimierung für modellprädiktive Steuerung
\end{itemize}

Die Vorlesung wird von intensiven wöchentlichen Computerübungen begleitet, die sowohl in in MATLAB und Python (6~ECTS) absolviert werden können. Es wird außerdem ein optionales Projekt (3~ECTS) angeboten. Dieses besteht in der Formulierung und Implementierung eines selbstgewählten optimalen Kontrollproblems und einer numerischen Lösungsmethode, die in einem Projektbericht dokumentiert und abschließend präsentiert wrird.
\subsubsection*{\large
    Literatur:
}
\begin{itemize}
\item
 M.~Diehl, S.~Gros: \href{https://www.syscop.de/files/2020ss/NOC/book-NOCSE.pdf}{\emph{Numerical Optimal Control}}, lecture notes. 
\item
J.B.~Rawlings, D.Q.~Mayne, M.~Diehl: \href{https://sites.engineering.ucsb.edu/\~jbraw/mpc/MPC-book-2nd-edition-4th-printing.pdf}{\emph{Model Predictive Control}}, 2nd Edition, Nobhill Publishing, 2017.
\item
J.~Betts: \emph{Practical Methods for Optimal Control and Estimation Using Nonlinear Programming}, SIAM, 2010.
\end{itemize}
\subsubsection*{\large
    Vorkenntnisse:
}
Notwendig: Analysis~I und II, Lineare Algebra~I und II Nützlich: Numerik I, Gewöhnliche Differentialgleichungen, Numerische Optimierung
\subsubsection*{\large
    Bemerkungen:
}
Zusammen mit dem optionalen Programmierprojekt wird die 6 ECTS Vorlesung als 9-ECTS-Kurs angerechnet.\\
Dieser Kurs wird auf Englisch angeboten.
\subsubsection*{\large
    Verwendbarkeit, Studien- und Prüfungsleistungen:
}

\begin{tabularx}{\textwidth}{ p{.5\textwidth}
    |X
    |X
    |X
    |X
}
 &
\makecell[c]{\rotatebox[origin=l]{90}{\parbox{
            8
            cm}{\begin{flushleft}
                Elective in Data (MScData24) (6.0 ECTS) \newline Wahlpflichtmodul Mathematik (BSc21) (6.0 ECTS)
            \end{flushleft} }}}
 &
\makecell[c]{\rotatebox[origin=l]{90}{\parbox{
            8
            cm}{\begin{flushleft}
                Mathematische Ergänzung (MEd18) (3.0 ECTS) \newline Wahlmodul (MSc14) (6.0 ECTS) \newline Wahlmodul (Option ''Individuelle Studiengestaltung'') (2HfB21) (6.0 ECTS)
            \end{flushleft} }}}
 &
\makecell[c]{\rotatebox[origin=l]{90}{\parbox{
            8
            cm}{\begin{flushleft}
                Teil des Moduls ''Angewandte Mathematik'' (MSc14) (5.5 ECTS) \newline Teil des Moduls ''Mathematik'' (MSc14) (5.5 ECTS)
            \end{flushleft} }}}
 &
\makecell[c]{\rotatebox[origin=l]{90}{\parbox{
            8
            cm}{\begin{flushleft}
                Teil des Vertiefungsmoduls (MSc14) (5.25 ECTS)
            \end{flushleft} }}}
\\
& \Var{veranstaltung["verwendbarkeit"].columns.index(y)}
& \Var{veranstaltung["verwendbarkeit"].columns.index(y)}
& \Var{veranstaltung["verwendbarkeit"].columns.index(y)}
& \Var{veranstaltung["verwendbarkeit"].columns.index(y)}
\\[2ex] \hline
\hline \rule[0mm]{0cm}{.6cm}PL: Mündliche Prüfung (Dauer: ca. 30 Minuten). \rule[-3mm]{0cm}{0cm}
 &
\makecell[c]{\xmark}
 &
 &
 &
\\
\hline \rule[0mm]{0cm}{.6cm}PL: Mündliche Prüfung über alle Teile des Moduls (Dauer:  45 Minuten) \rule[-3mm]{0cm}{0cm}
 &
 &
 &
 &
\makecell[c]{\xmark}
\\
\hline \rule[0mm]{0cm}{.6cm}PL: Mündliche Prüfung über alle Teile des Moduls (Dauer: 30 Minuten) \rule[-3mm]{0cm}{0cm}
 &
 &
 &
\makecell[c]{\xmark}
 &
\\
\hline \rule[0mm]{0cm}{.6cm}SL: Bestehen eines mündlichen Abschlusstests. \rule[-3mm]{0cm}{0cm}
 &
 &
\makecell[c]{\xmark}
 &
 &
\\
\hline \rule[0mm]{0cm}{.6cm}SL: Erreichen von mindestens 50\% der Punkte, die insgesamt durch die Bearbeitung der für die Übung ausgegebenen Übungsaufgaben erreicht werden können. \rule[-3mm]{0cm}{0cm}
 &
\makecell[c]{\xmark}
 &
\makecell[c]{\xmark}
 &
\makecell[c]{\xmark}
 &
\makecell[c]{\xmark}
\\
\end{tabularx}




\clearpage
\phantomsection
\thispagestyle{empty}
\vspace*{\fill}
\begin{center}
    \Huge\bfseries 2a. Fachdidaktik
\end{center}
\addcontentsline{toc}{section}{\textbf{2a. Fachdidaktik}}
\addtocontents{toc}{\medskip\hrule\medskip}\vspace*{\fill}\vspace*{\fill}\clearpage
\vfill
\thispagestyle{empty}
\clearpage

\clearpage\hrule\vskip1pt\hrule
\section*{\Large Einführung in die Fachdidaktik der Mathematik}
\addcontentsline{toc}{subsection}{Einführung in die Fachdidaktik der Mathematik\ \textcolor{gray}{(\em Katharina Böcherer-Linder)}}
\vskip-2ex
{\it Katharina Böcherer-Linder}
\hfill{D, 5.0 ECTS}\\
Vorlesung mit Übung: Mo, 10--12 Uhr, SR 226, \href{https://www.openstreetmap.org/?mlat=48.003472&mlon=7.848195#map=19/48.003472/7.848195}{Hermann-Herder-Str. 10}\\
Tutorat (2-stündig zur Wahl): 8--10 Uhr, Fr, 14--16 Uhr, SR 127, \href{https://www.openstreetmap.org/?mlat=48.000637&mlon=7.846006#map=19/48.000636/7.846006}{Ernst-Zermelo-Str. 1}\\
\subsubsection*{\large
    Inhalt:
}
Mathematikdidaktische Prinzipien sowie deren lerntheoretische Grundlagen und Möglichkeiten unterrichtlicher Umsetzung (auch z.B. mit Hilfe digitaler Medien). \\
Theoretische Konzepte zu zentralen mathematischen Denkhandlungen wie Begriffsbilden, Modellieren, Problemlösen und Argumentieren. \\
Mathematikdidaktische Konstrukte: Verstehenshürden, Präkonzepte, Grundvorstellungen, spezifische Schwierigkeiten zu ausgewählten mathematischen Inhalten. \\
Konzepte für den Umgang mit Heterogenität unter Berücksichtigung fachspezifischer Besonderheiten (z.B. Rechenschwäche oder mathematische Hochbegabung).\\
Stufen begrifflicher Strenge und Formalisierungen sowie deren altersgemäße Umsetzung.
\subsubsection*{\large
    Vorkenntnisse:
}
Erforderliche Vorkenntnisse sind die Grundvorlesungen in Mathematik (Analysis, Lineare Algebra). \\
Die Veranstaltung "`Einführung in die Mathematikdidaktik"' wird deswegen frühestens ab dem 4.~Fachsemester empfohlen.
\subsubsection*{\large
    Bemerkungen:
}
Die Veranstaltung ist Pflicht in der Lehramtsoption des Zwei-Hauptfächer-Bachelor-Studiengangs. Sie  setzt sich zusammen aus Vorlesungsanteilen und Anteilen mit Übungs- und Seminarcharakter. Die drei Lehrformen lassen sich dabei nicht völlig klar voneinander trennen.
Der Besuch des „Didaktischen Seminars“ (etwa zweiwöchentlich, Dienstag abends, 19:30 Uhr) wird erwartet!
\subsubsection*{\large
    Verwendbarkeit, Studien- und Prüfungsleistungen:
}

\begin{tabularx}{\textwidth}{ p{.5\textwidth}
    |X
}
 &
\makecell[c]{\rotatebox[origin=l]{90}{\parbox{
            4
            cm}{\begin{flushleft}
                (Einführung in die) Fachdidaktik Mathematik (2HfB21, MEH21, MEB21, MEdual24) (5.0 ECTS)
            \end{flushleft} }}}
\\
& \Var{veranstaltung["verwendbarkeit"].columns.index(y)}
\\[2ex] \hline
\hline \rule[0mm]{0cm}{.6cm}SL: Bestehen der Abschlussklausur (Dauer 1 bis 3 Stunden). \rule[-3mm]{0cm}{0cm}
 &
\makecell[c]{\xmark}
\\
\hline \rule[0mm]{0cm}{.6cm}SL: Erfolgreiche schriftliche Bearbeitung von mindestens zwei Dritteln der Übungsaufgaben. \rule[-3mm]{0cm}{0cm}
 &
\makecell[c]{\xmark}
\\
\hline \rule[0mm]{0cm}{.6cm}SL: Regelmäßige Teilnahme an der Veranstaltung (wie in der Prüfungsordnung definiert). \rule[-3mm]{0cm}{0cm}
 &
\makecell[c]{\xmark}
\\
\end{tabularx}




\clearpage\hrule\vskip1pt\hrule
\section*{\Large Didaktik der Funktionen und der Analysis}
\addcontentsline{toc}{subsection}{Didaktik der Funktionen und der Analysis\ \textcolor{gray}{(\em Katharina Böcherer-Linder)}}
\vskip-2ex
{\it Katharina Böcherer-Linder}
\hfill{D, 3.0 ECTS}\\
Seminar: Do, 9--12 Uhr, SR 404, \href{https://www.openstreetmap.org/?mlat=48.000637&mlon=7.846006#map=19/48.000636/7.846006}{Ernst-Zermelo-Str. 1}\\
\subsubsection*{\large
    Inhalt:
}
Exemplarische Umsetzungen der theoretischen Konzepte zu zentralen mathematischen Denkhandlungen wie Begriffsbilden, Modellieren, Problemlösen und Argumentieren für die Inhaltsbereiche Funktionen und Analysis. \\
Verstehenshürden, Präkonzepte, Grundvorstellungen, spezifische Schwierigkeiten zu den Inhaltsbereichen Funktionen
und Analysis. \\
Grundlegende Möglichkeiten und Grenzen von Medien, insbesondere von computergestützten mathematischen Werkzeugen und deren Anwendung für die Inhaltsbereiche Funktionen und Analysis.
Analyse Individueller mathematischer Lernprozesse und Fehler sowie Entwicklung individueller Fördermaßnahmen zu
den Inhaltsbereichen Funktionen und Analysis.
\subsubsection*{\large
    Literatur:
}
\begin{itemize}
\item
R. Dankwerts, D. Vogel: \emph{Analysis verständlich unterrichten}. Heidelberg: Spektrum, 2006. 
 \item
G. Greefrath, R. Oldenburg, H.-S. Siller, V. Ulm, H.-G. Weigand: \emph{Didaktik der Analysis. Aspekte und Grundvorstellungen zentraler Begriffe}. Berlin, Heidelberg: Springer 2016.
\end{itemize}
\subsubsection*{\large
    Vorkenntnisse:
}
Einführung in die Fachdidaktik der Mathematik sowie Kenntnisse aus Analysis und Numerik.
\subsubsection*{\large
    Bemerkungen:
}
Die beiden Teile können in verschiedenen Semestern absolviert werden, haben aber eine gemeinsame Abschlussklausur,
die jedes Semester angeboten und nach Absolvieren beider Teile geschrieben wird.
\subsubsection*{\large
    Verwendbarkeit, Studien- und Prüfungsleistungen:
}

\begin{tabularx}{\textwidth}{ p{.5\textwidth}
    |X
}
 &
\makecell[c]{\rotatebox[origin=l]{90}{\parbox{
            4
            cm}{\begin{flushleft}
                Fachdidaktik der mathematischen Teilgebiete (MEd18, MEH21, MEB21) (3.0 ECTS)
            \end{flushleft} }}}
\\
& \Var{veranstaltung["verwendbarkeit"].columns.index(y)}
\\[2ex] \hline
\hline \rule[0mm]{0cm}{.6cm}PL: Klausur über beide Modulteile. \rule[-3mm]{0cm}{0cm}
 &
\makecell[c]{\xmark}
\\
\hline \rule[0mm]{0cm}{.6cm}SL: Regelmäßige Teilnahme an der Veranstaltung (wie in der Prüfungsordnung definiert). \rule[-3mm]{0cm}{0cm}
 &
\makecell[c]{\xmark}
\\
\hline \rule[0mm]{0cm}{.6cm}SL: Seminarvortrag mit praktischem und theoretischem Teil. \rule[-3mm]{0cm}{0cm}
 &
\makecell[c]{\xmark}
\\
\hline \rule[0mm]{0cm}{.6cm}SL: Wöchentliche Lektüre und gegebenenfalls Hausübung. \rule[-3mm]{0cm}{0cm}
 &
\makecell[c]{\xmark}
\\
\end{tabularx}




\clearpage\hrule\vskip1pt\hrule
\section*{\Large Didaktik der Stochastik und der Algebra}
\addcontentsline{toc}{subsection}{Didaktik der Stochastik und der Algebra\ \textcolor{gray}{(\em Anika Dreher)}}
\vskip-2ex
{\it Anika Dreher}
\hfill{D, 3.0 ECTS}\\
Seminar: Fr, 9--12 Uhr, SR 226, \href{https://www.openstreetmap.org/?mlat=48.003472&mlon=7.848195#map=19/48.003472/7.848195}{Hermann-Herder-Str. 10}\\
\subsubsection*{\large
    Inhalt:
}
Exemplarische Umsetzungen der theoretischen Konzepte zu zentralen mathematischen Denkhandlungen wie Begriffsbilden, Modellieren, Problemlösen und Argumentieren für die Inhaltsbereiche Stochastik und Algebra. \\
Verstehenshürden, Präkonzepte, Grundvorstellungen, spezifische Schwierigkeiten zu den Inhaltsbereichen Stochastik und Algebra.\\
Grundlegende Möglichkeiten und Grenzen von Medien, insbesondere von computergestützten mathematischen Werkzeugen und deren Anwendung für die Inhaltsbereiche Stochastik und Algebra. \\
Analyse Individueller mathematischer Lernprozesse und Fehler sowie Entwicklung individueller Fördermaßnahmen zu den Inhaltsbereichen Stochastik und Algebra.
\subsubsection*{\large
    Literatur:
}
\begin{itemize}
\item 
G. Malle: \emph{Didaktische Probleme der elementaren Algebra}. Braunschweig, Wiesbaden: Vieweg 1993. 
\item
A. Eichler, M. Vogel: \emph{Leitidee Daten und Zufall. Von konkreten Beispielen zur Didaktik der Stochastik}. Wiesbaden:
Vieweg 2009.
\end{itemize}
\subsubsection*{\large
    Vorkenntnisse:
}
Einführung in die Fachdidaktik der Mathematik sowie Kenntisse aus Stochastik und Algebra.
\subsubsection*{\large
    Bemerkungen:
}
Die beiden Teile können in verschiedenen Semestern absolviert werden, haben aber eine gemeinsame Abschlussklausur, die jedes Semester angeboten und nach Absolvieren beider Teile geschrieben wird.
\subsubsection*{\large
    Verwendbarkeit, Studien- und Prüfungsleistungen:
}

\begin{tabularx}{\textwidth}{ p{.5\textwidth}
    |X
}
 &
\makecell[c]{\rotatebox[origin=l]{90}{\parbox{
            4
            cm}{\begin{flushleft}
                Fachdidaktik der mathematischen Teilgebiete (MEd18, MEH21, MEB21) (3.0 ECTS)
            \end{flushleft} }}}
\\
& \Var{veranstaltung["verwendbarkeit"].columns.index(y)}
\\[2ex] \hline
\hline \rule[0mm]{0cm}{.6cm}PL: Klausur über beide Modulteile. \rule[-3mm]{0cm}{0cm}
 &
\makecell[c]{\xmark}
\\
\hline \rule[0mm]{0cm}{.6cm}SL: Regelmäßige Teilnahme an der Veranstaltung (wie in der Prüfungsordnung definiert). \rule[-3mm]{0cm}{0cm}
 &
\makecell[c]{\xmark}
\\
\hline \rule[0mm]{0cm}{.6cm}SL: Seminarvortrag mit praktischem und theoretischem Teil. \rule[-3mm]{0cm}{0cm}
 &
\makecell[c]{\xmark}
\\
\hline \rule[0mm]{0cm}{.6cm}SL: Wöchentliche Lektüre und gegebenenfalls Hausübung. \rule[-3mm]{0cm}{0cm}
 &
\makecell[c]{\xmark}
\\
\end{tabularx}




\clearpage\hrule\vskip1pt\hrule
\section*{\Large Fachdidaktikseminar: Medieneinsatz im Mathematikunterricht}
\addcontentsline{toc}{subsection}{Fachdidaktikseminar: Medieneinsatz im Mathematikunterricht\ \textcolor{gray}{(\em Jürgen Kury)}}
\vskip-2ex
{\it Jürgen Kury}
\hfill{D, 4.0 ECTS}\\
Seminar: Mi, 15--18 Uhr, SR 127, \href{https://www.openstreetmap.org/?mlat=48.000637&mlon=7.846006#map=19/48.000636/7.846006}{Ernst-Zermelo-Str. 1}\\
\subsubsection*{\large
    Inhalt:
}
Der Einsatz von Unterrichtsmedien im Mathematikunterricht gewinnt sowohl auf der Ebene der Unterrichtsplanung wie auch der der Unterrichtsrealisierung an Bedeutung. Vor dem Hintergrund konstruktivistischer Lerntheorien zeigt sich, dass der reflektierte Einsatz unter anderem von Computerprogrammen die mathematische Begriffsbildung nachhaltig unterstützen kann. So erlaubt beispielsweise das Experimentieren mit Computerprogrammen mathematische Strukturen zu entdecken, ohne dass dies von einzelnen Routineoperationen (wie z.~B. Termumformung) überdeckt würde. Es ergeben sich daraus tiefgreifende Konsequenzen für den Mathematikunterricht. Von daher setzt sich dieses Seminar zum Ziel, den Studierenden die notwendigen Entscheidungs- und Handlungskompetenzen zu vermitteln, um zukünftige Mathematiklehrer auf ihre berufliche Tätigkeit vorzubereiten. Ausgehend von ersten Überlegungen zur Unterrichtsplanung werden anschließend Computer und Tablets hinsichtlich ihres jeweiligen didaktischen Potentials untersucht und während eines Unterrichtsbesuchs mit Lernenden erprobt.

Die dabei exemplarisch vorgestellten Systeme sind:
\begin{itemize}
\item
  dynamische Geometrie Software: Geogebra
\item
 Tabellenkalkulation: Excel
\item  Apps für Smartphones und Tablets
\end{itemize}
Die Studierenden sollen Unterrichtssequenzen ausarbeiten, die dann mit Schülern erprobt und reflektiert werden (soweit dies möglich sein wird).
\subsubsection*{\large
    Vorkenntnisse:
}
Nützlich: Grundvorlesungen in Mathematik
\subsubsection*{\large
    Verwendbarkeit, Studien- und Prüfungsleistungen:
}

\begin{tabularx}{\textwidth}{ p{.5\textwidth}
}
\\
\\[2ex] \hline
\end{tabularx}




\clearpage\hrule\vskip1pt\hrule
\section*{\Large Fachdidaktikseminare der PH Freiburg}
\addcontentsline{toc}{subsection}{Fachdidaktikseminare der PH Freiburg\ \textcolor{gray}{(\em Dozent:inn:en der PH Freiburg)}}
\vskip-2ex
{\it Dozent:inn:en der PH Freiburg}
\hfill{D, 4.0 ECTS}\\
\subsubsection*{\large
    Inhalt:
}
Für das Modul „Fachdidaktische Entwicklung“ können auch geeignete Veranstaltungen an der PH Freiburg absolviert
werden, sofern dort Studienplätze zur Verfügung stehen. Ob Veranstaltungen geeignet sind, sprechen Sie bitte vorab
mit Frau Böcherer-Linder ab; ob Studienplätze zur Verfügung stehen, müssen Sie bei Interessen an einer Veranstaltung
von den Dozent:inn:en erfragen.
\subsubsection*{\large
    Vorkenntnisse:
}
Für das Modul „Fachdidaktische Entwicklung“ können auch geeignete Veranstaltungen an der PH Freiburg absolviert
werden, sofern dort Studienplätze zur Verfügung stehen. Ob Veranstaltungen geeignet sind, sprechen Sie bitte vorab
mit Frau Böcherer-Linder ab; ob Studienplätze zur Verfügung stehen, müssen Sie bei Interessen an einer Veranstaltung
von den Dozent:inn:en erfragen.
\subsubsection*{\large
    Bemerkungen:
}
Für das Modul „Fachdidaktische Entwicklung“ können auch geeignete Veranstaltungen an der PH Freiburg absolviert
werden, sofern dort Studienplätze zur Verfügung stehen. Ob Veranstaltungen geeignet sind, sprechen Sie bitte vorab
mit Frau Böcherer-Linder ab; ob Studienplätze zur Verfügung stehen, müssen Sie bei Interessen an einer Veranstaltung
von den Dozent:inn:en erfragen.
\subsubsection*{\large
    Verwendbarkeit, Studien- und Prüfungsleistungen:
}

\begin{tabularx}{\textwidth}{ p{.5\textwidth}
}
\\
\\[2ex] \hline
\end{tabularx}




\clearpage\hrule\vskip1pt\hrule
\section*{\Large Modul ''Fachdidaktische Forschung'':}
\addcontentsline{toc}{subsection}{Modul ''Fachdidaktische Forschung'':\ \textcolor{gray}{(\em Dozent:inn:en der PH Freiburg)}}
\vskip-2ex
{\it Dozent:inn:en der PH Freiburg}
\hfill{, 4.0 ECTS}\\
\subsubsection*{\large
    Inhalt:
}
Die drei zusammengehörigen Veranstaltungen des Moduls bereiten auf das Anfertigen einer empirischen Masterarbeit in der Mathematikdidaktik vor. Das Angebot wird von allen Professor:innen der PH mit mathematikdidaktischen Forschungsprojekten der Sekundarstufe 1 und 2 gemeinsam konzipiert und von einem dieser Forschenden durchgeführt. Im Anschluss besteht das Angebot, bei einem/einer dieser Personen eine fachdidaktische Masterarbeit anzufertigen – meist eingebunden in größere laufende Forschungsprojekte.

Die Haupziele des Moduls sind die Fähigkeit zur Rezeption mathematikdidaktischer Forschung zur Klärung praxisrelevanter Fragen sowie die Planung einer empirischen mathematikdidaktischen Masterarbeit.
Es wird abgehalten werden als Mischung aus Seminar, Erarbeitung von Forschungsthemen in Gruppenarbeit sowie aktivem Arbeiten mit Forschungsdaten. Literatur wird abhängig von den angebotenen Forschungsthemen innerhalb der jeweiligen Veranstaltungen angegeben werden. Die Teile können auch in verschiedenen Semestern besucht werden, zum Beispiel Teil~1 im zweiten Mastersemester und Teil~2 in der Kompaktphase des dritten Mastersemesters nach dem Praxissemester.
\subsubsection*{\large
    Bemerkungen:
}
Dreiteiliges Modul für die Studierenden im M.Ed., die eine fachdidaktische Master-Arbeit in Mathematik schreiben möchten. Teilnahme nur nach persönlicher Anmeldung bis Ende der Vorlesungszeit des Vorsemesters in der Abteilung für Didaktik. Die Aufnahmekapazitäten sind beschränkt. \par
Voranmeldung: Wer neu an diesem Modul teilnehmen möchte, meldet sich bitte bis zum 30.09.2024 per E-Mail bei
\href{mailto:didaktik@math.uni-freiburg.de}{didaktik@math.uni-freiburg.de} und bei \href{mailto:erens@ph-freiburg.de}{Ralf Erens}.
\subsubsection*{\large
    Verwendbarkeit, Studien- und Prüfungsleistungen:
}

\begin{tabularx}{\textwidth}{ p{.5\textwidth}
}
\\
\\[2ex] \hline
\end{tabularx}




\clearpage\hrule\vskip1pt\hrule
\section*{\Large Teil 1: Fachdidaktische Entwicklungsforschung zu ausgewählten Schwerpunkten}
\addcontentsline{toc}{subsection}{Teil 1: Fachdidaktische Entwicklungsforschung zu ausgewählten Schwerpunkten\ \textcolor{gray}{(\em Frank Reinhold)}}
\vskip-2ex
{\it Frank Reinhold}
\hfill{D, 6.0 ECTS}\\
Seminar: Mo, 14--16 Uhr, Raum (PH) noch nicht bekannt, \href{https://www.openstreetmap.org/?mlat=47.98132\&mlon=7.89420\#map=17/47.98132/7.89420}{PH Freiburg}\\
\subsubsection*{\large
    Inhalt:
}
In dieser ersten Veranstaltung des Moduls findet eine Einführung in Strategien empirischer fachdidaktischer Forschung statt (Forschungsfragen, Forschungsstände, Forschungsdesigns). Studierende vertiefen ihre Fähigkeiten der wissenschaftlichen Recherche und der Bewertung fachdidaktischer Forschung.
\subsubsection*{\large
    Bemerkungen:
}
This course will only be offered in German.
\subsubsection*{\large
    Verwendbarkeit, Studien- und Prüfungsleistungen:
}

\begin{tabularx}{\textwidth}{ p{.5\textwidth}
    |X
}
 &
\makecell[c]{\rotatebox[origin=l]{90}{\parbox{
            4
            cm}{\begin{flushleft}
                Fachdidaktische Forschung (MEd18, MEH21, MEB21) (6.0 ECTS)
            \end{flushleft} }}}
\\
& \Var{veranstaltung["verwendbarkeit"].columns.index(y)}
\\[2ex] \hline
\hline \rule[0mm]{0cm}{.6cm}SL: In allen drei Teilen des Moduls: Bearbeitung von Aufgaben nach Maßgabe der Lehrenden im Umfang von insgesamt etwa 60 Stunden. \rule[-3mm]{0cm}{0cm}
 &
\makecell[c]{\xmark}
\\
\hline \rule[0mm]{0cm}{.6cm}SL: Regelmäßige Teilnahme an der Veranstaltung (wie in der Prüfungsordnung definiert). \rule[-3mm]{0cm}{0cm}
 &
\makecell[c]{\xmark}
\\
\end{tabularx}




\clearpage\hrule\vskip1pt\hrule
\section*{\Large Teil 2: Methoden der mathematikdidaktischen Forschung}
\addcontentsline{toc}{subsection}{Teil 2: Methoden der mathematikdidaktischen Forschung\ \textcolor{gray}{(\em Frank Reinhold)}}
\vskip-2ex
{\it Frank Reinhold}
\hfill{D, 6.0 ECTS}\\
Seminar: Mo, 16--19 Uhr, Raum (PH) noch nicht bekannt, \href{https://www.openstreetmap.org/?mlat=47.98132\&mlon=7.89420\#map=17/47.98132/7.89420}{PH Freiburg}\\
\subsubsection*{\large
    Inhalt:
}
In der zweiten Veranstaltung des Moduls (im letzten Semesterdrittel) werden die Studierenden durch konkrete Arbeit mit bestehenden Daten (Interviews, Schülerprodukte, Experimentaldaten) in zentrale qualitative und quantitative Forschungsmethoden eingeführt.
\subsubsection*{\large
    Verwendbarkeit, Studien- und Prüfungsleistungen:
}

\begin{tabularx}{\textwidth}{ p{.5\textwidth}
    |X
}
 &
\makecell[c]{\rotatebox[origin=l]{90}{\parbox{
            4
            cm}{\begin{flushleft}
                Fachdidaktische Forschung (MEd18, MEH21, MEB21) (6.0 ECTS)
            \end{flushleft} }}}
\\
& \Var{veranstaltung["verwendbarkeit"].columns.index(y)}
\\[2ex] \hline
\hline \rule[0mm]{0cm}{.6cm}SL: In allen drei Teilen des Moduls: Bearbeitung von Aufgaben nach Maßgabe der Lehrenden im Umfang von insgesamt etwa 60 Stunden. \rule[-3mm]{0cm}{0cm}
 &
\makecell[c]{\xmark}
\\
\hline \rule[0mm]{0cm}{.6cm}SL: Regelmäßige Teilnahme an der Veranstaltung (wie in der Prüfungsordnung definiert). \rule[-3mm]{0cm}{0cm}
 &
\makecell[c]{\xmark}
\\
\end{tabularx}




\clearpage\hrule\vskip1pt\hrule
\section*{\Large Teil 3: Entwicklung und Optimierung eines fachdidaktischen Forschungsprojekts}
\addcontentsline{toc}{subsection}{Teil 3: Entwicklung und Optimierung eines fachdidaktischen Forschungsprojekts\ \textcolor{gray}{(\em Dozent:inn:en der PH Freiburg)}}
\vskip-2ex
{\it Dozent:inn:en der PH Freiburg}
\hfill{D, 6.0 ECTS}\\
Termine nach Vereinbarung \\
\subsubsection*{\large
    Inhalt:
}
Begleitseminar zur Master-Arbeit
\subsubsection*{\large
    Bemerkungen:
}
This seminar will only be offered in German.
\subsubsection*{\large
    Verwendbarkeit, Studien- und Prüfungsleistungen:
}

\begin{tabularx}{\textwidth}{ p{.5\textwidth}
    |X
}
 &
\makecell[c]{\rotatebox[origin=l]{90}{\parbox{
            4
            cm}{\begin{flushleft}
                Fachdidaktische Forschung (MEd18, MEH21, MEB21) (6.0 ECTS)
            \end{flushleft} }}}
\\
& \Var{veranstaltung["verwendbarkeit"].columns.index(y)}
\\[2ex] \hline
\hline \rule[0mm]{0cm}{.6cm}SL: In allen drei Teilen des Moduls: Bearbeitung von Aufgaben nach Maßgabe der Lehrenden im Umfang von insgesamt etwa 60 Stunden. \rule[-3mm]{0cm}{0cm}
 &
\makecell[c]{\xmark}
\\
\hline \rule[0mm]{0cm}{.6cm}SL: Regelmäßige Teilnahme an der Veranstaltung (wie in der Prüfungsordnung definiert). \rule[-3mm]{0cm}{0cm}
 &
\makecell[c]{\xmark}
\\
\end{tabularx}




\clearpage
\phantomsection
\thispagestyle{empty}
\vspace*{\fill}
\begin{center}
    \Huge\bfseries 2b. Tutoratsmodul
\end{center}
\addcontentsline{toc}{section}{\textbf{2b. Tutoratsmodul}}
\addtocontents{toc}{\medskip\hrule\medskip}\vspace*{\fill}\vspace*{\fill}\clearpage
\vfill
\thispagestyle{empty}
\clearpage

\clearpage\hrule\vskip1pt\hrule
\section*{\Large \href{https://home.mathematik.uni-freiburg.de/ldl/index1.html}{Lernen durch Lehren}}
\addcontentsline{toc}{subsection}{Lernen durch Lehren\ \textcolor{gray}{(\em Susanne Knies)}}
\vskip-2ex
{\it Susanne Knies}
\hfill{D, 3.0 ECTS}\\
% Webseite: \url{https://home.mathematik.uni-freiburg.de/ldl/index1.html}
\subsubsection*{\large
    Inhalt:
}
Was macht ein gutes Tutorat aus? Im ersten Workshop wird diese Frage diskutiert und es werden Tipps und Anregungen mitgegeben. Im zweiten Workshop werden die Erfahrungen ausgetauscht.
\subsubsection*{\large
    Bemerkungen:
}
Voraussetzung für die Teilnahme ist eine Tutoratsstelle zu einer Vorlesung des Mathematischen Instituts im laufenden
Semester (mindestens eine zweistündige oder zwei einstündige Übungsgruppen über das ganze Semester).

Kann im M.Sc.-Studiengang Mathematik zweimal verwendet werden.
\subsubsection*{\large
    Verwendbarkeit, Studien- und Prüfungsleistungen:
}

\begin{tabularx}{\textwidth}{ p{.5\textwidth}
    |X
}
 &
\makecell[c]{\rotatebox[origin=l]{90}{\parbox{
            4
            cm}{\begin{flushleft}
                Wahlmodul (BSc21) (3.0 ECTS) \newline Wahlmodul (MSc14) (3.0 ECTS) \newline Wahlmodul (MScData24) (3.0 ECTS) \newline Wahlmodul (Option ''Individuelle Studiengestaltung'') (2HfB21) (3.0 ECTS)
            \end{flushleft} }}}
\\
& \Var{veranstaltung["verwendbarkeit"].columns.index(y)}
\\[2ex] \hline
\hline \rule[0mm]{0cm}{.6cm}SL: Teilnahme an beiden Terminen des Tutoratsworkshops. 
Regelmäßige Teilnahme an der Tutorenbesprechung;
Zwei gegenseitige Tutoratsbesuche mit einem (oder mehreren) anderen Modulteilnehmern. \rule[-3mm]{0cm}{0cm}
 &
\makecell[c]{\xmark}
\\
\end{tabularx}




\clearpage
\phantomsection
\thispagestyle{empty}
\vspace*{\fill}
\begin{center}
    \Huge\bfseries 2c. Praktische Übungen
\end{center}
\addcontentsline{toc}{section}{\textbf{2c. Praktische Übungen}}
\addtocontents{toc}{\medskip\hrule\medskip}\vspace*{\fill}\vspace*{\fill}\clearpage
\vfill
\thispagestyle{empty}
\clearpage

\clearpage\hrule\vskip1pt\hrule
\section*{\Large Praktische Übung zu Einführung in Theorie und Numerik Partieller Differentialgleichungen}
\addcontentsline{toc}{subsection}{Praktische Übung zu Einführung in Theorie und Numerik Partieller Differentialgleichungen\ \textcolor{gray}{(\em Sören Bartels)}}
\vskip-2ex
{\it Sören Bartels, Assistenz: Vera Jackisch}
\hfill{E, 3.0 ECTS}\\
Praktische Übung: 2-stündig, Termin wird noch festgelegt \\
\subsubsection*{\large
    Inhalt:
}
Die Praktische Übung begleitet die gleichnamige Vorlesung mit Programmieraufgaben zum Vorlesungsstoff.
\subsubsection*{\large
    Vorkenntnisse:
}
Siehe bei der Vorlesung – zusätzlich: Programmierkenntnisse.
\subsubsection*{\large
    Bemerkungen:
}
Dieser Kurs wird auf Englisch angeboten.
\subsubsection*{\large
    Verwendbarkeit, Studien- und Prüfungsleistungen:
}

\begin{tabularx}{\textwidth}{ p{.5\textwidth}
    |X
}
 &
\makecell[c]{\rotatebox[origin=l]{90}{\parbox{
            4
            cm}{\begin{flushleft}
                Mathematische Ergänzung (MEd18) (3.0 ECTS) \newline Wahlmodul (BSc21) (3.0 ECTS) \newline Wahlmodul (MSc14) (3.0 ECTS) \newline Wahlmodul (MScData24) (3.0 ECTS) \newline Wahlmodul (Option ''Individuelle Studiengestaltung'') (2HfB21) (3.0 ECTS)
            \end{flushleft} }}}
\\
& \Var{veranstaltung["verwendbarkeit"].columns.index(y)}
\\[2ex] \hline
\hline \rule[0mm]{0cm}{.6cm}SL: Erreichen von mindestens 50\% der Punkte, die insgesamt durch die Bearbeitung der für die Übung ausgegebenen Computeraufgaben erreicht werden können. \rule[-3mm]{0cm}{0cm}
 &
\makecell[c]{\xmark}
\\
\end{tabularx}




\clearpage\hrule\vskip1pt\hrule
\section*{\Large Praktische Übung zu Numerik}
\addcontentsline{toc}{subsection}{Praktische Übung zu Numerik\ \textcolor{gray}{(\em Sören Bartels)}}
\vskip-2ex
{\it Sören Bartels, Assistenz: Tatjana Schreiber}
\hfill{D, 3.0 ECTS}\\
Praktische Übung: 2-stündig 14-täglich, verschiedene Termine \\
\subsubsection*{\large
    Inhalt:
}
In den begleitenden praktischen Übungen zur Vorlesung Numerik I werden die in der Vorlesung entwickelten und analysierten Algorithmen praktisch umgesetzt und experimentell getestet. Die Implementierung erfolgt in den Programmiersprachen Matlab, C++ und Python. Elementare Programmierkenntnisse werden dabei vorausgesetzt.
\subsubsection*{\large
    Vorkenntnisse:
}
Siehe bei der Vorlesung {\em Numerik I} (die gleichzeitig gehört werden oder schon absolviert sein soll).
Zusätzlich: Elementare Programmiervorkenntnisse zum Beispiel aus dem Kurs {\em Einführung in die Programmierung für Studierende der Naturwissenschaften}.
\subsubsection*{\large
    Bemerkungen:
}
This course is only offered in German.
\subsubsection*{\large
    Verwendbarkeit, Studien- und Prüfungsleistungen:
}

\begin{tabularx}{\textwidth}{ p{.5\textwidth}
    |X
}
 &
\makecell[c]{\rotatebox[origin=l]{90}{\parbox{
            4
            cm}{\begin{flushleft}
                Mathematische Ergänzung (MEd18) (3.0 ECTS) \newline Numerik (BSc21) (3.0 ECTS) \newline Praktische Übung (2HfB21, MEH21, MEB21) (3.0 ECTS) \newline Wahlmodul (Option ''Individuelle Studiengestaltung'') (2HfB21) (3.0 ECTS)
            \end{flushleft} }}}
\\
& \Var{veranstaltung["verwendbarkeit"].columns.index(y)}
\\[2ex] \hline
\hline \rule[0mm]{0cm}{.6cm}SL: Erreichen von mindestens 50\% der Punkte, die insgesamt durch die Bearbeitung der für die Übung ausgegebenen Computeraufgaben erreicht werden können. \rule[-3mm]{0cm}{0cm}
 &
\makecell[c]{\xmark}
\\
\end{tabularx}




\clearpage
\phantomsection
\thispagestyle{empty}
\vspace*{\fill}
\begin{center}
    \Huge\bfseries 3a. Proseminare
\end{center}
\addcontentsline{toc}{section}{\textbf{3a. Proseminare}}
\addtocontents{toc}{\medskip\hrule\medskip}\vspace*{\fill}\vspace*{\fill}\clearpage
\vfill
\thispagestyle{empty}
\clearpage

\clearpage\hrule\vskip1pt\hrule
\section*{\Large \href{ https://home.mathematik.uni-freiburg.de/knies/lehre/ws2425/index.html}{Gewöhnliche Differentialgleichungen und Anwendungen}}
\addcontentsline{toc}{subsection}{Gewöhnliche Differentialgleichungen und Anwendungen\ \textcolor{gray}{(\em Susanne Knies, Ludwig Striet)}}
\vskip-2ex
{\it Susanne Knies, Ludwig Striet}
\hfill{D, 3.0 ECTS}\\
Seminar: Do, 12--14 Uhr, SR 125, \href{https://www.openstreetmap.org/?mlat=48.000637&mlon=7.846006#map=19/48.000636/7.846006}{Ernst-Zermelo-Str. 1}\\
Vorbesprechung 15.07., 13 Uhr, SR 403, \href{https://www.openstreetmap.org/?mlat=48.000637&mlon=7.846006#map=19/48.000636/7.846006}{Ernst-Zermelo-Str. 1}\\
% Webseite: \url{ https://home.mathematik.uni-freiburg.de/knies/lehre/ws2425/index.html}
\subsubsection*{\large
    Inhalt:
}
Zahlreiche dynamische Prozesse in den Naturwissenschaften können durch Gewöhnliche Differentialgleichungen modelliert werden. In diesem Proseminar beschäftigen wir uns mit expliziten Lösungsmethoden für Differentialgleichungen sowie den Anwendungssituationen (Reaktionskinetik, Räuber-Beute Modelle, Mathematisches Pendel, unterschiedliche Wachstumprozesse, . . . ) die durch sie beschrieben werden.
\subsubsection*{\large
    Literatur:
}
Vortragsthemen und Literatur finden Sie auf der Webseite!
\subsubsection*{\large
    Vorkenntnisse:
}
Analysis~I und II, Lineare Algebra~I und II
\subsubsection*{\large
    Bemerkungen:
}
Note that this course is only offered in German.
\subsubsection*{\large
    Verwendbarkeit, Studien- und Prüfungsleistungen:
}

\begin{tabularx}{\textwidth}{ p{.5\textwidth}
    |X
}
 &
\makecell[c]{\rotatebox[origin=l]{90}{\parbox{
            4
            cm}{\begin{flushleft}
                Proseminar (2HfB21, BSc21, MEH21, MEB21) (3.0 ECTS)
            \end{flushleft} }}}
\\
& \Var{veranstaltung["verwendbarkeit"].columns.index(y)}
\\[2ex] \hline
\hline \rule[0mm]{0cm}{.6cm}PL: Etwa 45- bis 90-minütiger Vortrag. \rule[-3mm]{0cm}{0cm}
 &
\makecell[c]{\xmark}
\\
\hline \rule[0mm]{0cm}{.6cm}SL: Regelmäßige Teilnahme an der Veranstaltung (wie in der Prüfungsordnung definiert). \rule[-3mm]{0cm}{0cm}
 &
\makecell[c]{\xmark}
\\
\end{tabularx}




\clearpage\hrule\vskip1pt\hrule
\section*{\Large \href{https://www.stochastik.uni-freiburg.de/de/lehre/ws-2024-2025/proseminar-streifzug-mathematik-ws-2024-2025/info-proseminar-streifzug-mathematik-ws-2024-2025}{Ein Streifzug durch die Mathematik}}
\addcontentsline{toc}{subsection}{Ein Streifzug durch die Mathematik\ \textcolor{gray}{(\em Angelika Rohde)}}
\vskip-2ex
{\it Angelika Rohde, Assistenz: Johannes Brutsche}
\hfill{D, 3.0 ECTS}\\
Seminar: Mi, 12--14 Uhr, SR 125, \href{https://www.openstreetmap.org/?mlat=48.000637&mlon=7.846006#map=19/48.000636/7.846006}{Ernst-Zermelo-Str. 1}\\
Voranmeldung \newline bei Frau Lippek im Sekretariat der Abteilung für Stochastik (Raum 245)\\
Vorbesprechung 16.07., 10: 15 Uhr, Raum 232, \href{https://www.openstreetmap.org/?mlat=48.000637&mlon=7.846006#map=19/48.000636/7.846006}{Ernst-Zermelo-Str. 1}\\
% Webseite: \url{https://www.stochastik.uni-freiburg.de/de/lehre/ws-2024-2025/proseminar-streifzug-mathematik-ws-2024-2025/info-proseminar-streifzug-mathematik-ws-2024-2025}
\subsubsection*{\large
    Inhalt:
}
Paul Erdős erzählte gerne von dem BUCH, in dem Gott die \textit{perfekten} Beweise für mathematische Sätze aufbewahrt, dem berühmten Zitat von G. H. Hardy entsprechend, dass es für hässliche Mathematik keinen dauerhaften Platz gibt' ([1], Vorwort). Im Versuch einer Bestapproximation an dieses BUCH haben Aigner und Ziegler in ihrem gleichnamigen Werk eine große Anzahl von Sätzen mit eleganten, raffinierten und teils überraschenden Beweisen zusammengetragen.\\ 
In diesem Proseminar soll eine Auswahl dieser Resultate vorgestellt werden. Das Spektrum der Themen erstreckt sich dabei über ganz verschiedenen Gebiete der Mathematik, von Zahlentheorie, Geometrie, Analysis und Kombinatorik bis hin zu Graphentheorie und umfasst namhafte Resultate, wie das Lemma von Littlewood und Offord, das Dinitz-Problem, Hilberts drittes Problem (seiner 23 beim Internationalen Mathematikerkongress in Paris 1900 vorgestellten Probleme), die Borsuk-Vermutung und viele mehr.
\subsubsection*{\large
    Literatur:
}
[1] Martin Aigner, Günter M. Ziegler: \emph{Das BUCH der Beweise} (5. Auf"|lage), Springer, 2018.
\subsubsection*{\large
    Vorkenntnisse:
}
Lineare Algebra~I und II, Analysis~I und II
\subsubsection*{\large
    Bemerkungen:
}
Note that this course is only offered in German.
\subsubsection*{\large
    Verwendbarkeit, Studien- und Prüfungsleistungen:
}

\begin{tabularx}{\textwidth}{ p{.5\textwidth}
    |X
}
 &
\makecell[c]{\rotatebox[origin=l]{90}{\parbox{
            4
            cm}{\begin{flushleft}
                Proseminar (2HfB21, BSc21, MEH21, MEB21) (3.0 ECTS)
            \end{flushleft} }}}
\\
& \Var{veranstaltung["verwendbarkeit"].columns.index(y)}
\\[2ex] \hline
\hline \rule[0mm]{0cm}{.6cm}PL: Etwa 45- bis 90-minütiger Vortrag. \rule[-3mm]{0cm}{0cm}
 &
\makecell[c]{\xmark}
\\
\hline \rule[0mm]{0cm}{.6cm}SL: Regelmäßige Teilnahme an der Veranstaltung (wie in der Prüfungsordnung definiert). \rule[-3mm]{0cm}{0cm}
 &
\makecell[c]{\xmark}
\\
\end{tabularx}




\clearpage\hrule\vskip1pt\hrule
\section*{\Large \href{ https://home.mathematik.uni-freiburg.de/soergel/ws2425ps.html}{Proseminar zur Algebra}}
\addcontentsline{toc}{subsection}{Proseminar zur Algebra\ \textcolor{gray}{(\em Wolfgang Soergel)}}
\vskip-2ex
{\it Wolfgang Soergel, Assistenz: Damian Sercombe}
\hfill{D, 3.0 ECTS}\\
Seminar: Di, 14--16 Uhr, SR 127, \href{https://www.openstreetmap.org/?mlat=48.000637&mlon=7.846006#map=19/48.000636/7.846006}{Ernst-Zermelo-Str. 1}\\
Voranmeldung \newline bis 14.07. per E-Mail an \href{mailto:wolfgang.soergel@math.uni-freiburg.de}{Wolfgang Soergel}\\
% Webseite: \url{ https://home.mathematik.uni-freiburg.de/soergel/ws2425ps.html}
\subsubsection*{\large
    Inhalt:
}
In diesem Proseminar sollen Themen besprochen werden, die ich aus verschiedenen Lehrbüchern und Skripten zu Grundvorlesungen in Linearer Algebra zusammensuche, die aber nicht zum Standardstoff gehören. Die Vorträge bauen dabei nur wenig aufeinander auf. 
\subsubsection*{\large
    Vorkenntnisse:
}
Lineare Algebra ~I und II, Analysis~I und II.
\subsubsection*{\large
    Bemerkungen:
}
This course is only offered in German.
\subsubsection*{\large
    Verwendbarkeit, Studien- und Prüfungsleistungen:
}

\begin{tabularx}{\textwidth}{ p{.5\textwidth}
    |X
}
 &
\makecell[c]{\rotatebox[origin=l]{90}{\parbox{
            4
            cm}{\begin{flushleft}
                Proseminar (2HfB21, BSc21, MEH21, MEB21) (3.0 ECTS)
            \end{flushleft} }}}
\\
& \Var{veranstaltung["verwendbarkeit"].columns.index(y)}
\\[2ex] \hline
\hline \rule[0mm]{0cm}{.6cm}PL: Etwa 45- bis 90-minütiger Vortrag. \rule[-3mm]{0cm}{0cm}
 &
\makecell[c]{\xmark}
\\
\hline \rule[0mm]{0cm}{.6cm}SL: Regelmäßige Teilnahme an der Veranstaltung (wie in der Prüfungsordnung definiert). \rule[-3mm]{0cm}{0cm}
 &
\makecell[c]{\xmark}
\\
\end{tabularx}




\clearpage
\phantomsection
\thispagestyle{empty}
\vspace*{\fill}
\begin{center}
    \Huge\bfseries 3b. Seminare
\end{center}
\addcontentsline{toc}{section}{\textbf{3b. Seminare}}
\addtocontents{toc}{\medskip\hrule\medskip}\vspace*{\fill}\vspace*{\fill}\clearpage
\vfill
\thispagestyle{empty}
\clearpage

\clearpage\hrule\vskip1pt\hrule
\section*{\Large \href{https://www.stochastik.uni-freiburg.de/de/lehre/ws-2024-2025/seminar-knotentheorie-ws-2024-2025}{Knotentheorie}}
\addcontentsline{toc}{subsection}{Knotentheorie\ \textcolor{gray}{(\em Ernst August v. Hammerstein)}}
\vskip-2ex
{\it Ernst August v. Hammerstein}
\hfill{D, 3.0 ECTS}\\
Seminar \newline geplant als Blockseminar nach dem Praxissemester, entweder mit wöchentlichen Terminen ab Januar 2025 oder als Blockseminar zum/nach Ende der Vorlesungszeit.\\
Voranmeldung \newline bis spätestens 18.07.2024 per Mail an \href{mailto:ernst.august.hammerstein@stochastik.uni-freiburg.de}{Ernst August v. Hammerstein}\\
Vorbesprechung 19.07., 16 Uhr, Raum 232, \href{https://www.openstreetmap.org/?mlat=48.000637&mlon=7.846006#map=19/48.000636/7.846006}{Ernst-Zermelo-Str. 1}\\
% Webseite: \url{https://www.stochastik.uni-freiburg.de/de/lehre/ws-2024-2025/seminar-knotentheorie-ws-2024-2025}
\subsubsection*{\large
    Inhalt:
}
Einen Knoten kann man mathematisch relativ einfach definieren als eine geschlossene Kurve im dreidimenionalen Raum $\mathbb{R}^3$. Aus dem täglichen Leben kennt man sicherlich bereits verschiedene Knotenarten, z.\,B. Kreuzknoten, Chirurgenknoten, Seemannsknoten u.a.m. Ziel der mathematischen Knotentheorie ist, charakteristische Größen zur Beschreibung und Klassifizierung von Knoten zu finden und damit evtl. auch entscheiden zu können, ob zwei Knoten äquivalent sind, d.\,h. durch bestimmte Operationen ineinander überführt werden können.

Mit Seilen, Schnüren oder Drähten kann man Knoten sowie einzelne Verknüfungen und Verschlingungen gut veranschaulichen, so dass angehende Lehrerinnen und Lehrer nicht nur in diesem Seminar, sondern vielleicht auch später einmal im Unterricht die Möglichkeit haben, das eine oder andere Resultat ganz praktisch darzustellen.
\subsubsection*{\large
    Literatur:
}
\begin{itemize} 
\item C.C. Adams: \textit{The Knot Book: An elementary introduction to the mathematical theory of knots}, Revised reprint, AMS, 2004.\\
Eine pdf-Datei des zuerst beim W.H. Freeman-Verlag erschienenen Buches findet man unter \url{https://www.math.cuhk.edu.hk/course\_builder/1920/math4900e/Adams--The\%20Knot\%20Book.pdf}.            
\item G. Burde, H. Zieschang: \href{https://www.maths.ed.ac.uk/~v1ranick/papers/burdzies.pdf}{\textit{Knots}} (Second Revides and Extended Edition), de Gruyter, 2003.
\item W.B.R. Lickorish: \href{http://www.redi-bw.de/start/unifr/EBooks-springer/10.1007/978-1-4612-0691-0}{\textit{An Introduction to Knot Theory}}, Springer, 1997.
\item C. Livingston: \href{https://www.math.cuhk.edu.hk/course\_builder/1920/math4900e/Livingston\%20C.---Knot\%20theory\%20(MAA,\%201996).pdf}{\textit{Knot Theory}}. Mathematical Association of America, 1993.
\end{itemize}
\subsubsection*{\large
    Vorkenntnisse:
}
Grundvorlesungen, evtl. auch ein wenig Topologie
\subsubsection*{\large
    Bemerkungen:
}
Restplätze können als Proseminarplätze vergeben werden.
\subsubsection*{\large
    Verwendbarkeit, Studien- und Prüfungsleistungen:
}

\begin{tabularx}{\textwidth}{ p{.5\textwidth}
    |X
    |X
}
 &
\makecell[c]{\rotatebox[origin=l]{90}{\parbox{
            4
            cm}{\begin{flushleft}
                Mathematische Ergänzung (MEd18) (3.0 ECTS) \newline Wahlmodul (Option ''Individuelle Studiengestaltung'') (2HfB21) (3.0 ECTS)
            \end{flushleft} }}}
 &
\makecell[c]{\rotatebox[origin=l]{90}{\parbox{
            4
            cm}{\begin{flushleft}
                Proseminar (2HfB21, BSc21, MEH21, MEB21) (3.0 ECTS)
            \end{flushleft} }}}
\\
& \Var{veranstaltung["verwendbarkeit"].columns.index(y)}
& \Var{veranstaltung["verwendbarkeit"].columns.index(y)}
\\[2ex] \hline
\hline \rule[0mm]{0cm}{.6cm}PL: Etwa 45- bis 90-minütiger Vortrag. \rule[-3mm]{0cm}{0cm}
 &
 &
\makecell[c]{\xmark}
\\
\hline \rule[0mm]{0cm}{.6cm}SL: Etwa 45- bis 90-minütiger Vortrag. \rule[-3mm]{0cm}{0cm}
 &
\makecell[c]{\xmark}
 &
\\
\hline \rule[0mm]{0cm}{.6cm}SL: Regelmäßige Teilnahme an der Veranstaltung (wie in der Prüfungsordnung definiert). \rule[-3mm]{0cm}{0cm}
 &
\makecell[c]{\xmark}
 &
\makecell[c]{\xmark}
\\
\end{tabularx}




\clearpage\hrule\vskip1pt\hrule
\section*{\Large Maschinelles Lernen und Stochastische Analysis}
\addcontentsline{toc}{subsection}{Maschinelles Lernen und Stochastische Analysis\ \textcolor{gray}{(\em Thorsten Schmidt)}}
\vskip-2ex
{\it Thorsten Schmidt, Assistenz: Moritz Ritter}
\hfill{D/E, 6.0 ECTS}\\
Seminar: Fr, 10--12 Uhr, SR 125, \href{https://www.openstreetmap.org/?mlat=48.000637&mlon=7.846006#map=19/48.000636/7.846006}{Ernst-Zermelo-Str. 1}\\
Voranmeldung \newline per E-Mail an \href{mailto:thorsten.schmidt@stochastik.uni-freiburg.de}{Thorsten Schmidt}\\
Vorbesprechung 18.10.\\
\subsubsection*{\large
    Inhalt:
}
Dieses Seminar wird sich auf theoretische Ergebnisse des maschinellen Lernens konzentrieren, einschließlich moderner universeller Approximationssätze, der Näherung von Filtermethoden durch Transformationen, der Anwendung von Methoden des maschinellen Lernens in Finanzmärkten und möglicherweise anderen verwandten Themen. Darüber hinaus werden wir Themen der stochastischen Analyse behandeln, wie die fraktionale Itô-Kalkulation, Unsicherheit, Filterung und optimalen Transport. Sie sind auch eingeladen, Themen vorzuschlagen.
\subsubsection*{\large
    Vorkenntnisse:
}
Das Seminar richtet sich an Studierende, die mindestens Stochastik und Maschinelles Lernen oder Wahrscheinlichkeitstheorie II gehört haben.
\subsubsection*{\large
    Bemerkungen:
}
Bei Interesse und vorhandenen Vorkenntnissen kann ein Seminar auch als Proseminar eingesetzt werden.
\subsubsection*{\large
    Verwendbarkeit, Studien- und Prüfungsleistungen:
}

\begin{tabularx}{\textwidth}{ p{.5\textwidth}
    |X
    |X
}
 &
\makecell[c]{\rotatebox[origin=l]{90}{\parbox{
            4
            cm}{\begin{flushleft}
                Elective in Data (MScData24) (6.0 ECTS) \newline Mathematisches Seminar (MSc14, BSc21) (6.0 ECTS) \newline Mathematisches Seminar (MScData24) (6.0 ECTS) \newline Wahlpflichtmodul Mathematik (BSc21) (6.0 ECTS)
            \end{flushleft} }}}
 &
\makecell[c]{\rotatebox[origin=l]{90}{\parbox{
            4
            cm}{\begin{flushleft}
                Mathematische Ergänzung (MEd18) (3.0 ECTS) \newline Wahlmodul (MSc14) (6.0 ECTS) \newline Wahlmodul (Option ''Individuelle Studiengestaltung'') (2HfB21) (6.0 ECTS)
            \end{flushleft} }}}
\\
& \Var{veranstaltung["verwendbarkeit"].columns.index(y)}
& \Var{veranstaltung["verwendbarkeit"].columns.index(y)}
\\[2ex] \hline
\hline \rule[0mm]{0cm}{.6cm}PL: Etwa 45- bis 90-minütiger Vortrag. \rule[-3mm]{0cm}{0cm}
 &
\makecell[c]{\xmark}
 &
\\
\hline \rule[0mm]{0cm}{.6cm}SL: Etwa 45- bis 90-minütiger Vortrag. \rule[-3mm]{0cm}{0cm}
 &
 &
\makecell[c]{\xmark}
\\
\hline \rule[0mm]{0cm}{.6cm}SL: Regelmäßige Teilnahme an der Veranstaltung (wie in der Prüfungsordnung definiert). \rule[-3mm]{0cm}{0cm}
 &
\makecell[c]{\xmark}
 &
\makecell[c]{\xmark}
\\
\end{tabularx}




\clearpage\hrule\vskip1pt\hrule
\section*{\Large Machine-Learning Methods in the Approximation of PDEs}
\addcontentsline{toc}{subsection}{Machine-Learning Methods in the Approximation of PDEs\ \textcolor{gray}{(\em Sören Bartels)}}
\vskip-2ex
{\it Sören Bartels, Assistenz: Tatjana Schreiber}
\hfill{D/E, 6.0 ECTS}\\
Seminar \newline  geplant als Blockseminar\\
Voranmeldung \newline per E-Mail an \href{mailto:bartels@mathematik.uni-freiburg.de}{Sören Bartels}\\
Vorbesprechung 08.07., 12: 30 Uhr, Büro 209, \href{https://www.openstreetmap.org/?mlat=48.003472&mlon=7.848195#map=19/48.003472/7.848195}{Hermann-Herder-Str. 10}\\
\subsubsection*{\large
    Inhalt:
}
In jüngster Zeit wurden Methoden des maschinellen Lernens zur Annäherung von Lösungen von partiellen Differentialgleichungen verwendet. Während sie in einigen Fällen zu zu Vorteilen gegenüber klassischen Ansätzen führen, ist ihre generelle Überlegenheit noch weitgehend offen. In diesem Seminar werden wir die wichtigsten Konzepte und jüngsten Entwicklungen besprechen.
\subsubsection*{\large
    Literatur:
}
\begin{itemize}
\item
B. Bohn, J. Garcke, M. Griebel: \emph{Algorithmic Mathematics in Machine Learning}, SIAM, 2024.
\item
P. C. Petersen: \emph{Neural Network Theory}, Lecture Notes, 2022. 
\end{itemize}
\subsubsection*{\large
    Vorkenntnisse:
}
Einführung in Theorie und Numerik partieller Differentialgleichungen
\subsubsection*{\large
    Bemerkungen:
}
Bei Interesse und vorhandenen Vorkenntnissen kann ein Seminar auch als Proseminar eingesetzt werden.
\subsubsection*{\large
    Verwendbarkeit, Studien- und Prüfungsleistungen:
}

\begin{tabularx}{\textwidth}{ p{.5\textwidth}
    |X
    |X
}
 &
\makecell[c]{\rotatebox[origin=l]{90}{\parbox{
            4
            cm}{\begin{flushleft}
                Elective in Data (MScData24) (6.0 ECTS) \newline Mathematisches Seminar (MSc14, BSc21) (6.0 ECTS) \newline Mathematisches Seminar (MScData24) (6.0 ECTS) \newline Wahlpflichtmodul Mathematik (BSc21) (6.0 ECTS)
            \end{flushleft} }}}
 &
\makecell[c]{\rotatebox[origin=l]{90}{\parbox{
            4
            cm}{\begin{flushleft}
                Mathematische Ergänzung (MEd18) (3.0 ECTS) \newline Wahlmodul (MSc14) (6.0 ECTS) \newline Wahlmodul (Option ''Individuelle Studiengestaltung'') (2HfB21) (6.0 ECTS)
            \end{flushleft} }}}
\\
& \Var{veranstaltung["verwendbarkeit"].columns.index(y)}
& \Var{veranstaltung["verwendbarkeit"].columns.index(y)}
\\[2ex] \hline
\hline \rule[0mm]{0cm}{.6cm}PL: Etwa 45- bis 90-minütiger Vortrag. \rule[-3mm]{0cm}{0cm}
 &
\makecell[c]{\xmark}
 &
\\
\hline \rule[0mm]{0cm}{.6cm}SL: Etwa 45- bis 90-minütiger Vortrag. \rule[-3mm]{0cm}{0cm}
 &
 &
\makecell[c]{\xmark}
\\
\hline \rule[0mm]{0cm}{.6cm}SL: Regelmäßige Teilnahme an der Veranstaltung (wie in der Prüfungsordnung definiert). \rule[-3mm]{0cm}{0cm}
 &
\makecell[c]{\xmark}
 &
\makecell[c]{\xmark}
\\
\end{tabularx}




\clearpage\hrule\vskip1pt\hrule
\section*{\Large Medical Data Science}
\addcontentsline{toc}{subsection}{Medical Data Science\ \textcolor{gray}{(\em Harald Binder)}}
\vskip-2ex
{\it Harald Binder}
\hfill{D/E, 6.0 ECTS}\\
Seminar: Mi, 10--11: 30 Uhr, HS Medizinische Biometrie, 1. OG, \href{https://www.openstreetmap.org/?mlat=48.002530&mlon=7.846776#map=19/48.002530/7.846776}{Stefan-Meier-Str. 26}\\
Voranmeldung \newline per E-Mail an \href{mailto:olga.sieber@uniklinik-freiburg.de}{Olga Sieber}\\
Vorbesprechung 17.07., HS Medizinische Biometrie, 1. OG, \href{https://www.openstreetmap.org/?mlat=48.002530&mlon=7.846776#map=19/48.002530/7.846776}{Stefan-Meier-Str. 26}\\
\subsubsection*{\large
    Inhalt:
}
Zur Beantwortung komplexer biomedizinischer Fragestellungen aus großen Datenmengen ist oft ein breites Spektrum an Analysewerkzeugen notwendig, z.B. Deep-Learning- oder allgemeiner Machine-Learning-Techniken, was häufig unter dem Begriff "`Medical Data Science"' zusammengefasst wird. Statistische Ansätze spielen eine wesentliche Rolle als Basis dafür. Eine Auswahl von Ansätzen soll in den Seminarvorträgen vorgestellt werden, die sich an kürzlich erschienenen Originalarbeiten orientieren. Die genaue thematische Ausrichtung wird noch festgelegt.
\subsubsection*{\large
    Literatur:
}
Hinweise auf einführende Literatur werden in der Vorbesprechung gegeben.
\subsubsection*{\large
    Vorkenntnisse:
}
Gute Kenntnisse in Wahrscheinlichkeitstheorie und Mathematischer Statistik.
\subsubsection*{\large
    Bemerkungen:
}
Das Seminar kann als Vorbereitung für eine Bachelor- oder Masterarbeit dienen.\\
Bei Interesse und vorhandenen Vorkenntnissen kann ein Seminar auch als Proseminar eingesetzt werden.
\subsubsection*{\large
    Verwendbarkeit, Studien- und Prüfungsleistungen:
}

\begin{tabularx}{\textwidth}{ p{.5\textwidth}
    |X
    |X
}
 &
\makecell[c]{\rotatebox[origin=l]{90}{\parbox{
            4
            cm}{\begin{flushleft}
                Elective in Data (MScData24) (6.0 ECTS) \newline Mathematisches Seminar (MSc14, BSc21) (6.0 ECTS) \newline Mathematisches Seminar (MScData24) (6.0 ECTS) \newline Wahlpflichtmodul Mathematik (BSc21) (6.0 ECTS)
            \end{flushleft} }}}
 &
\makecell[c]{\rotatebox[origin=l]{90}{\parbox{
            4
            cm}{\begin{flushleft}
                Mathematische Ergänzung (MEd18) (3.0 ECTS) \newline Wahlmodul (MSc14) (6.0 ECTS) \newline Wahlmodul (Option ''Individuelle Studiengestaltung'') (2HfB21) (6.0 ECTS)
            \end{flushleft} }}}
\\
& \Var{veranstaltung["verwendbarkeit"].columns.index(y)}
& \Var{veranstaltung["verwendbarkeit"].columns.index(y)}
\\[2ex] \hline
\hline \rule[0mm]{0cm}{.6cm}PL: Etwa 45- bis 90-minütiger Vortrag. \rule[-3mm]{0cm}{0cm}
 &
\makecell[c]{\xmark}
 &
\\
\hline \rule[0mm]{0cm}{.6cm}SL: Etwa 45- bis 90-minütiger Vortrag. \rule[-3mm]{0cm}{0cm}
 &
 &
\makecell[c]{\xmark}
\\
\hline \rule[0mm]{0cm}{.6cm}SL: Regelmäßige Teilnahme an der Veranstaltung (wie in der Prüfungsordnung definiert). \rule[-3mm]{0cm}{0cm}
 &
\makecell[c]{\xmark}
 &
\makecell[c]{\xmark}
\\
\end{tabularx}




\clearpage\hrule\vskip1pt\hrule
\section*{\Large \href{https://home.mathematik.uni-freiburg.de/analysis/2024_WiSe_Lehre/2024_WiSe_Seminar_MinimalSurfaces/}{Minimalflächen}}
\addcontentsline{toc}{subsection}{Minimalflächen\ \textcolor{gray}{(\em Guofang Wang)}}
\vskip-2ex
{\it Guofang Wang, Assistenz: Xuwen Zhang}
\hfill{D/E, 6.0 ECTS}\\
Seminar: Mi, 16--18 Uhr, SR 125, \href{https://www.openstreetmap.org/?mlat=48.000637&mlon=7.846006#map=19/48.000636/7.846006}{Ernst-Zermelo-Str. 1}\\
Vorbesprechung 17.07., 16 Uhr\\
% Webseite: \url{https://home.mathematik.uni-freiburg.de/analysis/2024_WiSe_Lehre/2024_WiSe_Seminar_MinimalSurfaces/}
\subsubsection*{\large
    Inhalt:
}
Minimalflächen sind Flächen im Raum mit „minimalem“ Flächeninhalt und lassen sich mithilfe holomorpher Funktionen beschreiben. Sie treten u.a. bei der Untersuchung von Seifenhäuten und der Konstruktion stabiler Objekte (z.B. in der Architektur) in Erscheinung. Bei der Untersuchung von Minimalflächen kommen elegante Methoden aus verschiedenen mathematischen Gebieten wie der Funktionentheorie, der Variationsrechnung, der Differentialgeometrie und der partiellen Differentialgleichung zur Anwendung.
\subsubsection*{\large
    Literatur:
}
\begin{itemize}
\item
R. Osserman: \emph{A survey of minimal surfaces}, Van Nostrand 1969. 
\item
J.-H. Eschenburg, J. Jost: \emph{Differentialgeometrie und Minimalflächen}, Springer 2007.
\item
E. Kuwert: \emph{Einführung in die Theorie der Minimalflächen}, Skript 1998.
\item
W. H. Meeks III, J. Pérez: \emph{A survey on classical minimal surface theory}.
\item
T. Colding, W. P. Minicozzi: \emph{Minimal Surfaces}, New York University 1999.
\end{itemize}
\subsubsection*{\large
    Vorkenntnisse:
}
Notwendig: Analysis III oder Mehrfachintegrale, und Funktionentheorie \\ Nützlich: Elementare Differentialgeometrie
\subsubsection*{\large
    Bemerkungen:
}
Bei Interesse und vorhandenen Vorkenntnissen kann ein Seminar auch als Proseminar eingesetzt werden.
\subsubsection*{\large
    Verwendbarkeit, Studien- und Prüfungsleistungen:
}

\begin{tabularx}{\textwidth}{ p{.5\textwidth}
    |X
    |X
}
 &
\makecell[c]{\rotatebox[origin=l]{90}{\parbox{
            4
            cm}{\begin{flushleft}
                Mathematische Ergänzung (MEd18) (3.0 ECTS) \newline Wahlmodul (MSc14) (6.0 ECTS) \newline Wahlmodul (MScData24) (6.0 ECTS) \newline Wahlmodul (Option ''Individuelle Studiengestaltung'') (2HfB21) (6.0 ECTS)
            \end{flushleft} }}}
 &
\makecell[c]{\rotatebox[origin=l]{90}{\parbox{
            4
            cm}{\begin{flushleft}
                Mathematisches Seminar (MSc14, BSc21) (6.0 ECTS) \newline Wahlpflichtmodul Mathematik (BSc21) (6.0 ECTS)
            \end{flushleft} }}}
\\
& \Var{veranstaltung["verwendbarkeit"].columns.index(y)}
& \Var{veranstaltung["verwendbarkeit"].columns.index(y)}
\\[2ex] \hline
\hline \rule[0mm]{0cm}{.6cm}PL: Etwa 45- bis 90-minütiger Vortrag. \rule[-3mm]{0cm}{0cm}
 &
 &
\makecell[c]{\xmark}
\\
\hline \rule[0mm]{0cm}{.6cm}SL: Etwa 45- bis 90-minütiger Vortrag. \rule[-3mm]{0cm}{0cm}
 &
\makecell[c]{\xmark}
 &
\\
\hline \rule[0mm]{0cm}{.6cm}SL: Regelmäßige Teilnahme an der Veranstaltung (wie in der Prüfungsordnung definiert). \rule[-3mm]{0cm}{0cm}
 &
\makecell[c]{\xmark}
 &
\makecell[c]{\xmark}
\\
\end{tabularx}




\clearpage\hrule\vskip1pt\hrule
\section*{\Large \href{ https://home.mathematik.uni-freiburg.de/geometrie/lehre/ws2024/Seminar/}{Seminar zur algebraischen Topologie}}
\addcontentsline{toc}{subsection}{Seminar zur algebraischen Topologie\ \textcolor{gray}{(\em Sebastian Goette)}}
\vskip-2ex
{\it Sebastian Goette, Assistenz: Mikhael Tëmkin}
\hfill{D/E, 6.0 ECTS}\\
Seminar: Di, 14--16 Uhr, SR 125, \href{https://www.openstreetmap.org/?mlat=48.000637&mlon=7.846006#map=19/48.000636/7.846006}{Ernst-Zermelo-Str. 1}\\
Vorbesprechung 16.07., SR 125, \href{https://www.openstreetmap.org/?mlat=48.000637&mlon=7.846006#map=19/48.000636/7.846006}{Ernst-Zermelo-Str. 1}\\
% Webseite: \url{ https://home.mathematik.uni-freiburg.de/geometrie/lehre/ws2024/Seminar/}
\subsubsection*{\large
    Inhalt:
}
Wir besprechen fortgeschrittene Themen der algebraischen Topologie.
Je nach Interesse der Teilnehmer könnten wir eines der folgenden Themen bearbeiten - wenn Sie andere Themenvroschläge haben, wenden Sie sich bitte an den Dozenten.
\begin{itemize}
\item Die Steenrod-Algebra. Eine Zusatzstruktur auf der Kohomologie modulo $p$
ermöglicht feinere Aussagen zur Existenz stetiger Abbildungen, etwa zur Existenz linear unabhängiger Vektorfelder auf Sphären. Die Wu-Formeln stellen einen Zusammenhang zu charakteristischen Klassen von Mannigfaltigkeiten her.
\item  Strukturierte Spektren. Um multiplikative (Ko-) Homologiefunktoren
durch Spektren darstellen zu können, braucht man eine abgeschlossene monoidale Kategorie von Spektren, beispielsweise
symmetrische oder orthogonale Spektren. In diesem Zusammenhang
lernen wir auch Modellstrukturen besser kennen.
\item $K$-Theorie und Indextheorie. Elliptische Differentialoperatoren auf kompakten Mannigfaltigkeiten sind Fredholm-Operatoren. Ihr Index lässt sich mit dem Satz von Atiyah-Singer topologische berechnen. Wir beweisen diesen Satz mit (überwiegend) topologischen Methoden und geben einige geometrische Anwendungen.
\end{itemize} 
\subsubsection*{\large
    Vorkenntnisse:
}
Algebraische Topologie~I und II
\subsubsection*{\large
    Bemerkungen:
}
TeilnehmerInnen übernehmen einen, bei Interesse auch mehrere Vorträge.
Für die restliche Zeit setzen wir die Veranstaltung als Lesekurs oder Spezialvorlesung fort. \\
Bei Interesse kann das Seminar auf Englisch stattfinden.
Bei Interesse und vorhandenen Vorkenntnissen kann ein Seminar auch als Proseminar eingesetzt werden.
\subsubsection*{\large
    Verwendbarkeit, Studien- und Prüfungsleistungen:
}

\begin{tabularx}{\textwidth}{ p{.5\textwidth}
    |X
    |X
}
 &
\makecell[c]{\rotatebox[origin=l]{90}{\parbox{
            4
            cm}{\begin{flushleft}
                Mathematische Ergänzung (MEd18) (3.0 ECTS) \newline Wahlmodul (MSc14) (6.0 ECTS) \newline Wahlmodul (MScData24) (6.0 ECTS) \newline Wahlmodul (Option ''Individuelle Studiengestaltung'') (2HfB21) (6.0 ECTS)
            \end{flushleft} }}}
 &
\makecell[c]{\rotatebox[origin=l]{90}{\parbox{
            4
            cm}{\begin{flushleft}
                Mathematisches Seminar (MSc14, BSc21) (6.0 ECTS) \newline Wahlpflichtmodul Mathematik (BSc21) (6.0 ECTS)
            \end{flushleft} }}}
\\
& \Var{veranstaltung["verwendbarkeit"].columns.index(y)}
& \Var{veranstaltung["verwendbarkeit"].columns.index(y)}
\\[2ex] \hline
\hline \rule[0mm]{0cm}{.6cm}PL: Etwa 45- bis 90-minütiger Vortrag. \rule[-3mm]{0cm}{0cm}
 &
 &
\makecell[c]{\xmark}
\\
\hline \rule[0mm]{0cm}{.6cm}SL: Etwa 45- bis 90-minütiger Vortrag. \rule[-3mm]{0cm}{0cm}
 &
\makecell[c]{\xmark}
 &
\\
\hline \rule[0mm]{0cm}{.6cm}SL: Regelmäßige Teilnahme an der Veranstaltung (wie in der Prüfungsordnung definiert). \rule[-3mm]{0cm}{0cm}
 &
\makecell[c]{\xmark}
 &
\makecell[c]{\xmark}
\\
\end{tabularx}




\clearpage\hrule\vskip1pt\hrule
\section*{\Large \href{http://home.mathematik.uni-freiburg.de/arithgeom/lehre.html}{Theorie der nicht-kommutativen Algebren}}
\addcontentsline{toc}{subsection}{Theorie der nicht-kommutativen Algebren\ \textcolor{gray}{(\em Annette Huber-Klawitter)}}
\vskip-2ex
{\it Annette Huber-Klawitter, Assistenz: Xier Ren}
\hfill{D/E, 6.0 ECTS}\\
Seminar: Fr, 8--10 Uhr, SR 404, \href{https://www.openstreetmap.org/?mlat=48.000637&mlon=7.846006#map=19/48.000636/7.846006}{Ernst-Zermelo-Str. 1}\\
Voranmeldung \newline per E-Mail an \href{mailto:ludmilla.frei@math.uni-freiburg.de}{Ludmilla Frei} oder persönlich in Raum 421\\
Vorbesprechung 15.07., 11 Uhr, SR 318, \href{https://www.openstreetmap.org/?mlat=48.000637&mlon=7.846006#map=19/48.000636/7.846006}{Ernst-Zermelo-Str. 1}\\
% Webseite: \url{http://home.mathematik.uni-freiburg.de/arithgeom/lehre.html}
\subsubsection*{\large
    Inhalt:
}
In this seminar, we are going to study finite dimensional (unital, possibly non-commutative) algebras over a (commutative) field $k$. Prototypes are the rings of square matrices over $k$, finite field extensions, or the algebra $k^n$ with diagonal multiplication. 

We will concentrate on path algebras of finite quivers (German: Köcher). Modules over them are equivalently described as representations of the quiver. Many algebraic properties can be directly understood from properties of the quiver. 

\subsubsection*{\large
    Literatur:
}
\begin{itemize}
\item
Frank Anderson, Kent Fuller: \emph{Rings and Categories of Modules}, GTM 13, Springer, 1992 
\item
Ralf Schiffler: \emph{Quiver Representations}, CMS Books in Mathematics, Springer, 2014 
\item
Alexander Kirillov Jr.: \emph{Quiver Representations}, GSM 174, AMS, 2016
\end{itemize}
\subsubsection*{\large
    Vorkenntnisse:
}
Notwendig: Lineare Algebra \\
Nützlich: Algebra und Zahlentheorie, kommutative Algebra
\subsubsection*{\large
    Bemerkungen:
}
Die Verständigung mit dem Assistenten erfolgt auf Englisch. Vorträge können auf Deutsch oder Englisch gehalten werden. \\
Bei Interesse und vorhandenen Vorkenntnissen kann ein Seminar auch als Proseminar eingesetzt werden.
\subsubsection*{\large
    Verwendbarkeit, Studien- und Prüfungsleistungen:
}

\begin{tabularx}{\textwidth}{ p{.5\textwidth}
    |X
    |X
}
 &
\makecell[c]{\rotatebox[origin=l]{90}{\parbox{
            4
            cm}{\begin{flushleft}
                Mathematische Ergänzung (MEd18) (3.0 ECTS) \newline Wahlmodul (MSc14) (6.0 ECTS) \newline Wahlmodul (MScData24) (6.0 ECTS) \newline Wahlmodul (Option ''Individuelle Studiengestaltung'') (2HfB21) (6.0 ECTS)
            \end{flushleft} }}}
 &
\makecell[c]{\rotatebox[origin=l]{90}{\parbox{
            4
            cm}{\begin{flushleft}
                Mathematisches Seminar (MSc14, BSc21) (6.0 ECTS) \newline Wahlpflichtmodul Mathematik (BSc21) (6.0 ECTS)
            \end{flushleft} }}}
\\
& \Var{veranstaltung["verwendbarkeit"].columns.index(y)}
& \Var{veranstaltung["verwendbarkeit"].columns.index(y)}
\\[2ex] \hline
\hline \rule[0mm]{0cm}{.6cm}PL: Etwa 45- bis 90-minütiger Vortrag. \rule[-3mm]{0cm}{0cm}
 &
 &
\makecell[c]{\xmark}
\\
\hline \rule[0mm]{0cm}{.6cm}SL: Etwa 45- bis 90-minütiger Vortrag. \rule[-3mm]{0cm}{0cm}
 &
\makecell[c]{\xmark}
 &
\\
\hline \rule[0mm]{0cm}{.6cm}SL: Regelmäßige Teilnahme an der Veranstaltung (wie in der Prüfungsordnung definiert). \rule[-3mm]{0cm}{0cm}
 &
\makecell[c]{\xmark}
 &
\makecell[c]{\xmark}
\\
\end{tabularx}




\end{document}