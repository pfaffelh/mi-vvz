\documentclass[a4paper,10pt]{article}
\usepackage[utf8]{inputenc}
\usepackage{latexsym, 
			amsfonts, 
			amssymb, 
			amsthm, 
			calc,
			%fontawesome5,
			graphicx,
			longtable, 
            makecell,
			nicefrac,
			paralist,
            pifont,
            rotating,
            tabularx,
            colortbl
			}
\usepackage[table]{xcolor}
\arrayrulecolor{lightgray} % Ändert die Farbe der Linien
% \arrayrulewidth=1pt   % Setzt die Dicke der Linien

\usepackage{hyperref}
\hypersetup{
  colorlinks = true, %Colours links instead of ugly boxes
  urlcolor = blue, %Colour for external hyperlinks
  linkcolor = blue, %Colour of internal links
  citecolor = blue %Colour of citations
}
\frenchspacing

\newcommand{\mailto}[1]{\href{mailto:#1}{#1}}
\renewenvironment{itemize}{\begin{list}{$\bullet$\ }{\itemsep.5ex\setlength{\topsep}{0.5\itemsep}\parsep0ex\labelsep1ex\settowidth{\labelwidth}{$\bullet$\ }\setlength{\leftmargin}{\labelwidth}\addtolength{\leftmargin}{3ex}\addtolength{\leftmargin}{\labelsep}}}{\end{list}} 
\usepackage{pifont} 
\newcommand{\xmark}{\ding{55}}
\usepackage{draftwatermark}
\SetWatermarkText{}
\SetWatermarkScale{1}

\setlength{\topmargin}{-2.5cm}
\setlength{\oddsidemargin}{-1cm}
\setlength{\textwidth}{18cm}
\setlength{\textheight}{26.5cm}
\parindent0em
\parskip1ex

\begin{document}

\hrule\vskip1pt\hrule\medskip

\resizebox{\textwidth}{!}{
    University of Freiburg -- Mathematical Institute
}

\medskip
\resizebox{\textwidth}{!}{Winter term 2024/25}

\bigskip
\resizebox{\textwidth}{!}{Supplements of the module handbooks}

\medskip\hrule\vskip1pt\hrule

\bigskip
\bigskip


\hfill Version \today

\thispagestyle{empty}
\clearpage
\tableofcontents

\clearpage
\addcontentsline{toc}{section}{{Hinweise}}\addtocontents{toc}{\medskip\hrule\medskip}











\clearpage
\phantomsection
\thispagestyle{empty}
\vspace*{\fill}
\begin{center}
    \Huge\bfseries 1a. Mandatory Lectures of the Study Programmes
\end{center}
\addcontentsline{toc}{section}{\textbf{1a. Mandatory Lectures of the Study Programmes}}
\addtocontents{toc}{\medskip\hrule\medskip}\vspace*{\fill}\vspace*{\fill}\clearpage
\vfill
\thispagestyle{empty}
\clearpage

\clearpage\hrule\vskip1pt\hrule
\section*{\Large \href{https://aam.uni-freiburg.de/agru/lehre/ws24/ana1/index.html}{Analysis I}}
\addcontentsline{toc}{subsection}{Analysis I\ \textcolor{gray}{(\em Michael Růžička)}}
\vskip-2ex
{\it Michael Růžička, Assistant: Alexei Gazca}
\hfill{D}\\
Lecture: Tue, Wed, 8--10 h, HS Rundbau, \href{https://www.openstreetmap.org/?mlat=48.001613&mlon=7.849254#map=19/48.001613/7.849254}{Albertstr. 21}\\
Tutorial: 2 hours, various dates \\
% Webseite: \url{https://aam.uni-freiburg.de/agru/lehre/ws24/ana1/index.html}
\subsubsection*{\large
    Content:
}
Analysis I is one of the two basic lectures in the mathematics course. It deals with concepts based on the notion of limit. The central topics are: induction, real and complex numbers, convergence of sequences and series, completeness, exponential function and trigonometric functions, continuity, derivation of functions of one variable and regulated integrals.
\subsubsection*{\large
    Literature:
}
To be announced in the lecture.
\subsubsection*{\large
    Prerequisites:
}
High school mathematics. \\
Attendance of the preliminary course (for students in mathematics) is recommended.
\subsubsection*{\large
    Remarks:
}
This course is only offered in German.
\subsubsection*{\large
    Usability and assessments:
}

\begin{tabularx}{\textwidth}{ p{.5\textwidth}
    |X
    |X
}
 &
\makecell[c]{\rotatebox[origin=l]{90}{\parbox{
            4
            cm}{\begin{flushleft}
                Analysis (2HfB21, BSc21, MEH21, MEB21) (9.0 ECTS)
            \end{flushleft} }}}
 &
\makecell[c]{\rotatebox[origin=l]{90}{\parbox{
            4
            cm}{\begin{flushleft}
                Analysis I (BScInfo19, BScPhys20) (9.0 ECTS)
            \end{flushleft} }}}
\\
& 1
& 1
\\[2ex] \hline
\hline \rule[0mm]{0cm}{.6cm}PL:  \rule[-3mm]{0cm}{0cm}
 &
\makecell[c]{\xmark}
 &
\\
\hline \rule[0mm]{0cm}{.6cm}SL:  \rule[-3mm]{0cm}{0cm}
 &
\makecell[c]{\xmark}
 &
\makecell[c]{\xmark}
\\
\end{tabularx}


\clearpage\hrule\vskip1pt\hrule
\section*{\Large Linear Algebra I}
\addcontentsline{toc}{subsection}{Linear Algebra I\ \textcolor{gray}{(\em Stefan Kebekus)}}
\vskip-2ex
{\it Stefan Kebekus, Assistant: Marius Amann}
\hfill{D}\\
Lecture: Mon, Thu, 8--10 h, HS Rundbau, \href{https://www.openstreetmap.org/?mlat=48.001613&mlon=7.849254#map=19/48.001613/7.849254}{Albertstr. 21}\\
Tutorial: 2 hours, various dates \\
\subsubsection*{\large
    Content:
}
Linear Algebra I is one of the two introductory lectures in the mathematics degree program that form the basis for further courses. Topics covered include: fundamental concepts (in particular fundamental concepts of set theory and equivalence relations), groups, fields, vector spaces over arbitrary fields, basis and dimension, linear mappings and transformation matrix, matrix calculus, linear systems of equations, Gaussian elimination, linear forms, dual space, quotient vector spaces and homomorphism theorem, determinant, eigenvalues, polynomials, characteristic polynomial, diagonalizability, affine spaces. The background to the mathematical content is explained in terms of ideas and the history of mathematics.
\subsubsection*{\large
    Literature:
}
To be announced in the lecture.
\subsubsection*{\large
    Prerequisites:
}
High school mathematics. \\ Attendance of the preliminary course (for students in mathematics) is recommended.
\subsubsection*{\large
    Remarks:
}
This course is only offered in German.
\subsubsection*{\large
    Usability and assessments:
}

\begin{tabularx}{\textwidth}{ p{.5\textwidth}
    |X
    |X
}
 &
\makecell[c]{\rotatebox[origin=l]{90}{\parbox{
            4
            cm}{\begin{flushleft}
                Linear Algebra (2HfB21, BSc21, MEH21) (9.0 ECTS)
            \end{flushleft} }}}
 &
\makecell[c]{\rotatebox[origin=l]{90}{\parbox{
            4
            cm}{\begin{flushleft}
                Linear Algebra (MEB21) (6.0 ECTS) \newline Linear Algebra I (as a non-subject-related elective module) (BScInfo19, BScPhys20) (9.0 ECTS)
            \end{flushleft} }}}
\\
& 1
& 1
\\[2ex] \hline
\hline \rule[0mm]{0cm}{.6cm}PL:  \rule[-3mm]{0cm}{0cm}
 &
\makecell[c]{\xmark}
 &
\\
\hline \rule[0mm]{0cm}{.6cm}SL:  \rule[-3mm]{0cm}{0cm}
 &
\makecell[c]{\xmark}
 &
\makecell[c]{\xmark}
\\
\end{tabularx}


\clearpage\hrule\vskip1pt\hrule
\section*{\Large Numerics I}
\addcontentsline{toc}{subsection}{Numerics I\ \textcolor{gray}{(\em Sören Bartels)}}
\vskip-2ex
{\it Sören Bartels, Assistant: Tatjana Schreiber}
\hfill{D}\\
Lecture: Wed, 14--16 h, HS Weismann-Haus, \href{https://www.openstreetmap.org/?mlat=48.001683&mlon=7.849361#map=19/48.001683/7.849361}{Albertstr. 21a}\\
Tutorial: 2 hours, every other week, various dates \\
\subsubsection*{\large
    Content:
}
Numerics is a sub-discipline of mathematics that deals with the practical solution of mathematical problems. As a rule, problems are not solved exactly but approximately, for which a sensible compromise between accuracy and computational effort must be found. The first part of the two-semester course focuses on questions of linear algebra such as solving linear systems of equations and determining the eigenvalues of a matrix. Attendance at the accompanying practical exercises ({\em Praktische Übung zur Numerik}) is recommended. These take place every 14 days, alternating with the lecture's tutorial.
\subsubsection*{\large
    Literature:
}
\begin{itemize}
\item
S.~Bartels: \emph{Numerik~3x9}. Springer, 2016.
\item
R.~Plato: \emph{Numerische Mathematik kompakt}. Vieweg, 2006.
\item
R.~Schaback, H.~Wendland: \emph{Numerische Mathematik}. Springer, 2004.
\item
J.~Stoer, R.~Burlisch: \emph{Numerische Mathematik~I, II}. Springer, 2007, 2005.
\item
G.~Hämmerlin, K.-H.~Hoffmann: \emph{Numerische Mathematik}. Springer, 1990.
\item
P.~Deuflhard, A.~Hohmann, F.~Bornemann: \emph{Numerische Mathematik~I, II}. DeGruyter, 2003.
\end{itemize}
\subsubsection*{\large
    Prerequisites:
}
Required: Linear Algebra~I \\
Recommended: Linear Algebra~II and Analysis~I (required for Numerics~II)
\subsubsection*{\large
    Remarks:
}
A computer exercise ({\em Praktische Übung zur Numerik}) is offered to accompany the lecture. \\
This course is only offered in German.
\subsubsection*{\large
    Usability and assessments:
}

\begin{tabularx}{\textwidth}{ p{.5\textwidth}
    |X
    |X
}
 &
\makecell[c]{\rotatebox[origin=l]{90}{\parbox{
            4
            cm}{\begin{flushleft}
                Numerics (2HfB21, MEH21) (4.5 ECTS) \newline Numerics (BSc21) (4.5 ECTS)
            \end{flushleft} }}}
 &
\makecell[c]{\rotatebox[origin=l]{90}{\parbox{
            4
            cm}{\begin{flushleft}
                Numerics I (MEB21) (5.0 ECTS)
            \end{flushleft} }}}
\\
& 1
& 1
\\[2ex] \hline
\hline \rule[0mm]{0cm}{.6cm}Kommentar:  \rule[-3mm]{0cm}{0cm}
 &
\makecell[c]{\xmark}
 &
\\
\hline \rule[0mm]{0cm}{.6cm}PL:  \rule[-3mm]{0cm}{0cm}
 &
\makecell[c]{\xmark}
 &
\makecell[c]{\xmark}
\\
\hline \rule[0mm]{0cm}{.6cm}SL:  \rule[-3mm]{0cm}{0cm}
 &
\makecell[c]{\xmark}
 &
\makecell[c]{\xmark}
\\
\end{tabularx}


\clearpage\hrule\vskip1pt\hrule
\section*{\Large \href{https://www.stochastik.uni-freiburg.de/de/lehre/ws-2024-2025/vorlesung-stochastik-I-ws-2024-2025}{Elementary Probability Theory I}}
\addcontentsline{toc}{subsection}{Elementary Probability Theory I\ \textcolor{gray}{(\em Angelika Rohde)}}
\vskip-2ex
{\it Angelika Rohde, Assistant: Johannes Brutsche}
\hfill{D}\\
Lecture: Fri, 10--12 h, HS Weismann-Haus, \href{https://www.openstreetmap.org/?mlat=48.001683&mlon=7.849361#map=19/48.001683/7.849361}{Albertstr. 21a}\\
Tutorial: 2 hours, every other week, various dates \\
% Webseite: \url{https://www.stochastik.uni-freiburg.de/de/lehre/ws-2024-2025/vorlesung-stochastik-I-ws-2024-2025}
\subsubsection*{\large
    Content:
}
Stochastic is, to put it loosely, the “mathematics of chance”, about which---possibly contrary to first impressions---many precise and not at all random statements can be formulated and proven. The aim of the lecture is to give an introduction to stochastic modeling, to explain some basic concepts and results of Stochastic and to illustrate them with examples. It is also intended as a motivating preparation for the lecture “Probability Theory” in the summer semester, especially for students in the B.Sc. in Mathematics. Topics covered include: Discrete and continuous random variables, probability spaces and measures, combinatorics, expected value, variance, correlation, generating functions, conditional probability, independence, weak law of large numbers, central limit theorem.
The lecture Elementary Probability Theory~II in the summer semester will mainly be devoted to statistical topics. If you are interested in a practical, computer-supported implementation of individual lecture contents, participation in the regularly offered practical excercise “Praktischen Übung Stochastik" is also recommended (in parallel or subsequently).
\subsubsection*{\large
    Literature:
}
\begin{itemize}
\item L.~Dümbgen: \emph{Stochastik für Informatiker}, Springer, 2003.
\item H.-O.~Georgii: \emph{Stochastik: Einführung in die Wahrscheinlichkeitstheorie und Statistik} (5.~Auf\/lage), De Gruyter, 2015.
\item N.~Henze: \href{https://www.redi-bw.de/start/unifr/EBooks-springer/10.1007/978-3-662-63840-8}{\emph{Stochastik für Einsteiger}}, (13.~Auf\/lage), Springer Spektrum, 2021. 
\item  N.~Henze: \href{https://www.redi-bw.de/start/unifr/EBooks-springer/10.1007/978-3-662-59563-3}{\emph{Stochastik: Eine Einführung mit Grundzügen der Maßtheorie}}, Springer Spektrum, 2019. 
\item  G.~Kersting, A.~Wakolbinger: \href{http://www.redi-bw.de/start/unifr/EBooks-springer/10.1007/978-3-0346-0414-7}{\emph{Elementare Stochastik}} (2. Auf\/lage), Birkhäuser, 2010. 
\end{itemize}
\subsubsection*{\large
    Prerequisites:
}
Required: Linear Algebra~I, Analysis~I and II. \\
Note that Linear Algebra~I can be attended in parallel.
\subsubsection*{\large
    Remarks:
}
This course is only offered in German.
\subsubsection*{\large
    Usability and assessments:
}

\begin{tabularx}{\textwidth}{ p{.5\textwidth}
    |X
}
 &
\makecell[c]{\rotatebox[origin=l]{90}{\parbox{
            4
            cm}{\begin{flushleft}
                Elementary Probability Theory I (BSc21, MEB21, MEdual24) (5.0 ECTS) \newline Elementary Probabilty Theory (2HfB21, MEH21) (4.5 ECTS)
            \end{flushleft} }}}
\\
& 1
\\[2ex] \hline
\hline \rule[0mm]{0cm}{.6cm}Kommentar:  \rule[-3mm]{0cm}{0cm}
 &
\makecell[c]{\xmark}
\\
\hline \rule[0mm]{0cm}{.6cm}PL:  \rule[-3mm]{0cm}{0cm}
 &
\makecell[c]{\xmark}
\\
\hline \rule[0mm]{0cm}{.6cm}SL:  \rule[-3mm]{0cm}{0cm}
 &
\makecell[c]{\xmark}
\\
\end{tabularx}


\clearpage\hrule\vskip1pt\hrule
\section*{\Large Additions to Analysis}
\addcontentsline{toc}{subsection}{Additions to Analysis\ \textcolor{gray}{(\em Nadine Große)}}
\vskip-2ex
{\it Nadine Große, Assistant: Jonah Reuß}
\hfill{D}\\
Lecture: Wed, 8--10 h, HS Weismann-Haus, \href{https://www.openstreetmap.org/?mlat=48.001683&mlon=7.849361#map=19/48.001683/7.849361}{Albertstr. 21a}\\
Tutorial: 2 hours, various dates \\
\subsubsection*{\large
    Content:
}
\textit{Multiple integration:} Jordan content in $\mathbb R^n$, Fubini's theorem, transformation theorem, divergence and rotation of vector fields, path and surface integrals in $\mathbb R^3$, Gauss' theorem, Stokes' theorem.\\
\textit{Complex analysis:} Introduction to the theory of holomorphic functions, Cauchy's integral theorem, Cauchy's integral formula and applications.
\subsubsection*{\large
    Literature:
}
\begin{itemize}
\item
K.~Königsberger: \emph{Analysis~2}, 5.~Auflage., Springer, 2004.
\item
W.~Walter: \emph{Analysis~2}, 5.~Auflage, Springer, 2002.
\item
E.~Freitag, R.~Busam: \emph{Funktionentheorie~I}, 4.~Auflage, Springer, 2006.
\item
R.~Remmert, G.~Schumacher: \emph{Funktionentheorie~1}. 5.~Auflage, Springer, 2002.
\end{itemize}
\subsubsection*{\large
    Prerequisites:
}
Required: Analysis~I and II, Linear Algebra~I and II
\subsubsection*{\large
    Remarks:
}
This course is only offered in German.
\subsubsection*{\large
    Usability and assessments:
}

\begin{tabularx}{\textwidth}{ p{.5\textwidth}
    |X
}
 &
\makecell[c]{\rotatebox[origin=l]{90}{\parbox{
            4
            cm}{\begin{flushleft}
                Further Chapters in Analysis (MEd18, MEH21, MEdual24) (5.0 ECTS)
            \end{flushleft} }}}
\\
& 1
\\[2ex] \hline
\hline \rule[0mm]{0cm}{.6cm}PL:  \rule[-3mm]{0cm}{0cm}
 &
\makecell[c]{\xmark}
\\
\hline \rule[0mm]{0cm}{.6cm}SL:  \rule[-3mm]{0cm}{0cm}
 &
\makecell[c]{\xmark}
\\
\end{tabularx}


\clearpage\hrule\vskip1pt\hrule
\section*{\Large Basics in Applied Mathematics}
\addcontentsline{toc}{subsection}{Basics in Applied Mathematics\ \textcolor{gray}{(\em Moritz Diehl, Patrick Dondl, Angelika Rohde)}}
\vskip-2ex
{\it Moritz Diehl, Patrick Dondl, Angelika Rohde, Assistant: Ben Deitmar, Coffi Aristide Hounkpe}
\hfill{E}\\
Lecture: Tue, Thu, 8--10 h, HS II, \href{https://www.openstreetmap.org/?mlat=48.002320&mlon=7.847924#map=19/48.002320/7.847924}{Albertstr. 23b}\\
Tutorial: 2 hours, date to be determined \\
Computer exercise: 2 hours, date to be determined \\
\subsubsection*{\large
    Content:
}
Information will follow!
\subsubsection*{\large
    Remarks:
}
Dieser Kurs wird auf Englisch angeboten.
\subsubsection*{\large
    Usability and assessments:
}

\begin{tabularx}{\textwidth}{ p{.5\textwidth}
}
\\
\\[2ex] \hline
\end{tabularx}


\clearpage
\phantomsection
\thispagestyle{empty}
\vspace*{\fill}
\begin{center}
    \Huge\bfseries 1b. Advanced 4-hour Lectures
\end{center}
\addcontentsline{toc}{section}{\textbf{1b. Advanced 4-hour Lectures}}
\addtocontents{toc}{\medskip\hrule\medskip}\vspace*{\fill}\vspace*{\fill}\clearpage
\vfill
\thispagestyle{empty}
\clearpage

\clearpage\hrule\vskip1pt\hrule
\section*{\Large \href{ https://home.mathematik.uni-freiburg.de/soergel/ws2425al.html}{Algebra and Number Theory}}
\addcontentsline{toc}{subsection}{Algebra and Number Theory\ \textcolor{gray}{(\em Wolfgang Soergel)}}
\vskip-2ex
{\it Wolfgang Soergel, Assistant: Damian Sercombe}
\hfill{D}\\
Lecture: Tue, Thu, 10--12 h, HS Weismann-Haus, \href{https://www.openstreetmap.org/?mlat=48.001683&mlon=7.849361#map=19/48.001683/7.849361}{Albertstr. 21a}\\
Tutorial: 2 hours, various dates \\
% Webseite: \url{ https://home.mathematik.uni-freiburg.de/soergel/ws2425al.html}
\subsubsection*{\large
    Content:
}
This lecture continues the linear algebra courses. It treats groups, rings, fields and applications in the number theory and geometry. The highlights of the lecture are the classification of finite fields, the impossibility of the trisection of angles with circle and ruler, the non-existence of a solution formula for the general equations of fifth degree and the quadratic reciprocity law.
\subsubsection*{\large
    Literature:
}
\begin{itemize}
\item Michael Artin: \emph{Algebra}, Birkhäuser 1998.
\item Siegfried Bosch: Algebra (8. Auf\/lage.), Springer Spektrum 2013.
\item Serge Lang: \emph{Algebra} (3. Auf\/lage.), Springer 2002.
\item Wolfgang Soergel: Script \emph{Algebra und Zahlentheorie}
\end{itemize}
\subsubsection*{\large
    Prerequisites:
}
Required: Linear Algebra~I and II
\subsubsection*{\large
    Remarks:
}
This course is only offered in German.
\subsubsection*{\large
    Usability and assessments:
}

\begin{tabularx}{\textwidth}{ p{.5\textwidth}
    |X
    |X
    |X
}
 &
\makecell[c]{\rotatebox[origin=l]{90}{\parbox{
            4
            cm}{\begin{flushleft}
                Algebra and Number Theory (2HfB21, MEH21) (9.0 ECTS) \newline Algebra and Number Theory (MEdual24) (9.0 ECTS) \newline Compulsory elective module in mathematics (BSc21) (9.0 ECTS) \newline Pure Mathematics (MSc14) (11.0 ECTS)
            \end{flushleft} }}}
 &
\makecell[c]{\rotatebox[origin=l]{90}{\parbox{
            4
            cm}{\begin{flushleft}
                Elective (MSc14) (9.0 ECTS) \newline Elective (MScData24) (9.0 ECTS)
            \end{flushleft} }}}
 &
\makecell[c]{\rotatebox[origin=l]{90}{\parbox{
            4
            cm}{\begin{flushleft}
                Introduction to Algebra and Number Theory (MEB21) (5.0 ECTS)
            \end{flushleft} }}}
\\
& 1
& 1
& 1
\\[2ex] \hline
\hline \rule[0mm]{0cm}{.6cm}PL:  \rule[-3mm]{0cm}{0cm}
 &
\makecell[c]{\xmark}
 &
 &
\\
\hline \rule[0mm]{0cm}{.6cm}PL: Oral exam about the first part of the lecture until Christmas (duration: max. 30 minutes) \rule[-3mm]{0cm}{0cm}
 &
 &
 &
\makecell[c]{\xmark}
\\
\hline \rule[0mm]{0cm}{.6cm}SL:  \rule[-3mm]{0cm}{0cm}
 &
\makecell[c]{\xmark}
 &
\makecell[c]{\xmark}
 &
\makecell[c]{\xmark}
\\
\end{tabularx}


\clearpage\hrule\vskip1pt\hrule
\section*{\Large Algebraic Number Theory}
\addcontentsline{toc}{subsection}{Algebraic Number Theory\ \textcolor{gray}{(\em Abhishek Oswal)}}
\vskip-2ex
{\it Abhishek Oswal, Assistant: Andreas Demleitner}
\hfill{E}\\
Lecture: Tue, Thu, 12--14 h, HS II, \href{https://www.openstreetmap.org/?mlat=48.002320&mlon=7.847924#map=19/48.002320/7.847924}{Albertstr. 23b}\\
Tutorial: 2 hours, date to be determined \\
\subsubsection*{\large
    Content:
}
Short description of topics: Number fields, Prime decomposition in Dedekind domains, Ideal class groups, Unit groups, Dirichlet's unit theorem, local fields, valuations, decomposition and inertia groups, introduction to class field theory.  
\subsubsection*{\large
    Literature:
}
Jürgen Neukirch: \emph{Algebraic Number Theory}, Springer, 1999.
\subsubsection*{\large
    Prerequisites:
}
Required: Algebra and Number Theory
\subsubsection*{\large
    Remarks:
}
Dieser Kurs wird auf Englisch angeboten.
\subsubsection*{\large
    Usability and assessments:
}

\begin{tabularx}{\textwidth}{ p{.5\textwidth}
    |X
    |X
}
 &
\makecell[c]{\rotatebox[origin=l]{90}{\parbox{
            4
            cm}{\begin{flushleft}
                Compulsory elective module in mathematics (BSc21) (9.0 ECTS) \newline Mathematical concentration (MEd18, MEH21) (9.0 ECTS) \newline Mathematics (MSc14) (11.0 ECTS) \newline Pure Mathematics (MSc14) (11.0 ECTS) \newline part of the concentration module (MSc14) (10.5 ECTS)
            \end{flushleft} }}}
 &
\makecell[c]{\rotatebox[origin=l]{90}{\parbox{
            4
            cm}{\begin{flushleft}
                Elective (MSc14) (9.0 ECTS) \newline Elective (MScData24) (9.0 ECTS) \newline Elective for individual studying (2HfB21) (9.0 ECTS)
            \end{flushleft} }}}
\\
& 1
& 1
\\[2ex] \hline
\hline \rule[0mm]{0cm}{.6cm}PL:  \rule[-3mm]{0cm}{0cm}
 &
\makecell[c]{\xmark}
 &
\\
\hline \rule[0mm]{0cm}{.6cm}SL:  \rule[-3mm]{0cm}{0cm}
 &
\makecell[c]{\xmark}
 &
\makecell[c]{\xmark}
\\
\end{tabularx}


\clearpage\hrule\vskip1pt\hrule
\section*{\Large Analysis III}
\addcontentsline{toc}{subsection}{Analysis III\ \textcolor{gray}{(\em Patrick Dondl)}}
\vskip-2ex
{\it Patrick Dondl, Assistant: Oliver Suchan}
\hfill{D}\\
Lecture: Mon, 12--14 h, HS Rundbau, \href{https://www.openstreetmap.org/?mlat=48.001613&mlon=7.849254#map=19/48.001613/7.849254}{Albertstr. 21}, Wed, 10--12 h, HS Weismann-Haus, \href{https://www.openstreetmap.org/?mlat=48.001683&mlon=7.849361#map=19/48.001683/7.849361}{Albertstr. 21a}\\
Tutorial: 2 hours, various dates \\
\subsubsection*{\large
    Content:
}
Lebesgue measure and measure theory, Lebesgue integral on measure spaces and Fubini's theorem, Fourier series and Fourier transform, Hilbert spaces. 
Differential forms, their integration and outer derivative. Stokes' theorem and Gauss' theorem.
\subsubsection*{\large
    Prerequisites:
}
Required: Analysis I and II, Linear Algebra I
\subsubsection*{\large
    Remarks:
}
This course is only offered in German.
\subsubsection*{\large
    Usability and assessments:
}

\begin{tabularx}{\textwidth}{ p{.5\textwidth}
    |X
    |X
}
 &
\makecell[c]{\rotatebox[origin=l]{90}{\parbox{
            4
            cm}{\begin{flushleft}
                Analysis III (BSc21) (9.0 ECTS) \newline Elective in Data (MScData24) (9.0 ECTS) \newline Mathematical concentration (MEd18, MEH21) (9.0 ECTS)
            \end{flushleft} }}}
 &
\makecell[c]{\rotatebox[origin=l]{90}{\parbox{
            4
            cm}{\begin{flushleft}
                Elective for individual studying (2HfB21) (9.0 ECTS)
            \end{flushleft} }}}
\\
& 1
& 1
\\[2ex] \hline
\hline \rule[0mm]{0cm}{.6cm}PL:  \rule[-3mm]{0cm}{0cm}
 &
\makecell[c]{\xmark}
 &
\\
\hline \rule[0mm]{0cm}{.6cm}SL:  \rule[-3mm]{0cm}{0cm}
 &
\makecell[c]{\xmark}
 &
\makecell[c]{\xmark}
\\
\end{tabularx}


\clearpage\hrule\vskip1pt\hrule
\section*{\Large \href{https://home.mathematik.uni-freiburg.de/geometrie/lehre/ws2024/DG/}{Differential Geometry}}
\addcontentsline{toc}{subsection}{Differential Geometry\ \textcolor{gray}{(\em Sebastian Goette)}}
\vskip-2ex
{\it Sebastian Goette, Assistant: Mikhael Tëmkin}
\hfill{D}\\
Lecture: Mon, Wed, 14--16 h, HS II, \href{https://www.openstreetmap.org/?mlat=48.002320&mlon=7.847924#map=19/48.002320/7.847924}{Albertstr. 23b}\\
Tutorial: 2 hours, date to be determined \\
% Webseite: \url{https://home.mathematik.uni-freiburg.de/geometrie/lehre/ws2024/DG/}
\subsubsection*{\large
    Content:
}
Differential geometry, especially Riemannian geometry, deals with the geometric properties of curved spaces.
Such spaces also occur in other areas of mathematics and physics, for example in geometric analysis, theoretical mechanics and the general theory of relativity.
\subsubsection*{\large
    Literature:
}
\begin{itemize}
\item{J. Cheeger, D. G. Ebin, {\em Comparison Theorems in Riemannian Geometry,\/} North-Holland, Amsterdam 1975.}
\item{S. Gallot, D. Hulin, J. Lafontaine, {\em Riemannian Geometry,\/} Springer, Berlin-Heidelberg-New York 1987.}
\item{P. Petersen, {\em Riemannian Geometry,\/} Grad. Texts Math.~171, Springer, New York, 2006.}
\end{itemize}
\subsubsection*{\large
    Prerequisites:
}
Required: Analysis~I–III, Lineare Algebra~I and II \\
Recommended: Analysis of Curves and Surfaces ("Kurven und Flächen"), Topology
\subsubsection*{\large
    Remarks:
}
A lecture on differential geometry~II is expected to be offered in the summer semester 2025. \\
This course will only be offered in German.
\subsubsection*{\large
    Usability and assessments:
}

\begin{tabularx}{\textwidth}{ p{.5\textwidth}
    |X
    |X
}
 &
\makecell[c]{\rotatebox[origin=l]{90}{\parbox{
            4
            cm}{\begin{flushleft}
                Compulsory elective module in mathematics (BSc21) (9.0 ECTS) \newline Mathematical concentration (MEd18, MEH21) (9.0 ECTS) \newline Mathematics (MSc14) (11.0 ECTS) \newline Pure Mathematics (MSc14) (11.0 ECTS) \newline part of the concentration module (MSc14) (10.5 ECTS)
            \end{flushleft} }}}
 &
\makecell[c]{\rotatebox[origin=l]{90}{\parbox{
            4
            cm}{\begin{flushleft}
                Elective (MSc14) (9.0 ECTS) \newline Elective (MScData24) (9.0 ECTS) \newline Elective for individual studying (2HfB21) (9.0 ECTS)
            \end{flushleft} }}}
\\
& 1
& 1
\\[2ex] \hline
\hline \rule[0mm]{0cm}{.6cm}PL:  \rule[-3mm]{0cm}{0cm}
 &
\makecell[c]{\xmark}
 &
\\
\hline \rule[0mm]{0cm}{.6cm}SL:  \rule[-3mm]{0cm}{0cm}
 &
\makecell[c]{\xmark}
 &
\makecell[c]{\xmark}
\\
\end{tabularx}


\clearpage\hrule\vskip1pt\hrule
\section*{\Large \href{https://home.mathematik.uni-freiburg.de/analysis/2024_WiSe_Lehre/2024_WiSe_PDE/}{Introduction to Partial Differential Equations}}
\addcontentsline{toc}{subsection}{Introduction to Partial Differential Equations\ \textcolor{gray}{(\em Guofang Wang)}}
\vskip-2ex
{\it Guofang Wang, Assistant: Christine Schmidt}
\hfill{D}\\
Lecture: Mon, Wed, 12--14 h, HS II, \href{https://www.openstreetmap.org/?mlat=48.002320&mlon=7.847924#map=19/48.002320/7.847924}{Albertstr. 23b}\\
Tutorial: 2 hours, date to be determined \\
% Webseite: \url{https://home.mathematik.uni-freiburg.de/analysis/2024_WiSe_Lehre/2024_WiSe_PDE/}
\subsubsection*{\large
    Content:
}
A large number of different problems from the natural sciences and geometry lead to partial differential equations. Consequently, there can be no talk of an all-encompassing theory. Nevertheless, there is a clear picture for linear equations, which is based on three prototypes: the potential equation $-\Delta u = f$, the heat equation $u_t - \Delta u = f$ and the wave equation $u_{tt} - \Delta u = f$, which we will examine in the lecture.
\subsubsection*{\large
    Literature:
}
\begin{itemize}
\item
E. DiBenedetto: \href{https://link.springer.com/book/10.1007/978-0-8176-4552-6}{\emph{Partial differential equations}}, Birkhäuser, 2010. 
\item
L. C. Evans: \href{http://home.ustc.edu.cn/\~xushijie/pdf/textbooks/pde-evans.pdf}{\emph{Partial Differential Equations}} (Second Edition), Graduate Studies in Mathematics 19, AMS, 2010.
\item
Q. Han: \href{https://pdfcoffee.com/a-basic-course-in-partial-differential-equations-qing-han-pdf-free.html}{\emph{A Basic Course in Partial Differential Equations}}, Graduate Studies in Mathematics 120, AMS, 2011. 
\item
J. Jost: \href{http://www.redi-bw.de/start/unifr/EBooks-springer/10.1007/978-1-4614-4809-9}{\emph{Partial Differential Equations}} (Third Edition), Springer, 2013. 
\end{itemize}
\subsubsection*{\large
    Prerequisites:
}
Required: Analysis III \\
Recommended: Complex Analysis ({\em Funktionentheorie})
\subsubsection*{\large
    Remarks:
}
This course is only offered in German.
\subsubsection*{\large
    Usability and assessments:
}

\begin{tabularx}{\textwidth}{ p{.5\textwidth}
    |X
    |X
}
 &
\makecell[c]{\rotatebox[origin=l]{90}{\parbox{
            4
            cm}{\begin{flushleft}
                Compulsory elective module in mathematics (BSc21) (9.0 ECTS) \newline Mathematical concentration (MEd18, MEH21) (9.0 ECTS) \newline Mathematics (MSc14) (11.0 ECTS) \newline Pure Mathematics (MSc14) (11.0 ECTS) \newline part of the concentration module (MSc14) (10.5 ECTS)
            \end{flushleft} }}}
 &
\makecell[c]{\rotatebox[origin=l]{90}{\parbox{
            4
            cm}{\begin{flushleft}
                Elective (MSc14) (9.0 ECTS) \newline Elective (MScData24) (9.0 ECTS) \newline Elective for individual studying (2HfB21) (9.0 ECTS)
            \end{flushleft} }}}
\\
& 1
& 1
\\[2ex] \hline
\hline \rule[0mm]{0cm}{.6cm}PL:  \rule[-3mm]{0cm}{0cm}
 &
\makecell[c]{\xmark}
 &
\\
\hline \rule[0mm]{0cm}{.6cm}SL:  \rule[-3mm]{0cm}{0cm}
 &
\makecell[c]{\xmark}
 &
\makecell[c]{\xmark}
\\
\end{tabularx}


\clearpage\hrule\vskip1pt\hrule
\section*{\Large Complex Analysis}
\addcontentsline{toc}{subsection}{Complex Analysis\ \textcolor{gray}{(\em David Criens)}}
\vskip-2ex
{\it David Criens, Assistant: Eric Trébuchon}
\hfill{D}\\
Lecture: Tue, Wed, 16--18 h, HS II, \href{https://www.openstreetmap.org/?mlat=48.002320&mlon=7.847924#map=19/48.002320/7.847924}{Albertstr. 23b}\\
Tutorial: 2 hours, date to be determined \\
\subsubsection*{\large
    Content:
}
Die Funktionentheorie beschäftigt sich mit Funktionen $f : \mathbb C \to \mathbb C$ , die komplexe Zahlen auf komplexe
Zahlen abbilden. Viele Konzepte der Analysis~I lassen sich direkt auf diesen Fall übertragen, z.\,B. die
Definition der Differenzierbarkeit. Man würde vielleicht erwarten, dass sich dadurch eine zur Analysis~I
analoge Theorie entwickelt, doch viel mehr ist wahr: Man erhält eine in vielerlei Hinsicht elegantere und
einfachere Theorie. Beispielsweise impliziert die komplexe Differenzierbarkeit auf einer offenen Menge, dass
eine Funktion sogar unendlich oft differenzierbar ist, und dies stimmt weiter mit Analytizität überein. Für
reelle Funktionen sind alle diese Begriffe unterschiedlich. Doch auch einige neue Ideen sind notwendig: Für
reelle Zahlen $a$, $b$ integriert man für
$$\int_a^b f(x) \mathrm dx$$
über die Elemente des Intervalls $[a, b]$ bzw. $[b, a]$. Sind $a$, $b$ jedoch komplexe Zahlen, ist nicht mehr so
klar, wie man ein solches Integral auf"|fassen soll. Man könnte z.\,B. in den komplexen Zahlen entlang der
Strecke, die $a, b \in \mathbb C$ verbindet, integrieren, oder aber entlang einer anderen Kurve, die von $a$ nach $b$ führt.
Führt dies zu einem wohldefinierten Integralbegriff oder hängt ein solches Kurvenintegral von der Wahl
der Kurve ab?
\subsubsection*{\large
    Prerequisites:
}
Analysis I+II, Lineare Algebra I
\subsubsection*{\large
    Remarks:
}
This course is only offered in German.
\subsubsection*{\large
    Usability and assessments:
}

\begin{tabularx}{\textwidth}{ p{.5\textwidth}
    |X
    |X
}
 &
\makecell[c]{\rotatebox[origin=l]{90}{\parbox{
            4
            cm}{\begin{flushleft}
                Compulsory elective module in mathematics (BSc21) (9.0 ECTS) \newline Mathematical concentration (MEd18, MEH21) (9.0 ECTS) \newline Pure Mathematics (MSc14) (11.0 ECTS)
            \end{flushleft} }}}
 &
\makecell[c]{\rotatebox[origin=l]{90}{\parbox{
            4
            cm}{\begin{flushleft}
                Elective (MSc14) (9.0 ECTS) \newline Elective (MScData24) (9.0 ECTS) \newline Elective for individual studying (2HfB21) (9.0 ECTS)
            \end{flushleft} }}}
\\
& 1
& 1
\\[2ex] \hline
\hline \rule[0mm]{0cm}{.6cm}PL:  \rule[-3mm]{0cm}{0cm}
 &
\makecell[c]{\xmark}
 &
\\
\hline \rule[0mm]{0cm}{.6cm}SL:  \rule[-3mm]{0cm}{0cm}
 &
\makecell[c]{\xmark}
 &
\makecell[c]{\xmark}
\\
\end{tabularx}


\clearpage\hrule\vskip1pt\hrule
\section*{\Large \href{ https://aam.uni-freiburg.de/agba/lehre/ws24/tun0/index.html}{Introduction to Theory and Numerics of Partial Differential Equations}}
\addcontentsline{toc}{subsection}{Introduction to Theory and Numerics of Partial Differential Equations\ \textcolor{gray}{(\em Sören Bartels)}}
\vskip-2ex
{\it Sören Bartels, Assistant: Vera Jackisch}
\hfill{E}\\
Lecture: Tue, Thu, 10--12 h, SR 226, \href{https://www.openstreetmap.org/?mlat=48.003472&mlon=7.848195#map=19/48.003472/7.848195}{Hermann-Herder-Str. 10}\\
Tutorial: 2 hours, date to be determined \\
% Webseite: \url{ https://aam.uni-freiburg.de/agba/lehre/ws24/tun0/index.html}
\subsubsection*{\large
    Content:
}
The aim of this course is to give an introduction into theory of linear partial differential equations and their finite difference as well as finite element approximations. Finite element methods for approximating partial differential equations have reached a high degree of maturity, and are an indispensable tool in science and technology. We provide an introduction to the construction, analysis, and implementation of finite element methods for different model problems. We will address elementary properties of linear partial differential equations along with their basic numerical approximation, the functional-analytical framework for rigorously establishing existence of solutions, and the construction and analysis of basic finite element methods.
\subsubsection*{\large
    Literature:
}
\begin{itemize}
\item  S. Bartels: Numerical Approximation of Partial Differential Equations, Springer 2016. 
\item  D. Braess: Finite Elemente, Springer 2007. 
\item  S. Brenner, R. Scott: Finite Elements, Springer 2008. 
\item  L. C. Evans: Partial Differential Equations, AMS 2010
\end{itemize}
\subsubsection*{\large
    Prerequisites:
}
Required: Analysis~I and II, Linear Algebra~I and II as well as knowledge about higher-dimensional integration (e.g. from Analysis~III or Extensions of Analysis) \\
Recommended:  Numerics for differential equations, Functional analysis
\subsubsection*{\large
    Remarks:
}
Dieser Kurs wird auf Englisch angeboten.
\subsubsection*{\large
    Usability and assessments:
}

\begin{tabularx}{\textwidth}{ p{.5\textwidth}
    |X
    |X
}
 &
\makecell[c]{\rotatebox[origin=l]{90}{\parbox{
            4
            cm}{\begin{flushleft}
                Advanced Lecture in Numerics (MScData24) (9.0 ECTS) \newline Applied Mathematics (MSc14) (11.0 ECTS) \newline Compulsory elective module in mathematics (BSc21) (9.0 ECTS) \newline Elective in Data (MScData24) (9.0 ECTS) \newline Mathematical concentration (MEd18, MEH21) (9.0 ECTS) \newline Mathematics (MSc14) (11.0 ECTS) \newline part of the concentration module (MSc14) (10.5 ECTS)
            \end{flushleft} }}}
 &
\makecell[c]{\rotatebox[origin=l]{90}{\parbox{
            4
            cm}{\begin{flushleft}
                Elective (MSc14) (9.0 ECTS) \newline Elective for individual studying (2HfB21) (9.0 ECTS)
            \end{flushleft} }}}
\\
& 1
& 1
\\[2ex] \hline
\hline \rule[0mm]{0cm}{.6cm}PL:  \rule[-3mm]{0cm}{0cm}
 &
\makecell[c]{\xmark}
 &
\\
\hline \rule[0mm]{0cm}{.6cm}SL:  \rule[-3mm]{0cm}{0cm}
 &
\makecell[c]{\xmark}
 &
\makecell[c]{\xmark}
\\
\end{tabularx}


\clearpage\hrule\vskip1pt\hrule
\section*{\Large \href{https://www.stochastik.uni-freiburg.de/de/lehre/ws-2024-2025/lecture-mathematical-statistics-ws-2024-2025}{Mathematical Statistics}}
\addcontentsline{toc}{subsection}{Mathematical Statistics\ \textcolor{gray}{(\em Ernst August v. Hammerstein)}}
\vskip-2ex
{\it Ernst August v. Hammerstein, Assistant: Sebastian Stroppel}
\hfill{E}\\
Lecture: Mon, Wed, 14--16 h, SR 404, \href{https://www.openstreetmap.org/?mlat=48.000637&mlon=7.846006#map=19/48.000636/7.846006}{Ernst-Zermelo-Str. 1}\\
Tutorial: 2 hours, date to be determined \\
% Webseite: \url{https://www.stochastik.uni-freiburg.de/de/lehre/ws-2024-2025/lecture-mathematical-statistics-ws-2024-2025}
\subsubsection*{\large
    Content:
}
The lecture builds on basic knowledge about Probability Theory. The fundamental problem of statistics is to infer from a sample of observations as precise as possible statements about the data-generating process or the underlying distributions of the data. For this purpose, the most important methods from statistical decision theory such as test and estimation methods are introduced in the lecture. \\ Key words hereto include Bayes estimators and tests, Neyman-Pearson test theory, maximum likelihood estimators, UMVU estimators, exponential families, linear models. Other topics include ordering principles for reducing the complexity of models (sufficiency and invariance). Statistical methods and procedures are used not only in the natural sciences and medicine, but in almost all areas in which data is collected and analyzed This includes, for example, economics (“econometrics”) and the social sciences (especially psychology). However, in the context of this lecture, we will focus less on applications, but---as the name suggests---more on the mathematical justification of the methods.
\subsubsection*{\large
    Literature:
}
\begin{itemize}
\item C. Czado, T. Schmidt: \href{https://link.springer.com/book/10.1007/978-3-642-17261-8}{\emph{Mathematische Statistik}}, Springer, 2011.
\item E.L. Lehmann, J.P. Romano:\href{https://link.springer.com/book/10.1007/978-3-030-70578-7}{\emph{Testing Statistical Hypotheses (Fourth Edition)}}, Springer, 2022.
\item E.L. Lehmann, G. Casella: \href{https://link.springer.com/book/10.1007/b98854}{\emph{Theory of Point Estimation, Second Edition}}, Springer, 1998.  
\item  L. Rüschendorf: \href{https://link.springer.com/book/10.1007/978-3-642-41997-3}{\emph{Mathematische Statistik}}, Springer Spektrum, 2014. 
\item  M. J. Schervish: \href{https://link.springer.com/book/10.1007/978-1-4612-4250-5}{\emph{Theory of Statistics}}, Springer, 1995.
\item J. Shao:  \href{https://link.springer.com/book/10.1007/b97553}{\emph{Mathematical Statistics}}, Springer, 2003. 
\item H. Witting: \emph{Mathematische Statistik I}, Teubner, 1985.
\end{itemize}
\subsubsection*{\large
    Prerequisites:
}
Probability Theory (in particular measure theory and conditional probabilities/expectations)
\subsubsection*{\large
    Remarks:
}
Dieser Kurs wird auf Englisch angeboten.
\subsubsection*{\large
    Usability and assessments:
}

\begin{tabularx}{\textwidth}{ p{.5\textwidth}
    |X
    |X
}
 &
\makecell[c]{\rotatebox[origin=l]{90}{\parbox{
            4
            cm}{\begin{flushleft}
                Advanced Lecture in Stochastics (MScData24) (11.0 ECTS) \newline Applied Mathematics (MSc14) (11.0 ECTS) \newline Compulsory elective module in mathematics (BSc21) (9.0 ECTS) \newline Elective in Data (MScData24) (11.0 ECTS) \newline Mathematical concentration (MEd18, MEH21) (9.0 ECTS) \newline Mathematics (MSc14) (11.0 ECTS) \newline part of the concentration module (MSc14) (10.5 ECTS)
            \end{flushleft} }}}
 &
\makecell[c]{\rotatebox[origin=l]{90}{\parbox{
            4
            cm}{\begin{flushleft}
                Elective (MSc14) (9.0 ECTS) \newline Elective for individual studying (2HfB21) (9.0 ECTS)
            \end{flushleft} }}}
\\
& 1
& 1
\\[2ex] \hline
\hline \rule[0mm]{0cm}{.6cm}PL:  \rule[-3mm]{0cm}{0cm}
 &
\makecell[c]{\xmark}
 &
\\
\hline \rule[0mm]{0cm}{.6cm}SL:  \rule[-3mm]{0cm}{0cm}
 &
\makecell[c]{\xmark}
 &
\makecell[c]{\xmark}
\\
\end{tabularx}


\clearpage\hrule\vskip1pt\hrule
\section*{\Large \href{https://pfaffelh.github.io/hp/2024WS_stochastic_processes.html}{Stochastic Processes (Probability Theory II)}}
\addcontentsline{toc}{subsection}{Stochastic Processes (Probability Theory II)\ \textcolor{gray}{(\em Peter Pfaffelhuber)}}
\vskip-2ex
{\it Peter Pfaffelhuber, Assistant: Samuel Adeosun}
\hfill{E}\\
Lecture: Mon, 10--12 h, HS II, \href{https://www.openstreetmap.org/?mlat=48.002320&mlon=7.847924#map=19/48.002320/7.847924}{Albertstr. 23b}\\
 Wed, 12--14 h, SR 127, \href{https://www.openstreetmap.org/?mlat=48.000637&mlon=7.846006#map=19/48.000636/7.846006}{Ernst-Zermelo-Str. 1}\\
Letcure (4 hours): asynchronous videos \\
% Webseite: \url{https://pfaffelh.github.io/hp/2024WS_stochastic_processes.html}
\subsubsection*{\large
    Content:
}
A stochastic process $(X_t)_{t\in I}$ is nothing more than a family of random variables, where $I$ is some index set modeling time. Simple examples are random walks, Markov chains, Brownian motion and derived processes. The latter play a particularly important role in the modeling of financial mathematics or questions from the sciences. We will first deal with martingales, which describe fair games. After constructing the Poisson process and Brownian motion, we will focus on properties of Brownian motion. Infinitesimal characteristics of a Markov process are described by generators, which allows a connection to the theory of partial differential equations. Finally, a generalization of the law of large numbers is discussed with the ergodic theorem for stationary stochastic processes. Furthermore, insights are given into a few areas of application, such as biomathematics or random graphs.
\subsubsection*{\large
    Literature:
}
\begin{itemize}
\item
 O. Kallenberg: \href{https://link.springer.com/book/10.1007/978-3-030-61871-1}{\emph{Foundations of Modern Probability}} (Third Edition), Springer, 2021.
\item
 A. Klenke: \href{https://link.springer.com/book/10.1007/978-3-662-62089-2}{\emph{Wahrscheinlichkeitstheorie}} (4. Auf\/lage), Springer, 2020. 
\item 
D. Williams: \href{https://edisciplinas.usp.br/pluginfile.php/343758/mod_folder/content/0/Probability With Martingales(Williams).pdf}{\emph{Probability with Martingales}}, Cambridge University Press, 1991. 
\end{itemize}
\subsubsection*{\large
    Prerequisites:
}
Probability Theory I
\subsubsection*{\large
    Remarks:
}
The lecture series started with the lecture {\em Probability Theory I} in summer 2024, and will continue In the summer semester 2025 with the lecture {\em Probability Theory~III (Stochastic Analysis)}.
\subsubsection*{\large
    Usability and assessments:
}

\begin{tabularx}{\textwidth}{ p{.5\textwidth}
    |X
    |X
}
 &
\makecell[c]{\rotatebox[origin=l]{90}{\parbox{
            4
            cm}{\begin{flushleft}
                Advanced Lecture in Stochastics (MScData24) (11.0 ECTS) \newline Applied Mathematics (MSc14) (11.0 ECTS) \newline Compulsory elective module in mathematics (BSc21) (9.0 ECTS) \newline Elective in Data (MScData24) (11.0 ECTS) \newline Mathematical concentration (MEd18, MEH21) (9.0 ECTS) \newline Mathematics (MSc14) (11.0 ECTS) \newline part of the concentration module (MSc14) (10.5 ECTS)
            \end{flushleft} }}}
 &
\makecell[c]{\rotatebox[origin=l]{90}{\parbox{
            4
            cm}{\begin{flushleft}
                Elective (MSc14) (9.0 ECTS) \newline Elective for individual studying (2HfB21) (9.0 ECTS)
            \end{flushleft} }}}
\\
& 1
& 1
\\[2ex] \hline
\hline \rule[0mm]{0cm}{.6cm}PL:  \rule[-3mm]{0cm}{0cm}
 &
\makecell[c]{\xmark}
 &
\\
\hline \rule[0mm]{0cm}{.6cm}SL:  \rule[-3mm]{0cm}{0cm}
 &
\makecell[c]{\xmark}
 &
\makecell[c]{\xmark}
\\
\end{tabularx}


\clearpage\hrule\vskip1pt\hrule
\section*{\Large Stochastic Integration and Financial Mathematics (Probability Theory III)}
\addcontentsline{toc}{subsection}{Stochastic Integration and Financial Mathematics (Probability Theory III)\ \textcolor{gray}{(\em Thorsten Schmidt)}}
\vskip-2ex
{\it Thorsten Schmidt, Assistant: Moritz Ritter}
\hfill{E}\\
Lecture: Mon, Wed, 12--14 h, SR 404, \href{https://www.openstreetmap.org/?mlat=48.000637&mlon=7.846006#map=19/48.000636/7.846006}{Ernst-Zermelo-Str. 1}\\
Tutorial: 2 hours, date to be determined \\
\subsubsection*{\large
    Content:
}
This lecture marks the culmination of our series on probability theory, achieving the ultimate goal of this series: 
the combination of stochastic analysis and financial mathematics---a field that has yielded an amazing wealth of fascinating results since the 1990s. The core is certainly the application of semimartingale theory to financial markets culminating in the fundamental theorem of asset pricing. This results is used everywhere
in financial markets for arbitrage-free pricing. 

After this we look into modern forms of stochastic analysis covering neural SDEs, signature methods, uncertainty and term structure models. The lecture will conclude with an examination of the latest applications of machine learning in financial markets and the reciprocal influence of stochastic analysis on machine learning.

\subsubsection*{\large
    Literature:
}
Relevant literature will be announced during the course.
\subsubsection*{\large
    Prerequisites:
}
Required: Probability Theory II (Stochastic Processes)
\subsubsection*{\large
    Remarks:
}
Diese Vorlesung wird auf Englisch angeboten
\subsubsection*{\large
    Usability and assessments:
}

\begin{tabularx}{\textwidth}{ p{.5\textwidth}
    |X
    |X
}
 &
\makecell[c]{\rotatebox[origin=l]{90}{\parbox{
            4
            cm}{\begin{flushleft}
                Advanced Lecture in Stochastics (MScData24) (11.0 ECTS) \newline Applied Mathematics (MSc14) (11.0 ECTS) \newline Compulsory elective module in mathematics (BSc21) (9.0 ECTS) \newline Elective in Data (MScData24) (11.0 ECTS) \newline Mathematical concentration (MEd18, MEH21) (9.0 ECTS) \newline Mathematics (MSc14) (11.0 ECTS) \newline part of the concentration module (MSc14) (10.5 ECTS)
            \end{flushleft} }}}
 &
\makecell[c]{\rotatebox[origin=l]{90}{\parbox{
            4
            cm}{\begin{flushleft}
                Elective (MSc14) (9.0 ECTS) \newline Elective for individual studying (2HfB21) (9.0 ECTS)
            \end{flushleft} }}}
\\
& 1
& 1
\\[2ex] \hline
\hline \rule[0mm]{0cm}{.6cm}PL:  \rule[-3mm]{0cm}{0cm}
 &
\makecell[c]{\xmark}
 &
\\
\hline \rule[0mm]{0cm}{.6cm}SL:  \rule[-3mm]{0cm}{0cm}
 &
\makecell[c]{\xmark}
 &
\makecell[c]{\xmark}
\\
\end{tabularx}


\clearpage\hrule\vskip1pt\hrule
\section*{\Large \href{https://home.mathematik.uni-freiburg.de/arithgeom/lehre/ws24/semialg.html}{Semi-Algebraic Geometry}}
\addcontentsline{toc}{subsection}{Semi-Algebraic Geometry\ \textcolor{gray}{(\em Annette Huber-Klawitter, Amador Martín Pizarro)}}
\vskip-2ex
{\it Annette Huber-Klawitter, Amador Martín Pizarro, Assistant: Christoph Brackenhofer}
\hfill{D}\\
Lecture: Tue, Thu, 10--12 h, HS II, \href{https://www.openstreetmap.org/?mlat=48.002320&mlon=7.847924#map=19/48.002320/7.847924}{Albertstr. 23b}\\
Tutorial: 2 hours, date to be determined \\
% Webseite: \url{https://home.mathematik.uni-freiburg.de/arithgeom/lehre/ws24/semialg.html}
\subsubsection*{\large
    Content:
}
Semi-algebraic geometry is about properties of subsets of $\textbf{R}^n$, which are given by inequalities of the form
\[ f(x_1, \dots, x_n)\geq 0\]
for polynomials $f\in\textbf{R}[X_1,\dots,X_n]$.

The theory has many different facets. On the one hand, it can be seen as a version of algebraic geometry over $\mathbf{R}$ (or even more generally over so-called real closed fields). On the other hand, the properties of these fields are a central tool for the model-theoretic proof of Tarski-Seidenberg's theorem on quantifier elimination in real closed fields. Geometrically, this is interpreted as a projection theorem.

From this theorem, a proof of Hilbert's 17th problem easily follows, which was solved by Artin in 1926.

\textit{Is every real polynomial $P \in \mathbf{R}[x_1, \dots, x_n]$, which takes a non-negative value for every $n$-tuple in $\mathbf{R}^n$, a sum of squares of rational functions (i.e., quotients of polynomials)?}

In the lecture, we will explore both aspects. Necessary tools from commutative algebra or model theory will be discussed according to the prior knowledge of the audience.
\subsubsection*{\large
    Literature:
}
\begin{itemize}
\item A.~Prestel: Lecture Notes \href{http://www.math.uni-konstanz.de/\~prestel/raskript.pdf}{\emph{Reelle Algebra}}.
\item
L.~van den Dries: \emph{Tame topology and o-minimal structures}, London Mathematical Society Lecture Note Series, Cambridge University Press, 1998. 
\item
Jacek Bochnak, Michel Coste \& Marie-Françoise Roy: \emph{Real Algebra}, Ergebnisse der Mathematik und ihrer Grenzgebiete 36, Springer Verlag, 1998.
\end{itemize}
\subsubsection*{\large
    Prerequisites:
}
Required: Algebra and Number Theory \\ Recommended: Knowledge in commutative algebra and algebraic geometry (cf. Kommutative Algebra und Einführung in die algebraische Geometrie), model theory
\subsubsection*{\large
    Remarks:
}
This course is only offered in German.
\subsubsection*{\large
    Usability and assessments:
}

\begin{tabularx}{\textwidth}{ p{.5\textwidth}
    |X
    |X
}
 &
\makecell[c]{\rotatebox[origin=l]{90}{\parbox{
            4
            cm}{\begin{flushleft}
                Compulsory elective module in mathematics (BSc21) (9.0 ECTS) \newline Mathematical concentration (MEd18, MEH21) (9.0 ECTS) \newline Mathematics (MSc14) (11.0 ECTS) \newline Pure Mathematics (MSc14) (11.0 ECTS) \newline part of the concentration module (MSc14) (10.5 ECTS)
            \end{flushleft} }}}
 &
\makecell[c]{\rotatebox[origin=l]{90}{\parbox{
            4
            cm}{\begin{flushleft}
                Elective (MSc14) (9.0 ECTS) \newline Elective (MScData24) (9.0 ECTS) \newline Elective for individual studying (2HfB21) (9.0 ECTS)
            \end{flushleft} }}}
\\
& 1
& 1
\\[2ex] \hline
\hline \rule[0mm]{0cm}{.6cm}PL:  \rule[-3mm]{0cm}{0cm}
 &
\makecell[c]{\xmark}
 &
\\
\hline \rule[0mm]{0cm}{.6cm}SL:  \rule[-3mm]{0cm}{0cm}
 &
\makecell[c]{\xmark}
 &
\makecell[c]{\xmark}
\\
\end{tabularx}


\clearpage\hrule\vskip1pt\hrule
\section*{\Large Set Theory – Independence Proofs}
\addcontentsline{toc}{subsection}{Set Theory – Independence Proofs\ \textcolor{gray}{(\em Maxwell Levine)}}
\vskip-2ex
{\it Maxwell Levine, Assistant: Hannes Jakob}
\hfill{E}\\
Lecture: Tue, Thu, 12--14 h, SR 404, \href{https://www.openstreetmap.org/?mlat=48.000637&mlon=7.846006#map=19/48.000636/7.846006}{Ernst-Zermelo-Str. 1}\\
Tutorial: 2 hours, date to be determined \\
\subsubsection*{\large
    Content:
}
How does one prove that something cannot be proved? More precisely, how does one prove that a particular statement does not follow from a particular collection of axioms?

These questions are often asked with respect to the axioms most commonly used by mathematicians: the axioms of Zermelo-Fraenkel set theory, or ZFC for short. In this course, we will develop the conceptual tools needed to understand independence proofs with respect to ZFC. On the way we will develop the theory of ordinal and cardinal numbers, the basics of inner model theory, and the method of forcing. In particular, we will show that Cantor's continuum hypothesis, the statement that $2^{\aleph_0}=\aleph_1$, is independent of ZFC. 

\subsubsection*{\large
    Literature:
}
\begin{itemize}
\item Thomas Jech: \emph{Set Theory}. The Third Millenium Edition, Springer, 2001. 
\item Kenneth Kunen: \emph{Set Theory: An Introduction to Independence Proofs}. North-Holland Pub. Co, 1980.
\end{itemize}

\subsubsection*{\large
    Prerequisites:
}
Mathematical Logic
\subsubsection*{\large
    Remarks:
}
Dieser Kurs wird auf Englisch angeboten.
\subsubsection*{\large
    Usability and assessments:
}

\begin{tabularx}{\textwidth}{ p{.5\textwidth}
    |X
    |X
}
 &
\makecell[c]{\rotatebox[origin=l]{90}{\parbox{
            4
            cm}{\begin{flushleft}
                Compulsory elective module in mathematics (BSc21) (9.0 ECTS) \newline Mathematical concentration (MEd18, MEH21) (9.0 ECTS) \newline Mathematics (MSc14) (11.0 ECTS) \newline Pure Mathematics (MSc14) (11.0 ECTS) \newline part of the concentration module (MSc14) (10.5 ECTS)
            \end{flushleft} }}}
 &
\makecell[c]{\rotatebox[origin=l]{90}{\parbox{
            4
            cm}{\begin{flushleft}
                Elective (MSc14) (9.0 ECTS) \newline Elective (MScData24) (9.0 ECTS) \newline Elective for individual studying (2HfB21) (9.0 ECTS)
            \end{flushleft} }}}
\\
& 1
& 1
\\[2ex] \hline
\hline \rule[0mm]{0cm}{.6cm}PL:  \rule[-3mm]{0cm}{0cm}
 &
\makecell[c]{\xmark}
 &
\\
\hline \rule[0mm]{0cm}{.6cm}SL:  \rule[-3mm]{0cm}{0cm}
 &
\makecell[c]{\xmark}
 &
\makecell[c]{\xmark}
\\
\end{tabularx}


\clearpage\hrule\vskip1pt\hrule
\section*{\Large Theory and Numerics for Partial Differential Equations – Nonlinear Problems}
\addcontentsline{toc}{subsection}{Theory and Numerics for Partial Differential Equations – Nonlinear Problems\ \textcolor{gray}{(\em Sören Bartels, Patrick Dondl)}}
\vskip-2ex
{\it Sören Bartels, Patrick Dondl}
\hfill{E}\\
Lecture (four hours) \\
Tutorial: 2 hours, date to be determined \\
\subsubsection*{\large
    Content:
}
The lecture addresses the development and analysis of numerical methods for the approximation of certain nonlinear partial differential equations. The considered model problems include harmonic maps into spheres, total-variation regularized minimization problems, and nonlinear bending models. For each of the problems, a suitable finite element discretization is devised, its convergence is analyzed and iterative solution procedures are developed. The lecture is complemented by theoretical and practical lab tutorials in which the results are deepened and experimentally tested. 
\subsubsection*{\large
    Literature:
}
\begin{itemize}
\item
S. Bartels: Numerical methods for nonlinear partial differential equations, Springer, 2015.
\item
M. Dobrowolski: Angewandte Funktionalanalysis, Springer, 2010.
\item
L.C. Evans: Partial Differential Equations, 2nd Edition, 2010. 
\end{itemize}
\subsubsection*{\large
    Prerequisites:
}
Introduction to Theory and Numerics for PDEs or Introduction to PDEs
\subsubsection*{\large
    Remarks:
}
This lecture is offered as a reading course.
\subsubsection*{\large
    Usability and assessments:
}

\begin{tabularx}{\textwidth}{ p{.5\textwidth}
    |X
    |X
}
 &
\makecell[c]{\rotatebox[origin=l]{90}{\parbox{
            4
            cm}{\begin{flushleft}
                Advanced Lecture in Numerics (MScData24) (11.0 ECTS) \newline Applied Mathematics (MSc14) (11.0 ECTS) \newline Compulsory elective module in mathematics (BSc21) (9.0 ECTS) \newline Elective in Data (MScData24) (11.0 ECTS) \newline Mathematical concentration (MEd18, MEH21) (9.0 ECTS) \newline Mathematics (MSc14) (11.0 ECTS) \newline part of the concentration module (MSc14) (10.5 ECTS)
            \end{flushleft} }}}
 &
\makecell[c]{\rotatebox[origin=l]{90}{\parbox{
            4
            cm}{\begin{flushleft}
                Elective (MSc14) (9.0 ECTS) \newline Elective for individual studying (2HfB21) (9.0 ECTS)
            \end{flushleft} }}}
\\
& 1
& 1
\\[2ex] \hline
\hline \rule[0mm]{0cm}{.6cm}PL:  \rule[-3mm]{0cm}{0cm}
 &
\makecell[c]{\xmark}
 &
\\
\hline \rule[0mm]{0cm}{.6cm}SL:  \rule[-3mm]{0cm}{0cm}
 &
\makecell[c]{\xmark}
 &
\makecell[c]{\xmark}
\\
\end{tabularx}


\clearpage\hrule\vskip1pt\hrule
\section*{\Large Reading courses}
\addcontentsline{toc}{subsection}{Reading courses\ \textcolor{gray}{(\em Alle Dozent:inn:en der Mathematik)}}
\vskip-2ex
{\it Alle Dozent:inn:en der Mathematik}
\hfill{D/E}\\
 \\
\subsubsection*{\large
    Content:
}
In a reading course, the material of a four-hour lecture is studied in supervised self-study. In rare cases, this may take place as part of a course; however, reading courses are not usually listed in the course catalog. If you are interested, please contact a professor or a private lecturer before the start of the course; typically, this will be the supervisor of your Master's thesis, as the reading course ideally serves as preparation for the Master's thesis (both in the M.Sc. and the M.Ed. programs).

The content of the reading course, the specific details, and the coursework requirements will be determined by the supervisor at the beginning of the lecture period. The workload should be equivalent to that of a four-hour lecture with exercises.
\subsubsection*{\large
    Usability and assessments:
}

\begin{tabularx}{\textwidth}{ p{.5\textwidth}
    |X
    |X
}
 &
\makecell[c]{\rotatebox[origin=l]{90}{\parbox{
            4
            cm}{\begin{flushleft}
                Elective (MSc14) (9.0 ECTS)
            \end{flushleft} }}}
 &
\makecell[c]{\rotatebox[origin=l]{90}{\parbox{
            4
            cm}{\begin{flushleft}
                Mathematics (MSc14) (11.0 ECTS) \newline Reading Course (MEd18, MEH21) (9.0 ECTS) \newline part of the concentration module (MSc14) (10.5 ECTS)
            \end{flushleft} }}}
\\
& 1
& 1
\\[2ex] \hline
\hline \rule[0mm]{0cm}{.6cm}PL:  \rule[-3mm]{0cm}{0cm}
 &
 &
\makecell[c]{\xmark}
\\
\hline \rule[0mm]{0cm}{.6cm}SL:  \rule[-3mm]{0cm}{0cm}
 &
\makecell[c]{\xmark}
 &
\makecell[c]{\xmark}
\\
\end{tabularx}


\clearpage
\phantomsection
\thispagestyle{empty}
\vspace*{\fill}
\begin{center}
    \Huge\bfseries 1c. Advanced 2-hour Lectures
\end{center}
\addcontentsline{toc}{section}{\textbf{1c. Advanced 2-hour Lectures}}
\addtocontents{toc}{\medskip\hrule\medskip}\vspace*{\fill}\vspace*{\fill}\clearpage
\vfill
\thispagestyle{empty}
\clearpage

\clearpage\hrule\vskip1pt\hrule
\section*{\Large \href{ https://sites.google.com/view/xuwenzhang/teaching/functions-of-bounded-variation-sets-of-finite-perimeter}{Functions of Bounded Variation and Sets of Finite Perimeter}}
\addcontentsline{toc}{subsection}{Functions of Bounded Variation and Sets of Finite Perimeter\ \textcolor{gray}{(\em Xuwen Zhang)}}
\vskip-2ex
{\it Xuwen Zhang}
\hfill{E}\\
Lecture: Mon, 14--16 h, SR 127, \href{https://www.openstreetmap.org/?mlat=48.000637&mlon=7.846006#map=19/48.000636/7.846006}{Ernst-Zermelo-Str. 1}\\
Tutorial: 2 hours, date to be determined \\
% Webseite: \url{ https://sites.google.com/view/xuwenzhang/teaching/functions-of-bounded-variation-sets-of-finite-perimeter}
\subsubsection*{\large
    Content:
}
We will study functions of bounded variation, which are functions whose weak first
partial derivatives are Radon measures. This is essentially the weakest definition of a function to be
differentiable in the measure-theoretic sense. After discussing the basic properties of them, we move on
to the study of sets of finite perimeter, which are Lebesgue measurable sets in the Euclidean space whose
indicator functions are BV functions. Sets of finite perimeter are fundamental in the modern Calculus of
Variations as they generalize in a natural measure-theoretic way the notion of sets with regular boundaries
and possess nice compactness, thus appearing in many Geometric Variational problems. If time permits,
we will discuss the (capillary) sessile drop problem as one important application.
\subsubsection*{\large
    Literature:
}
• Evans, Lawrence C. and Gariepy, Ronald F. Measure theory and fine properties of functions.
CRC Press, Boca Raton, FL, 2015.
• Maggi, Francesco. Sets of finite perimeter and geometric variational problems: an introduction
to Geometric Measure Theory. No. 135. Cambridge University Press, 2012.
\subsubsection*{\large
    Prerequisites:
}
Basic knowledge in measure theory and analysis is required.
\subsubsection*{\large
    Usability and assessments:
}

\begin{tabularx}{\textwidth}{ p{.5\textwidth}
    |X
    |X
}
 &
\makecell[c]{\rotatebox[origin=l]{90}{\parbox{
            4
            cm}{\begin{flushleft}
                Compulsory elective module in mathematics (BSc21) (6.0 ECTS) \newline part of the concentration module (MSc14) (5.25 ECTS) \newline part of the module ''Mathematics'' (MSc14) (5.5 ECTS) \newline part of the module ''Pure Mathematics'' (MSc14) (5.5 ECTS)
            \end{flushleft} }}}
 &
\makecell[c]{\rotatebox[origin=l]{90}{\parbox{
            4
            cm}{\begin{flushleft}
                Elective (MSc14) (6.0 ECTS) \newline Elective (MScData24) (6.0 ECTS) \newline Elective for individual studying (2HfB21) (6.0 ECTS)
            \end{flushleft} }}}
\\
& 1
& 1
\\[2ex] \hline
\hline \rule[0mm]{0cm}{.6cm}PL:  \rule[-3mm]{0cm}{0cm}
 &
\makecell[c]{\xmark}
 &
\\
\hline \rule[0mm]{0cm}{.6cm}SL:  \rule[-3mm]{0cm}{0cm}
 &
\makecell[c]{\xmark}
 &
\makecell[c]{\xmark}
\\
\end{tabularx}


\clearpage\hrule\vskip1pt\hrule
\section*{\Large \href{https://www.finance.uni-freiburg.de/studium-und-lehre/ws2425/fao2425}{Futures and Options}}
\addcontentsline{toc}{subsection}{Futures and Options\ \textcolor{gray}{(\em Eva Lütkebohmert-Holtz)}}
\vskip-2ex
{\it Eva Lütkebohmert-Holtz, Assistant: Hongyi Shen}
\hfill{E}\\
Lecture: Mon, 10--12 h, HS 1098, \href{https://www.openstreetmap.org/?mlat=47.993706&mlon=7.846060#map=19/47.993706/7.846060}{KG I}\\
Tutorial: Thu, 10--12 h, HS 1098, \href{https://www.openstreetmap.org/?mlat=47.993706&mlon=7.846060#map=19/47.993706/7.846060}{KG I}\\
% Webseite: \url{https://www.finance.uni-freiburg.de/studium-und-lehre/ws2425/fao2425}
\subsubsection*{\large
    Content:
}
This course covers an introduction to financial markets and products. Besides futures and standard put and call options
of European and American type we also discuss interest-rate sensitive instruments such as swaps.

For the valuation of financial derivatives we first introduce financial models in discrete time as the Cox--Ross--Rubinstein
model and explain basic principles of risk-neutral valuation. Finally, we will discuss the famous Black--Scholes model
which represents a continuous time model for option pricing.
\subsubsection*{\large
    Literature:
}
\begin{itemize}
\item
D. M. Chance, R. Brooks: \emph{An Introduction to Derivatives and Risk Management} (10th edition), Cengage, 2016. 
\item
J. C. Hull: \emph{Options, Futures, and other Derivatives} (11th global edition), Pearson, 2021.
\item 
S. E. Shreve: \href{https://link.springer.com/book/10.1007/978-0-387-22527-2}{\emph{Stochastic Calculus for Finance I: The Binomial Asset Pricing Model}}, Springer, 2004. 
\item 
R. A. Strong: \emph{Derivatives. An Introduction} (Second edition), South-Western, 2004.
\end{itemize}
\subsubsection*{\large
    Prerequisites:
}
Elementary Probability Theory~I
\subsubsection*{\large
    Remarks:
}
The course  is offered for the first year in the Finance profile of the M.Sc. Economics programme as well as for students of M.Sc. and B.Sc. Mathematics, M.Sc. Mathematics in Data and Technology and M.Sc. Volkswirtschaftslehre. In the M.Sc. Mathematics, it can also count as elective in economics for the specialization in financial mathematics. For students who are currently in the B.Sc. Mathematics programme, but plan to continue with this special profile, it is therefore recommended to credit this course for the latter profile and not for B.Sc. Mathematics.
\subsubsection*{\large
    Usability and assessments:
}

\begin{tabularx}{\textwidth}{ p{.5\textwidth}
    |X
    |X
}
 &
\makecell[c]{\rotatebox[origin=l]{90}{\parbox{
            4
            cm}{\begin{flushleft}
                Additional module in mathematics (MEd18) (6.0 ECTS) \newline Elective (MSc14) (6.0 ECTS) \newline Elective for individual studying (2HfB21) (6.0 ECTS)
            \end{flushleft} }}}
 &
\makecell[c]{\rotatebox[origin=l]{90}{\parbox{
            4
            cm}{\begin{flushleft}
                Compulsory elective module in mathematics (BSc21) (6.0 ECTS) \newline Elective in Data (MScData24) (6.0 ECTS) \newline part of the concentration module (MSc14) (5.25 ECTS) \newline part of the module ''Applied Mathematics'' (MSc14) (5.5 ECTS) \newline part of the module ''Mathematics'' (MSc14) (5.5 ECTS)
            \end{flushleft} }}}
\\
& 1
& 1
\\[2ex] \hline
\hline \rule[0mm]{0cm}{.6cm}PL:  \rule[-3mm]{0cm}{0cm}
 &
 &
\makecell[c]{\xmark}
\\
\hline \rule[0mm]{0cm}{.6cm}SL:  \rule[-3mm]{0cm}{0cm}
 &
\makecell[c]{\xmark}
 &
\\
\end{tabularx}


\clearpage\hrule\vskip1pt\hrule
\section*{\Large \href{https://sites.google.com/view/maximilianstegemeyer/teaching/lie-gruppen-und-symmetrische-r\%C3\%A4ume-ws-2425}{Lie Groups and Symmetric Spaces}}
\addcontentsline{toc}{subsection}{Lie Groups and Symmetric Spaces\ \textcolor{gray}{(\em Maximilian Stegemeyer)}}
\vskip-2ex
{\it Maximilian Stegemeyer}
\hfill{D}\\
Lecture: Thu, 14--16 h, SR 404, \href{https://www.openstreetmap.org/?mlat=48.000637&mlon=7.846006#map=19/48.000636/7.846006}{Ernst-Zermelo-Str. 1}\\
Tutorial: 2 hours, date to be determined \\
% Webseite: \url{https://sites.google.com/view/maximilianstegemeyer/teaching/lie-gruppen-und-symmetrische-r\%C3\%A4ume-ws-2425}
\subsubsection*{\large
    Content:
}
Lie groups and operations of Lie groups play a central role in geometry and topology. They can be used to describe continuous symmetries, one of the most important concepts of mathematics and physics. Exploiting symmetries, e.g. when describing homogeneous spaces, makes it easier to solve many specific problems and often provides a deeper insight into the structures examined. In addition, the geometry and topology of Lie groups and homogeneous spaces is of great interest.

 In this lecture, we start with introducing the basic theory of Lie groups and Lie algebras, especially with insights into the structure theory of Lie algebras. In the second part we will look at homogeneous spaces with a special focus on Riemannian symmetric spaces. The latter form an important class of examples of Riemannian manifolds. In addition to the Lie-theoretical aspects, a special focus will always be on the homogeneous Riemannian metrics of the respective spaces.
\subsubsection*{\large
    Literature:
}
\begin{itemize}
\item
S.~Helgason. \emph{Differential geometry and symmetric spaces}. American Mathematical Soc., 2001.
\item
J.M.~Lee: \emph{Smooth manifolds}. Springer New York, 2012.
\item
B.~O'Neill: \emph{Semi-Riemannian geometry with applications to relativity}. Academic press, 1983.
\item
W.~Ziller: \emph{Lie Groups. Representation Theory and Symmetric Spaces}. Lecture Notes, 2010.
\end{itemize}

\subsubsection*{\large
    Prerequisites:
}
Differential geometry~I
\subsubsection*{\large
    Remarks:
}
This course is only offered in German.
\subsubsection*{\large
    Usability and assessments:
}

\begin{tabularx}{\textwidth}{ p{.5\textwidth}
    |X
    |X
}
 &
\makecell[c]{\rotatebox[origin=l]{90}{\parbox{
            4
            cm}{\begin{flushleft}
                Compulsory elective module in mathematics (BSc21) (6.0 ECTS) \newline part of the concentration module (MSc14) (5.25 ECTS) \newline part of the module ''Mathematics'' (MSc14) (5.5 ECTS) \newline part of the module ''Pure Mathematics'' (MSc14) (5.5 ECTS)
            \end{flushleft} }}}
 &
\makecell[c]{\rotatebox[origin=l]{90}{\parbox{
            4
            cm}{\begin{flushleft}
                Elective (MSc14) (6.0 ECTS) \newline Elective (MScData24) (6.0 ECTS) \newline Elective for individual studying (2HfB21) (6.0 ECTS)
            \end{flushleft} }}}
\\
& 1
& 1
\\[2ex] \hline
\hline \rule[0mm]{0cm}{.6cm}PL:  \rule[-3mm]{0cm}{0cm}
 &
\makecell[c]{\xmark}
 &
\\
\hline \rule[0mm]{0cm}{.6cm}SL:  \rule[-3mm]{0cm}{0cm}
 &
\makecell[c]{\xmark}
 &
\makecell[c]{\xmark}
\\
\end{tabularx}


\clearpage\hrule\vskip1pt\hrule
\section*{\Large Markov Chains}
\addcontentsline{toc}{subsection}{Markov Chains\ \textcolor{gray}{(\em David Criens)}}
\vskip-2ex
{\it David Criens, Assistant: Dario Kieffer}
\hfill{E}\\
Lecture: Thu, 12--14 h, SR 226, \href{https://www.openstreetmap.org/?mlat=48.003472&mlon=7.848195#map=19/48.003472/7.848195}{Hermann-Herder-Str. 10}\\
Tutorial: 2 hours, date to be determined \\
\subsubsection*{\large
    Content:
}
The class of Markov chains is an important class of (discrete-time) stochastic processes that are used frequently to model for example the spread of infections, queuing systems or switches of economic scenarios. Their main characteristic is the Markov property, which roughly means that the future depends on the past only through the current state. In this lecture we provide the mathematical foundation of the theory of Markov chains. In particular, we learn about path properties, such as recurrence and transience, state classifications and discuss convergence to the equilibrium. We also study extensions to continuous time. On the way we discuss applications to biology, queuing systems and resource management. If the time allows, we also take a look at Markov chains with random transition probabilities, so-called random walks in random environment, which is a prominent model in the field of random media. 

\subsubsection*{\large
    Literature:
}
J. R. Norris: \emph{Markov Chains}, Cambridge University Press, 1997
\subsubsection*{\large
    Prerequisites:
}
Required: Elementary Probability Theory~I \\ Recommended: Analysis~III,  Probability Theory~I
\subsubsection*{\large
    Remarks:
}
Dieser Kurs wird auf englisch angeboten.
\subsubsection*{\large
    Usability and assessments:
}

\begin{tabularx}{\textwidth}{ p{.5\textwidth}
    |X
    |X
}
 &
\makecell[c]{\rotatebox[origin=l]{90}{\parbox{
            4
            cm}{\begin{flushleft}
                Additional module in mathematics (MEd18) (3.0 ECTS) \newline Elective (MSc14) (6.0 ECTS) \newline Elective for individual studying (2HfB21) (6.0 ECTS)
            \end{flushleft} }}}
 &
\makecell[c]{\rotatebox[origin=l]{90}{\parbox{
            4
            cm}{\begin{flushleft}
                Compulsory elective module in mathematics (BSc21) (6.0 ECTS) \newline Elective in Data (MScData24) (6.0 ECTS) \newline part of the concentration module (MSc14) (5.25 ECTS) \newline part of the module ''Applied Mathematics'' (MSc14) (5.5 ECTS) \newline part of the module ''Mathematics'' (MSc14) (5.5 ECTS)
            \end{flushleft} }}}
\\
& 1
& 1
\\[2ex] \hline
\hline \rule[0mm]{0cm}{.6cm}PL:  \rule[-3mm]{0cm}{0cm}
 &
 &
\makecell[c]{\xmark}
\\
\hline \rule[0mm]{0cm}{.6cm}SL:  \rule[-3mm]{0cm}{0cm}
 &
\makecell[c]{\xmark}
 &
\makecell[c]{\xmark}
\\
\end{tabularx}


\clearpage\hrule\vskip1pt\hrule
\section*{\Large \href{https://pfaffelh.github.io/hp/2024WS_measure_theory.html}{Measure Theory}}
\addcontentsline{toc}{subsection}{Measure Theory\ \textcolor{gray}{(\em Peter Pfaffelhuber)}}
\vskip-2ex
{\it Peter Pfaffelhuber, Assistant: Samuel Adeosun}
\hfill{E}\\
Tutorial: 2 hours: Wed, 10--12 h, HS II, \href{https://www.openstreetmap.org/?mlat=48.002320&mlon=7.847924#map=19/48.002320/7.847924}{Albertstr. 23b}\\
 \\
% Webseite: \url{https://pfaffelh.github.io/hp/2024WS_measure_theory.html}
\subsubsection*{\large
    Content:
}
Measure Theory is the foundation of advanced probability theory. In this course, we build on knowledge in analysis and provide all necessary results for later classes in statistics, probabilistic machine learning and stochastic processes. It contains set systems, constructions of measures using outer measures, the integral, and product measures.
\subsubsection*{\large
    Literature:
}
\begin{itemize}
 \item H. Bauer. \emph{Measure and Integration Theory}. deGruyter, 2001. 
\item V. Bogatchev. \emph{Measure Theory}. Springer, 2007.
\item O. Kallenberg. \emph{Foundations of Modern Probability Theory}. Springer, 2021.
\end{itemize}
\subsubsection*{\large
    Prerequisites:
}
Basic courses in analysis, and an understanding of mathematical proofs.
\subsubsection*{\large
    Remarks:
}
This course is based on self-study of the material, but comes with graded exercises.
\subsubsection*{\large
    Usability and assessments:
}

\begin{tabularx}{\textwidth}{ p{.5\textwidth}
    |X
}
 &
\makecell[c]{\rotatebox[origin=l]{90}{\parbox{
            4
            cm}{\begin{flushleft}
                Elective in Data (MScData24) (6.0 ECTS)
            \end{flushleft} }}}
\\
& 1
\\[2ex] \hline
\hline \rule[0mm]{0cm}{.6cm}PL:  \rule[-3mm]{0cm}{0cm}
 &
\makecell[c]{\xmark}
\\
\hline \rule[0mm]{0cm}{.6cm}SL:  \rule[-3mm]{0cm}{0cm}
 &
\makecell[c]{\xmark}
\\
\end{tabularx}


\clearpage\hrule\vskip1pt\hrule
\section*{\Large \href{https://aam.uni-freiburg.de/agsa/lehre/ws24/numsde/index.html}{Numerical Approximation of Stochastic Differential Equations}}
\addcontentsline{toc}{subsection}{Numerical Approximation of Stochastic Differential Equations\ \textcolor{gray}{(\em Diyora Salimova)}}
\vskip-2ex
{\it Diyora Salimova, Assistant:  Ilkhom Mukhammadiev}
\hfill{E}\\
Lecture: Tue, Fri, 12--14 h, SR 226, \href{https://www.openstreetmap.org/?mlat=48.003472&mlon=7.848195#map=19/48.003472/7.848195}{Hermann-Herder-Str. 10}\\
Tutorial: 2 hours, date to be determined \\
Computer exercise: 2 hours, date to be determined \\
% Webseite: \url{https://aam.uni-freiburg.de/agsa/lehre/ws24/numsde/index.html}
\subsubsection*{\large
    Content:
}
The aim of this course is to enable the students to carry out simulations and their mathematical analysis for stochastic models originating from applications such as mathematical finance and physics. For this, the course teaches a decent knowledge on stochastic differential equations (SDEs) and their solutions. Furthermore, different numerical methods for SDEs, their underlying ideas, convergence properties, and implementation issues are studied.
\subsubsection*{\large
    Literature:
}
\begin{itemize}
\item
P. E. Kloeden and E. Platen: \emph{Numerical Solution of Stochastic Differential Equations.} Springer-Verlag, Berlin, 1992. 
\item
Bernt Oksendal: \emph{Stochastic Differential Equations}, Springer, 2010.
\end{itemize}
\subsubsection*{\large
    Prerequisites:
}
Probability and measure theory,  basic numerical analysis and basics of MATLAB programming.
\subsubsection*{\large
    Usability and assessments:
}

\begin{tabularx}{\textwidth}{ p{.5\textwidth}
    |X
    |X
}
 &
\makecell[c]{\rotatebox[origin=l]{90}{\parbox{
            4
            cm}{\begin{flushleft}
                Additional module in mathematics (MEd18) (3.0 ECTS) \newline Elective (MSc14) (6.0 ECTS) \newline Elective for individual studying (2HfB21) (6.0 ECTS)
            \end{flushleft} }}}
 &
\makecell[c]{\rotatebox[origin=l]{90}{\parbox{
            4
            cm}{\begin{flushleft}
                Compulsory elective module in mathematics (BSc21) (6.0 ECTS) \newline Elective in Data (MScData24) (6.0 ECTS) \newline part of the concentration module (MSc14) (5.25 ECTS) \newline part of the module ''Applied Mathematics'' (MSc14) (5.5 ECTS) \newline part of the module ''Mathematics'' (MSc14) (5.5 ECTS)
            \end{flushleft} }}}
\\
& 1
& 1
\\[2ex] \hline
\hline \rule[0mm]{0cm}{.6cm}PL:  \rule[-3mm]{0cm}{0cm}
 &
 &
\makecell[c]{\xmark}
\\
\hline \rule[0mm]{0cm}{.6cm}SL:  \rule[-3mm]{0cm}{0cm}
 &
\makecell[c]{\xmark}
 &
\makecell[c]{\xmark}
\\
\end{tabularx}


\clearpage\hrule\vskip1pt\hrule
\section*{\Large \href{https://www.syscop.de/teaching/ws2024/numerical-optimal-control}{Numerical Optimal Control}}
\addcontentsline{toc}{subsection}{Numerical Optimal Control\ \textcolor{gray}{(\em Moritz Diehl)}}
\vskip-2ex
{\it Moritz Diehl, Assistant: Florian Messerer}
\hfill{E}\\
Tutorial / flipped classroom: Tue, 14--16 h, HS II, \href{https://www.openstreetmap.org/?mlat=48.002320&mlon=7.847924#map=19/48.002320/7.847924}{Albertstr. 23b}\\
% Webseite: \url{https://www.syscop.de/teaching/ws2024/numerical-optimal-control}
\subsubsection*{\large
    Content:
}
The aim of the course is to give an introduction to numerical methods for the solution of optimal control problems in science and engineering. The focus is on both discrete time and continuous time optimal control in continuous state spaces. It is intended for a mixed audience of students from mathematics, engineering and computer science.

The course covers the following topics:
\begin{itemize}
\item Introduction to Dynamic Systems and Optimization
\item  Rehearsal of Newton-type methods and Numerical Optimization
\item  Algorithmic Differentiation
\item  Discrete Time Optimal Control
\item  Dynamic Programming
\item  Continuous Time Optimal Control
\item  Numerical Simulation Methods
\item  Hamilton–Jacobi–Bellmann Equation
\item  Pontryagin and the Indirect Approach
\item  Direct Optimal Control
\item  Real-Time Optimization for Model Predictive Control
\end{itemize}

The lecture is accompanied by intensive weekly computer exercises offered both in MATLAB and Python (6~ECTS) and an optional project (3~ECTS). The project consists in the formulation and implementation of a self-chosen optimal control problem and numerical solution method, resulting in documented computer code, a project report, and a public presentation.
\subsubsection*{\large
    Literature:
}
\begin{itemize}
\item
 M.~Diehl, S.~Gros: \href{https://www.syscop.de/files/2020ss/NOC/book-NOCSE.pdf}{\emph{Numerical Optimal Control}}, lecture notes. 
\item
J.B.~Rawlings, D.Q.~Mayne, M.~Diehl: \href{https://sites.engineering.ucsb.edu/\~jbraw/mpc/MPC-book-2nd-edition-4th-printing.pdf}{\emph{Model Predictive Control}}, 2nd Edition, Nobhill Publishing, 2017.
\item
J.~Betts: \emph{Practical Methods for Optimal Control and Estimation Using Nonlinear Programming}, SIAM, 2010.
\end{itemize}
\subsubsection*{\large
    Prerequisites:
}
Required: Analysis~I and II, Linear Algebra~I and II \\
Recommended: Numerics I, Ordinary Differential Equations, Numerical Optimization
\subsubsection*{\large
    Remarks:
}
Together with the optional programming project, the 6 ECTS lecture counts as a 9 ECTS course.
\subsubsection*{\large
    Usability and assessments:
}

\begin{tabularx}{\textwidth}{ p{.5\textwidth}
    |X
    |X
}
 &
\makecell[c]{\rotatebox[origin=l]{90}{\parbox{
            4
            cm}{\begin{flushleft}
                Additional module in mathematics (MEd18) (3.0 ECTS) \newline Elective (MSc14) (6.0 ECTS) \newline Elective for individual studying (2HfB21) (6.0 ECTS)
            \end{flushleft} }}}
 &
\makecell[c]{\rotatebox[origin=l]{90}{\parbox{
            4
            cm}{\begin{flushleft}
                Compulsory elective module in mathematics (BSc21) (6.0 ECTS) \newline Elective in Data (MScData24) (6.0 ECTS) \newline part of the concentration module (MSc14) (5.25 ECTS) \newline part of the module ''Applied Mathematics'' (MSc14) (5.5 ECTS) \newline part of the module ''Mathematics'' (MSc14) (5.5 ECTS)
            \end{flushleft} }}}
\\
& 1
& 1
\\[2ex] \hline
\hline \rule[0mm]{0cm}{.6cm}PL:  \rule[-3mm]{0cm}{0cm}
 &
 &
\makecell[c]{\xmark}
\\
\hline \rule[0mm]{0cm}{.6cm}SL:  \rule[-3mm]{0cm}{0cm}
 &
\makecell[c]{\xmark}
 &
\makecell[c]{\xmark}
\\
\end{tabularx}


\clearpage
\phantomsection
\thispagestyle{empty}
\vspace*{\fill}
\begin{center}
    \Huge\bfseries 2a. Mathematics Education
\end{center}
\addcontentsline{toc}{section}{\textbf{2a. Mathematics Education}}
\addtocontents{toc}{\medskip\hrule\medskip}\vspace*{\fill}\vspace*{\fill}\clearpage
\vfill
\thispagestyle{empty}
\clearpage

\clearpage\hrule\vskip1pt\hrule
\section*{\Large Introduction to Mathematics Education}
\addcontentsline{toc}{subsection}{Introduction to Mathematics Education\ \textcolor{gray}{(\em Katharina Böcherer-Linder)}}
\vskip-2ex
{\it Katharina Böcherer-Linder}
\hfill{D}\\
Mon 10--12 h, SR 226, \href{https://www.openstreetmap.org/?mlat=48.003472&mlon=7.848195#map=19/48.003472/7.848195}{Hermann-Herder-Str. 10}, Fri, 8--10 h, SR 127, \href{https://www.openstreetmap.org/?mlat=48.000637&mlon=7.846006#map=19/48.000636/7.846006}{Ernst-Zermelo-Str. 1}\\
 Fri, 14--16 h, SR 127, \href{https://www.openstreetmap.org/?mlat=48.000637&mlon=7.846006#map=19/48.000636/7.846006}{Ernst-Zermelo-Str. 1}\\
\subsubsection*{\large
    Content:
}
Mathematics didactic principles and their learning theory foundations and possibilities of teaching implementation (also e.g. with the help of digital media). \\
Theoretical concepts on central mathematical thinking activities such as concept formation, modeling, problem solving and reasoning. \\
Mathematics didactic constructs: Barriers to understanding, pre-concepts, basic ideas, specific difficulties with selected mathematical content. \\
Concepts for dealing with heterogeneity, taking into account subject-specific characteristics particularities (e.g. dyscalculia or mathematical giftedness).\\\
Levels of conceptual rigour and formalization as well as their age-appropriate implementation.
\subsubsection*{\large
    Prerequisites:
}
Required: Analysis~I, Linear Algebra~I
\subsubsection*{\large
    Remarks:
}
The course is compulsory in the teaching degree option of the two-main-subject Bachelor`s degree program. It is made up of lecture components and parts with exercise and seminar character. The three forms of teaching cannot be not be completely separated from each other.
Attendance at the “Didactic Seminar” (approximately fortnightly, tuesday evenings, 19:30) is expected!\\
This course is only offered in German.
\subsubsection*{\large
    Usability and assessments:
}

\begin{tabularx}{\textwidth}{ p{.5\textwidth}
    |X
}
 &
\makecell[c]{\rotatebox[origin=l]{90}{\parbox{
            4
            cm}{\begin{flushleft}
                (Introduction to) Mathematics Education (2HfB21, MEH21, MEB21, MEdual24) (5.0 ECTS)
            \end{flushleft} }}}
\\
& 1
\\[2ex] \hline
\hline \rule[0mm]{0cm}{.6cm}SL:  \rule[-3mm]{0cm}{0cm}
 &
\makecell[c]{\xmark}
\\
\end{tabularx}


\clearpage\hrule\vskip1pt\hrule
\section*{\Large Mathematics Education ‒ Functions and Analysis}
\addcontentsline{toc}{subsection}{Mathematics Education ‒ Functions and Analysis\ \textcolor{gray}{(\em Katharina Böcherer-Linder)}}
\vskip-2ex
{\it Katharina Böcherer-Linder}
\hfill{D}\\
Seminar: Thu, 9--12 h, SR 404, \href{https://www.openstreetmap.org/?mlat=48.000637&mlon=7.846006#map=19/48.000636/7.846006}{Ernst-Zermelo-Str. 1}\\
\subsubsection*{\large
    Content:
}
Exemplary implementations of the theoretical concepts of central mathematical thought processes such as concept formation, modeling, problem solving and reasoning for the content areas of functions and analysis. \\
Barriers to understanding, pre-concepts, basic ideas, specific difficulties for the content areas of functions
and analysis. \\
Fundamental possibilities and limitations of media, in particular of computer-aided mathematical tools mathematical tools and their application for the content areas of functions and analysis.
Analysis of individual mathematical learning processes and errors as well as development individual support measures for
the content areas of functions and analysis.
\subsubsection*{\large
    Literature:
}
\begin{itemize}
\item
R. Dankwerts, D. Vogel: \emph{Analysis verständlich unterrichten}. Heidelberg: Spektrum, 2006. 
 \item
G. Greefrath, R. Oldenburg, H.-S. Siller, V. Ulm, H.-G. Weigand: \emph{Didaktik der Analysis. Aspekte und Grundvorstellungen zentraler Begriffe}. Berlin, Heidelberg: Springer 2016.
\end{itemize}
\subsubsection*{\large
    Prerequisites:
}
Required: Introduction to the didactics of mathematics, Knowledge about analysis and numerics
\subsubsection*{\large
    Remarks:
}
The two parts can be completed in different semesters semesters, but have a joint final exam, which is offered every semester and written after completing both parts. \\ This course is only offered in German.
\subsubsection*{\large
    Usability and assessments:
}

\begin{tabularx}{\textwidth}{ p{.5\textwidth}
    |X
}
 &
\makecell[c]{\rotatebox[origin=l]{90}{\parbox{
            4
            cm}{\begin{flushleft}
                Mathematics Education for Specific Areas of Mathematics (MEd18, MEH21, MEB21) (3.0 ECTS)
            \end{flushleft} }}}
\\
& 1
\\[2ex] \hline
\hline \rule[0mm]{0cm}{.6cm}PL:  \rule[-3mm]{0cm}{0cm}
 &
\makecell[c]{\xmark}
\\
\hline \rule[0mm]{0cm}{.6cm}SL:  \rule[-3mm]{0cm}{0cm}
 &
\makecell[c]{\xmark}
\\
\end{tabularx}


\clearpage\hrule\vskip1pt\hrule
\section*{\Large Mathematics Education ‒ Probability Theory and Algebra}
\addcontentsline{toc}{subsection}{Mathematics Education ‒ Probability Theory and Algebra\ \textcolor{gray}{(\em Anika Dreher)}}
\vskip-2ex
{\it Anika Dreher}
\hfill{D}\\
Seminar: Fri, 9--12 h, SR 226, \href{https://www.openstreetmap.org/?mlat=48.003472&mlon=7.848195#map=19/48.003472/7.848195}{Hermann-Herder-Str. 10}\\
\subsubsection*{\large
    Content:
}
Exemplary implementations of the theoretical concepts of central mathematical thought processes such as concept formation, modeling, problem solving and reasoning for the content areas of stochastics and algebra. \\
Barriers to understanding, pre-concepts, basic ideas, specific difficulties for the content areas of stochastics and algebra.\\
Basic possibilities and limitations of media, especially computer-based mathematical tools and their mathematical tools and their application for the content areas of stochastics and algebra. and algebra. \\
Analysis of individual mathematical learning processes and errors as well as development individual support measures for the content areas of stochastics and algebra.
\subsubsection*{\large
    Literature:
}
\begin{itemize}
\item 
G. Malle: \emph{Didaktische Probleme der elementaren Algebra}. Braunschweig, Wiesbaden: Vieweg 1993. 
\item
A. Eichler, M. Vogel: \emph{Leitidee Daten und Zufall. Von konkreten Beispielen zur Didaktik der Stochastik}. Wiesbaden:
Vieweg 2009.
\end{itemize}
\subsubsection*{\large
    Prerequisites:
}
Required: Introduction to the didactics of mathematics, knowledge from stochastics and algebra.
\subsubsection*{\large
    Remarks:
}
The two parts can be completed in different semesters semesters, but have a joint final exam, which is offered every semester and written after completing both parts. \\ This course is only offered in German.
\subsubsection*{\large
    Usability and assessments:
}

\begin{tabularx}{\textwidth}{ p{.5\textwidth}
    |X
}
 &
\makecell[c]{\rotatebox[origin=l]{90}{\parbox{
            4
            cm}{\begin{flushleft}
                Mathematics Education for Specific Areas of Mathematics (MEd18, MEH21, MEB21) (3.0 ECTS)
            \end{flushleft} }}}
\\
& 1
\\[2ex] \hline
\hline \rule[0mm]{0cm}{.6cm}PL:  \rule[-3mm]{0cm}{0cm}
 &
\makecell[c]{\xmark}
\\
\hline \rule[0mm]{0cm}{.6cm}SL:  \rule[-3mm]{0cm}{0cm}
 &
\makecell[c]{\xmark}
\\
\end{tabularx}


\clearpage\hrule\vskip1pt\hrule
\section*{\Large Mathematics education seminar: Media Use in Teaching Mathematics}
\addcontentsline{toc}{subsection}{Mathematics education seminar: Media Use in Teaching Mathematics\ \textcolor{gray}{(\em Jürgen Kury)}}
\vskip-2ex
{\it Jürgen Kury}
\hfill{D}\\
Seminar: Wed, 15--18 h, SR 127, \href{https://www.openstreetmap.org/?mlat=48.000637&mlon=7.846006#map=19/48.000636/7.846006}{Ernst-Zermelo-Str. 1}\\
\subsubsection*{\large
    Content:
}
The use of teaching media in mathematics lessons wins both at the level of lesson planning and lesson realization in importance. Against the background of constructivist learning theories shows that the reflective use of computer programs, among other things mathematical concept formation in the long term. For example experimenting with computer programs allows mathematical structures to be discovered, without this being overshadowed by individual routine operations (such as term transformation) would be covered up. This has far-reaching consequences for mathematics lessons. For this reason, this seminar aims to provide students the necessary decision-making and action skills to prepare future mathematics teachers for their professional activities. Starting from initial considerations about lesson planning, computers and tablets with regard to their respective didactic potential and tested with learners during a classroom visit. The exemplary systems presented are:
\begin{itemize}
\item dynamic geometry Software: Geogebra
\item Spreadsheets: Excel
\item Apps for Smartphones and tablets
\end{itemize}
The students should develop teaching sequences, which will then be tested and reflected on with pupils (where this will be possible).
\subsubsection*{\large
    Prerequisites:
}
Recommended: Basic Courses in mathematics \\ This course will only be offered in German.
\subsubsection*{\large
    Usability and assessments:
}

\begin{tabularx}{\textwidth}{ p{.5\textwidth}
}
\\
\\[2ex] \hline
\end{tabularx}


\clearpage\hrule\vskip1pt\hrule
\section*{\Large Mathematics education seminars at Freiburg University of Education}
\addcontentsline{toc}{subsection}{Mathematics education seminars at Freiburg University of Education\ \textcolor{gray}{(\em Dozent:inn:en der PH Freiburg)}}
\vskip-2ex
{\it Dozent:inn:en der PH Freiburg}
\hfill{D}\\
\subsubsection*{\large
    Content:
}
Für das Modul „Fachdidaktische Entwicklung“ können auch geeignete Veranstaltungen an der PH Freiburg absolviert
werden, sofern dort Studienplätze zur Verfügung stehen. Ob Veranstaltungen geeignet sind, sprechen Sie bitte vorab
mit Frau Böcherer-Linder ab; ob Studienplätze zur Verfügung stehen, müssen Sie bei Interessen an einer Veranstaltung
von den Dozent:inn:en erfragen.
\subsubsection*{\large
    Prerequisites:
}
Für das Modul „Fachdidaktische Entwicklung“ können auch geeignete Veranstaltungen an der PH Freiburg absolviert
werden, sofern dort Studienplätze zur Verfügung stehen. Ob Veranstaltungen geeignet sind, sprechen Sie bitte vorab
mit Frau Böcherer-Linder ab; ob Studienplätze zur Verfügung stehen, müssen Sie bei Interessen an einer Veranstaltung
von den Dozent:inn:en erfragen.
\subsubsection*{\large
    Remarks:
}
For the module "Fachdidaktische Entwicklung", suitable suitable courses can also be completed at the PH Freiburg if places are available there. To find out whether courses are suitable are suitable, please discuss in advance with Ms. Böcherer-Linder in advance; you must check whether places are available if you are interested in a course from the lecturers if you are interested in a course. \\ Most suitable courses will be offered in German.
\subsubsection*{\large
    Usability and assessments:
}

\begin{tabularx}{\textwidth}{ p{.5\textwidth}
}
\\
\\[2ex] \hline
\end{tabularx}


\clearpage\hrule\vskip1pt\hrule
\section*{\Large Module ''Research in Mathematics Education'':}
\addcontentsline{toc}{subsection}{Module ''Research in Mathematics Education'':\ \textcolor{gray}{(\em Dozent:inn:en der PH Freiburg)}}
\vskip-2ex
{\it Dozent:inn:en der PH Freiburg}
\subsubsection*{\large
    Content:
}
The three related courses of the module prepare students for an empirical Master thesis in mathematics didactics. The course is jointly designed by all professors at the PH with mathematics didactics research projects at secondary levels 1 and 2 and is carried out by one of these researchers. Afterwards, students have the opportunity to start Master thesis with one of these supervisors - usually integrated into larger ongoing research projects.
The main objectives of the module are the ability to receive mathematics didactic research in order to didactic research to clarify questions of practical relevance and to plan an empirical mathematics didactics Master thesis.
It will be held as a mixture of seminar, development of research topics in groups and active work with research data. Recommended literature will be depending on the research topics offered within the respective courses. The parts can also be attended in different semesters, for example part~1 in the second Master semester and part~2 in the compact phase of the third Master semester after the practical semester.
\subsubsection*{\large
    Remarks:
}
Three-part module for M.Ed. students who would like to write a didactic Master thesis in mathematics. Participation only after personal registration by the end of the lecture period of the previous semester in the Department of Didactics. Admission capacity is limited. \par
Pre-registration: If you would like to take part in this module, please register by by 30.09.2024 by e-mail to
\href{mailto:didaktik@math.uni-freiburg.de}{didaktik@math.uni-freiburg.de} and to \href{mailto:erens@ph-freiburg.de}{Ralf Erens}.
This course will only be offered in German.
\subsubsection*{\large
    Usability and assessments:
}

\begin{tabularx}{\textwidth}{ p{.5\textwidth}
}
\\
\\[2ex] \hline
\end{tabularx}


\clearpage\hrule\vskip1pt\hrule
\section*{\Large Part 1: Development Research in Selected Focus Areas of Mathematics Education}
\addcontentsline{toc}{subsection}{Part 1: Development Research in Selected Focus Areas of Mathematics Education\ \textcolor{gray}{(\em Frank Reinhold)}}
\vskip-2ex
{\it Frank Reinhold}
\hfill{D}\\
Seminar: Mon, 14--16 h, Raum (PH) noch nicht bekannt, \href{https://www.openstreetmap.org/?mlat=47.98132\&mlon=7.89420\#map=17/47.98132/7.89420}{PH Freiburg}\\
\subsubsection*{\large
    Content:
}
This first course of the module provides an introduction to strategies of empirical didactic research (research questions, research status, research designs). Students deepen their skills in scientific research and the evaluation of subject-specific didactic research.
\subsubsection*{\large
    Remarks:
}
This course will only be offered in German.
\subsubsection*{\large
    Usability and assessments:
}

\begin{tabularx}{\textwidth}{ p{.5\textwidth}
    |X
}
 &
\makecell[c]{\rotatebox[origin=l]{90}{\parbox{
            4
            cm}{\begin{flushleft}
                Research in Mathematics Education (MEd18, MEH21, MEB21) (6.0 ECTS)
            \end{flushleft} }}}
\\
& 1
\\[2ex] \hline
\hline \rule[0mm]{0cm}{.6cm}SL:  \rule[-3mm]{0cm}{0cm}
 &
\makecell[c]{\xmark}
\\
\end{tabularx}


\clearpage\hrule\vskip1pt\hrule
\section*{\Large Part 2: Research Methods in Mathematics Education}
\addcontentsline{toc}{subsection}{Part 2: Research Methods in Mathematics Education\ \textcolor{gray}{(\em Frank Reinhold)}}
\vskip-2ex
{\it Frank Reinhold}
\hfill{D}\\
Seminar: Mon, 16--19 h, Raum (PH) noch nicht bekannt, \href{https://www.openstreetmap.org/?mlat=47.98132\&mlon=7.89420\#map=17/47.98132/7.89420}{PH Freiburg}\\
\subsubsection*{\large
    Content:
}
In the second course of the module (in the last third of the semester) students are introduced to central qualitative and quantitative research methods through concrete work with existing data (interviews, student products, experimental data), students are introduced to central qualitative and quantitative research methods.
\subsubsection*{\large
    Usability and assessments:
}

\begin{tabularx}{\textwidth}{ p{.5\textwidth}
    |X
}
 &
\makecell[c]{\rotatebox[origin=l]{90}{\parbox{
            4
            cm}{\begin{flushleft}
                Research in Mathematics Education (MEd18, MEH21, MEB21) (6.0 ECTS)
            \end{flushleft} }}}
\\
& 1
\\[2ex] \hline
\hline \rule[0mm]{0cm}{.6cm}SL:  \rule[-3mm]{0cm}{0cm}
 &
\makecell[c]{\xmark}
\\
\end{tabularx}


\clearpage\hrule\vskip1pt\hrule
\section*{\Large Part 3: Developing and Optimising a Research Project in Mathematics Education}
\addcontentsline{toc}{subsection}{Part 3: Developing and Optimising a Research Project in Mathematics Education\ \textcolor{gray}{(\em Dozent:inn:en der PH Freiburg)}}
\vskip-2ex
{\it Dozent:inn:en der PH Freiburg}
\hfill{D}\\
 \\
\subsubsection*{\large
    Content:
}
Accompanying seminar for the Master thesis
\subsubsection*{\large
    Remarks:
}
This seminar will only be offered in German.
\subsubsection*{\large
    Usability and assessments:
}

\begin{tabularx}{\textwidth}{ p{.5\textwidth}
    |X
}
 &
\makecell[c]{\rotatebox[origin=l]{90}{\parbox{
            4
            cm}{\begin{flushleft}
                Research in Mathematics Education (MEd18, MEH21, MEB21) (6.0 ECTS)
            \end{flushleft} }}}
\\
& 1
\\[2ex] \hline
\hline \rule[0mm]{0cm}{.6cm}SL:  \rule[-3mm]{0cm}{0cm}
 &
\makecell[c]{\xmark}
\\
\end{tabularx}


\clearpage
\phantomsection
\thispagestyle{empty}
\vspace*{\fill}
\begin{center}
    \Huge\bfseries 2b. Tutorial Module
\end{center}
\addcontentsline{toc}{section}{\textbf{2b. Tutorial Module}}
\addtocontents{toc}{\medskip\hrule\medskip}\vspace*{\fill}\vspace*{\fill}\clearpage
\vfill
\thispagestyle{empty}
\clearpage

\clearpage\hrule\vskip1pt\hrule
\section*{\Large \href{https://home.mathematik.uni-freiburg.de/ldl/index1.html}{Learning by Teaching}}
\addcontentsline{toc}{subsection}{Learning by Teaching\ \textcolor{gray}{(\em Susanne Knies)}}
\vskip-2ex
{\it Susanne Knies}
\hfill{D}\\
% Webseite: \url{https://home.mathematik.uni-freiburg.de/ldl/index1.html}
\subsubsection*{\large
    Content:
}
What characterizes a good tutorial? This question will be discussed in the first workshop and tips and suggestions will be given. Experiences will be shared in the second workshop.

\subsubsection*{\large
    Remarks:
}
Prerequisite for participation is a tutoring position for a lecture of the Institute of Mathematics in the current semester (at least one two-hour or two one-hour tutorial groups over the whole semester). \\ Can be used twice in the M.Sc. program in Mathematics. \\ This course is only offered in German.
\subsubsection*{\large
    Usability and assessments:
}

\begin{tabularx}{\textwidth}{ p{.5\textwidth}
    |X
}
 &
\makecell[c]{\rotatebox[origin=l]{90}{\parbox{
            4
            cm}{\begin{flushleft}
                Elective (BSc21) (3.0 ECTS) \newline Elective (MSc14) (3.0 ECTS) \newline Elective (MScData24) (3.0 ECTS) \newline Elective for individual studying (2HfB21) (3.0 ECTS)
            \end{flushleft} }}}
\\
& 1
\\[2ex] \hline
\hline \rule[0mm]{0cm}{.6cm}Kommentar:  \rule[-3mm]{0cm}{0cm}
 &
\makecell[c]{\xmark}
\\
\hline \rule[0mm]{0cm}{.6cm}SL:  \rule[-3mm]{0cm}{0cm}
 &
\makecell[c]{\xmark}
\\
\end{tabularx}


\clearpage
\phantomsection
\thispagestyle{empty}
\vspace*{\fill}
\begin{center}
    \Huge\bfseries 2c. Computer Exercises
\end{center}
\addcontentsline{toc}{section}{\textbf{2c. Computer Exercises}}
\addtocontents{toc}{\medskip\hrule\medskip}\vspace*{\fill}\vspace*{\fill}\clearpage
\vfill
\thispagestyle{empty}
\clearpage

\clearpage\hrule\vskip1pt\hrule
\section*{\Large Computer exercises for Introduction to Theory and Numerics of Partial Differential Equations}
\addcontentsline{toc}{subsection}{Computer exercises for Introduction to Theory and Numerics of Partial Differential Equations\ \textcolor{gray}{(\em Sören Bartels)}}
\vskip-2ex
{\it Sören Bartels, Assistant: Vera Jackisch}
\hfill{E}\\
Computer exercise: 2 hours, date to be determined \\
\subsubsection*{\large
    Content:
}
The computer tutorial accompanies the lecture with programming exercises.
\subsubsection*{\large
    Prerequisites:
}
See the lecture -- additionally: programming knowledge.
\subsubsection*{\large
    Remarks:
}
Dieser Kurs wird auf Englisch angeboten.
\subsubsection*{\large
    Usability and assessments:
}

\begin{tabularx}{\textwidth}{ p{.5\textwidth}
    |X
}
 &
\makecell[c]{\rotatebox[origin=l]{90}{\parbox{
            4
            cm}{\begin{flushleft}
                Additional module in mathematics (MEd18) (3.0 ECTS) \newline Elective (BSc21) (3.0 ECTS) \newline Elective (MSc14) (3.0 ECTS) \newline Elective (MScData24) (3.0 ECTS) \newline Elective for individual studying (2HfB21) (3.0 ECTS)
            \end{flushleft} }}}
\\
& 1
\\[2ex] \hline
\hline \rule[0mm]{0cm}{.6cm}SL:  \rule[-3mm]{0cm}{0cm}
 &
\makecell[c]{\xmark}
\\
\end{tabularx}


\clearpage\hrule\vskip1pt\hrule
\section*{\Large Computer exercises in Numerics}
\addcontentsline{toc}{subsection}{Computer exercises in Numerics\ \textcolor{gray}{(\em Sören Bartels)}}
\vskip-2ex
{\it Sören Bartels, Assistant: Tatjana Schreiber}
\hfill{D}\\
 \\
\subsubsection*{\large
    Content:
}
In the computer tutorial accompanying the Numerics (first term) lecture the algorithms developed and analyzed in the lecture are put into practice and and tested experimentally. The implementation is carried out in the programming languages Matlab, C++ and Python. Elementary programming knowledge is assumed.
\subsubsection*{\large
    Prerequisites:
}
See the lecture {\em Numerics I} (which should be attended in parallel or should already have been completed).
Additionally: Elementary programming knowledge.
\subsubsection*{\large
    Remarks:
}
This course is only offered in German.
\subsubsection*{\large
    Usability and assessments:
}

\begin{tabularx}{\textwidth}{ p{.5\textwidth}
    |X
}
 &
\makecell[c]{\rotatebox[origin=l]{90}{\parbox{
            4
            cm}{\begin{flushleft}
                Additional module in mathematics (MEd18) (3.0 ECTS) \newline Computer Exercise (2HfB21, MEH21, MEB21) (3.0 ECTS) \newline Elective for individual studying (2HfB21) (3.0 ECTS) \newline Numerics (BSc21) (3.0 ECTS)
            \end{flushleft} }}}
\\
& 1
\\[2ex] \hline
\hline \rule[0mm]{0cm}{.6cm}Kommentar:  \rule[-3mm]{0cm}{0cm}
 &
\makecell[c]{\xmark}
\\
\hline \rule[0mm]{0cm}{.6cm}SL:  \rule[-3mm]{0cm}{0cm}
 &
\makecell[c]{\xmark}
\\
\end{tabularx}


\clearpage
\phantomsection
\thispagestyle{empty}
\vspace*{\fill}
\begin{center}
    \Huge\bfseries 3a. Undergraduate Seminars
\end{center}
\addcontentsline{toc}{section}{\textbf{3a. Undergraduate Seminars}}
\addtocontents{toc}{\medskip\hrule\medskip}\vspace*{\fill}\vspace*{\fill}\clearpage
\vfill
\thispagestyle{empty}
\clearpage

\clearpage\hrule\vskip1pt\hrule
\section*{\Large \href{ https://home.mathematik.uni-freiburg.de/knies/lehre/ws2425/index.html}{Ordinary Differential Equations and Applications}}
\addcontentsline{toc}{subsection}{Ordinary Differential Equations and Applications\ \textcolor{gray}{(\em Susanne Knies, Ludwig Striet)}}
\vskip-2ex
{\it Susanne Knies, Ludwig Striet}
\hfill{D}\\
Seminar: Thu, 12--14 h, SR 125, \href{https://www.openstreetmap.org/?mlat=48.000637&mlon=7.846006#map=19/48.000636/7.846006}{Ernst-Zermelo-Str. 1}\\
Preliminary Meeting 15.07., 13 h, SR 403, \href{https://www.openstreetmap.org/?mlat=48.000637&mlon=7.846006#map=19/48.000636/7.846006}{Ernst-Zermelo-Str. 1}\\
% Webseite: \url{ https://home.mathematik.uni-freiburg.de/knies/lehre/ws2425/index.html}
\subsubsection*{\large
    Content:
}
Numerous dynamic processes in the natural sciences can be modeled by ordinary differential equations. In this proseminar we will deal with explicit solution methods for differential equations as well as the application situations (reaction kinetics, predator-prey models, mathematical pendulum, different growth processes, . . . ) which can be described by them.
\subsubsection*{\large
    Literature:
}
Lecture topics and literature can be found on the website!
\subsubsection*{\large
    Prerequisites:
}
Analysis~I and II, Lineare Algebra~I and II
\subsubsection*{\large
    Remarks:
}
Note that this course is only offered in German.
\subsubsection*{\large
    Usability and assessments:
}

\begin{tabularx}{\textwidth}{ p{.5\textwidth}
    |X
}
 &
\makecell[c]{\rotatebox[origin=l]{90}{\parbox{
            4
            cm}{\begin{flushleft}
                Undergraduate Seminar (2HfB21, BSc21, MEH21, MEB21) (3.0 ECTS)
            \end{flushleft} }}}
\\
& 1
\\[2ex] \hline
\hline \rule[0mm]{0cm}{.6cm}PL:  \rule[-3mm]{0cm}{0cm}
 &
\makecell[c]{\xmark}
\\
\hline \rule[0mm]{0cm}{.6cm}SL:  \rule[-3mm]{0cm}{0cm}
 &
\makecell[c]{\xmark}
\\
\end{tabularx}


\clearpage\hrule\vskip1pt\hrule
\section*{\Large \href{https://www.stochastik.uni-freiburg.de/de/lehre/ws-2024-2025/proseminar-streifzug-mathematik-ws-2024-2025/info-proseminar-streifzug-mathematik-ws-2024-2025}{A Ramble through Mathematics}}
\addcontentsline{toc}{subsection}{A Ramble through Mathematics\ \textcolor{gray}{(\em Angelika Rohde)}}
\vskip-2ex
{\it Angelika Rohde, Assistant: Johannes Brutsche}
\hfill{D}\\
Seminar: Wed, 12--14 h, SR 125, \href{https://www.openstreetmap.org/?mlat=48.000637&mlon=7.846006#map=19/48.000636/7.846006}{Ernst-Zermelo-Str. 1}\\
Preregistration \\
Preliminary Meeting 16.07., 10: 15 h, Raum 232, \href{https://www.openstreetmap.org/?mlat=48.000637&mlon=7.846006#map=19/48.000636/7.846006}{Ernst-Zermelo-Str. 1}\\
% Webseite: \url{https://www.stochastik.uni-freiburg.de/de/lehre/ws-2024-2025/proseminar-streifzug-mathematik-ws-2024-2025/info-proseminar-streifzug-mathematik-ws-2024-2025}
\subsubsection*{\large
    Content:
}
Paul Erd\H{o}s  liked to talk about the \emph{BOOK} in which God keeps the \textit{perfect} proofs of mathematical theorems, according to the famous quote by G. H. Hardy that "there is no permanent place for ugly mathematics" ([1], Preface). In an attempt at a best approximation to this \emph{BOOK}, Aigner and Ziegler have published a large number of sentences with elegant, sophisticated, and sometimes surprising evidence. In this proseminar, a selection of these results will be presented. The spectrum of topics covers all different areas of mathematics, from number theory, geometry, analysis, and combinatorics to graph theory and includes well-known results, such as Littlewood and Offord's lemma, the Dinitz problem, Hilbert's third problem (of his 23 problems presented at the International Congress of Mathematicians in Paris in 1900), the Borsuk conjecture, and many more.
\subsubsection*{\large
    Literature:
}
[1] Martin Aigner, Günter M. Ziegler: \emph{Das BUCH der Beweise} (5. Auf\/lage), Springer, 2018.
\subsubsection*{\large
    Prerequisites:
}
Linear Algebra~I and II, Analysis~I and II
\subsubsection*{\large
    Remarks:
}
Note that this course is only offered in German.
\subsubsection*{\large
    Usability and assessments:
}

\begin{tabularx}{\textwidth}{ p{.5\textwidth}
    |X
}
 &
\makecell[c]{\rotatebox[origin=l]{90}{\parbox{
            4
            cm}{\begin{flushleft}
                Undergraduate Seminar (2HfB21, BSc21, MEH21, MEB21) (3.0 ECTS)
            \end{flushleft} }}}
\\
& 1
\\[2ex] \hline
\hline \rule[0mm]{0cm}{.6cm}PL:  \rule[-3mm]{0cm}{0cm}
 &
\makecell[c]{\xmark}
\\
\hline \rule[0mm]{0cm}{.6cm}SL:  \rule[-3mm]{0cm}{0cm}
 &
\makecell[c]{\xmark}
\\
\end{tabularx}


\clearpage\hrule\vskip1pt\hrule
\section*{\Large \href{ https://home.mathematik.uni-freiburg.de/soergel/ws2425ps.html}{Undergraduate seminar in Algebra}}
\addcontentsline{toc}{subsection}{Undergraduate seminar in Algebra\ \textcolor{gray}{(\em Wolfgang Soergel)}}
\vskip-2ex
{\it Wolfgang Soergel, Assistant: Damian Sercombe}
\hfill{D}\\
Seminar: Tue, 14--16 h, SR 127, \href{https://www.openstreetmap.org/?mlat=48.000637&mlon=7.846006#map=19/48.000636/7.846006}{Ernst-Zermelo-Str. 1}\\
Preregistration \\
% Webseite: \url{ https://home.mathematik.uni-freiburg.de/soergel/ws2425ps.html}
\subsubsection*{\large
    Content:
}
In this proseminar we will discuss topics that are found in various textbooks and scripts for basic lectures in linear algebra but which are not part of the standard material. The lectures build on each other only slightly.
\subsubsection*{\large
    Prerequisites:
}
Linear Algebra ~I and II, Analysis~I and II.
\subsubsection*{\large
    Remarks:
}
This course is only offered in German.
\subsubsection*{\large
    Usability and assessments:
}

\begin{tabularx}{\textwidth}{ p{.5\textwidth}
    |X
}
 &
\makecell[c]{\rotatebox[origin=l]{90}{\parbox{
            4
            cm}{\begin{flushleft}
                Undergraduate Seminar (2HfB21, BSc21, MEH21, MEB21) (3.0 ECTS)
            \end{flushleft} }}}
\\
& 1
\\[2ex] \hline
\hline \rule[0mm]{0cm}{.6cm}PL:  \rule[-3mm]{0cm}{0cm}
 &
\makecell[c]{\xmark}
\\
\hline \rule[0mm]{0cm}{.6cm}SL:  \rule[-3mm]{0cm}{0cm}
 &
\makecell[c]{\xmark}
\\
\end{tabularx}


\clearpage
\phantomsection
\thispagestyle{empty}
\vspace*{\fill}
\begin{center}
    \Huge\bfseries 3b. Seminars
\end{center}
\addcontentsline{toc}{section}{\textbf{3b. Seminars}}
\addtocontents{toc}{\medskip\hrule\medskip}\vspace*{\fill}\vspace*{\fill}\clearpage
\vfill
\thispagestyle{empty}
\clearpage

\clearpage\hrule\vskip1pt\hrule
\section*{\Large \href{https://www.stochastik.uni-freiburg.de/de/lehre/ws-2024-2025/seminar-knotentheorie-ws-2024-2025}{Knot Theory}}
\addcontentsline{toc}{subsection}{Knot Theory\ \textcolor{gray}{(\em Ernst August v. Hammerstein)}}
\vskip-2ex
{\it Ernst August v. Hammerstein}
\hfill{D}\\
Seminar \\
Preregistration \\
Preliminary Meeting 19.07., 16 h, Raum 232, \href{https://www.openstreetmap.org/?mlat=48.000637&mlon=7.846006#map=19/48.000636/7.846006}{Ernst-Zermelo-Str. 1}\\
% Webseite: \url{https://www.stochastik.uni-freiburg.de/de/lehre/ws-2024-2025/seminar-knotentheorie-ws-2024-2025}
\subsubsection*{\large
    Content:
}
A knot can be mathematically defined relatively simply as a closed curve in the three-dimensional space $\mathbb{R}^3$. From everyday life, one is certainly already familiar with different types of knots, e.g, surgeon`s knot, sailor`s knots, and many more. The aim of mathematical knot theory is to find characteristic quantities for the description and classification of knots and thus possibly also to be able to decide whether two knots are equivalent,  i.e., if they can be transformed into one another through certain operations.
Ropes, cords or wires can be used to illustrate knots as well as interlacings. Prospective teachers can use these not only in this seminar, but perhaps also later in the classroom to display different results in a very practical way.
\subsubsection*{\large
    Literature:
}
\begin{itemize} 
\item C.C. Adams: \textit{The Knot Book: An elementary introduction to the mathematical theory of knots}, Revised reprint, AMS, 2004.\\
A pdf file of a preliminary version can be found under \url{https://www.math.cuhk.edu.hk/course\_builder/1920/math4900e/Adams--The\%20Knot\%20Book.pdf}.            
\item G. Burde, H. Zieschang: \href{https://www.maths.ed.ac.uk/~v1ranick/papers/burdzies.pdf}{\textit{Knots}} (Second Revides and Extended Edition), de Gruyter, 2003.
\item W.B.R. Lickorish: \href{http://www.redi-bw.de/start/unifr/EBooks-springer/10.1007/978-1-4612-0691-0}{\textit{An Introduction to Knot Theory}}, Springer, 1997.
\item C. Livingston: \href{https://www.math.cuhk.edu.hk/course\_builder/1920/math4900e/Livingston\%20C.---Knot\%20theory\%20(MAA,\%201996).pdf}{\textit{Knot Theory}}. Mathematical Association of America, 1993.
\end{itemize}
\subsubsection*{\large
    Prerequisites:
}
Required: Basic Mathematics courses. \\ Possibly a little  knowledge in topology in addition.
\subsubsection*{\large
    Remarks:
}
Remaining places can be allocated as proseminar places. \\ This course is only offered in German.
\subsubsection*{\large
    Usability and assessments:
}

\begin{tabularx}{\textwidth}{ p{.5\textwidth}
    |X
    |X
}
 &
\makecell[c]{\rotatebox[origin=l]{90}{\parbox{
            4
            cm}{\begin{flushleft}
                Additional module in mathematics (MEd18) (3.0 ECTS) \newline Elective for individual studying (2HfB21) (3.0 ECTS)
            \end{flushleft} }}}
 &
\makecell[c]{\rotatebox[origin=l]{90}{\parbox{
            4
            cm}{\begin{flushleft}
                Undergraduate Seminar (2HfB21, BSc21, MEH21, MEB21) (3.0 ECTS)
            \end{flushleft} }}}
\\
& 1
& 1
\\[2ex] \hline
\hline \rule[0mm]{0cm}{.6cm}PL:  \rule[-3mm]{0cm}{0cm}
 &
 &
\makecell[c]{\xmark}
\\
\hline \rule[0mm]{0cm}{.6cm}SL:  \rule[-3mm]{0cm}{0cm}
 &
\makecell[c]{\xmark}
 &
\makecell[c]{\xmark}
\\
\end{tabularx}


\clearpage\hrule\vskip1pt\hrule
\section*{\Large Machine Learning and Stochastic Analysis}
\addcontentsline{toc}{subsection}{Machine Learning and Stochastic Analysis\ \textcolor{gray}{(\em Thorsten Schmidt)}}
\vskip-2ex
{\it Thorsten Schmidt, Assistant: Moritz Ritter}
\hfill{D/E}\\
Seminar: Fri, 10--12 h, SR 125, \href{https://www.openstreetmap.org/?mlat=48.000637&mlon=7.846006#map=19/48.000636/7.846006}{Ernst-Zermelo-Str. 1}\\
Preregistration \\
Preliminary Meeting 18.10.\\
\subsubsection*{\large
    Content:
}
This seminar will focus on theoretical machine learning results, including modern universal approximation theorems,
approximation of filtering methods through transformes, application of machine learning methods in financial markets
and possibly other related topics. Moreover, we will cover topics in stochastic analysis, like fractional Ito calculus,
uncertainty, filtering and optimal transport. You are also invited to suggest related topics. 

\subsubsection*{\large
    Prerequisites:
}
Required: Basic Probability and either Machine Learning or Probability Theory II (Stochastic Processes).
\subsubsection*{\large
    Remarks:
}
If students are interested and have the required theoretic background, seminars can can also be used as a proseminar. 
\subsubsection*{\large
    Usability and assessments:
}

\begin{tabularx}{\textwidth}{ p{.5\textwidth}
    |X
    |X
}
 &
\makecell[c]{\rotatebox[origin=l]{90}{\parbox{
            4
            cm}{\begin{flushleft}
                Additional module in mathematics (MEd18) (3.0 ECTS) \newline Elective (MSc14) (6.0 ECTS) \newline Elective for individual studying (2HfB21) (6.0 ECTS)
            \end{flushleft} }}}
 &
\makecell[c]{\rotatebox[origin=l]{90}{\parbox{
            4
            cm}{\begin{flushleft}
                Compulsory elective module in mathematics (BSc21) (6.0 ECTS) \newline Elective in Data (MScData24) (6.0 ECTS) \newline Mathematical Seminar (MSc14, BSc21) (6.0 ECTS) \newline Mathematical Seminar (MScData24) (6.0 ECTS)
            \end{flushleft} }}}
\\
& 1
& 1
\\[2ex] \hline
\hline \rule[0mm]{0cm}{.6cm}PL:  \rule[-3mm]{0cm}{0cm}
 &
 &
\makecell[c]{\xmark}
\\
\hline \rule[0mm]{0cm}{.6cm}SL:  \rule[-3mm]{0cm}{0cm}
 &
\makecell[c]{\xmark}
 &
\makecell[c]{\xmark}
\\
\end{tabularx}


\clearpage\hrule\vskip1pt\hrule
\section*{\Large Machine-Learning Methods in the Approximation of PDEs}
\addcontentsline{toc}{subsection}{Machine-Learning Methods in the Approximation of PDEs\ \textcolor{gray}{(\em Sören Bartels)}}
\vskip-2ex
{\it Sören Bartels, Assistant: Tatjana Schreiber}
\hfill{D/E}\\
Seminar \\
Preregistration \\
Preliminary Meeting 08.07., 12: 30 h, Office 209, \href{https://www.openstreetmap.org/?mlat=48.003472&mlon=7.848195#map=19/48.003472/7.848195}{}\\
\subsubsection*{\large
    Content:
}
Machine-learning methods have recently been used to approximate solutions of partial differential equations. While in some cases they lead to advantages over classical approaches, their general superiority is widely open. In the seminar we will review the main concepts and recent developments. 
\subsubsection*{\large
    Literature:
}
\begin{itemize}
\item
B. Bohn, J. Garcke, M. Griebel: \emph{Algorithmic Mathematics in Machine Learning}, SIAM, 2024.
\item
P. C. Petersen: \emph{Neural Network Theory}, Lecture Notes, 2022. 
\end{itemize}
\subsubsection*{\large
    Prerequisites:
}
Introduction to Theory and Numerics for PDEs
\subsubsection*{\large
    Remarks:
}
If students are interested and have the required theoretic background, seminars can can also be used as a proseminar. 
\subsubsection*{\large
    Usability and assessments:
}

\begin{tabularx}{\textwidth}{ p{.5\textwidth}
    |X
    |X
}
 &
\makecell[c]{\rotatebox[origin=l]{90}{\parbox{
            4
            cm}{\begin{flushleft}
                Additional module in mathematics (MEd18) (3.0 ECTS) \newline Elective (MSc14) (6.0 ECTS) \newline Elective for individual studying (2HfB21) (6.0 ECTS)
            \end{flushleft} }}}
 &
\makecell[c]{\rotatebox[origin=l]{90}{\parbox{
            4
            cm}{\begin{flushleft}
                Compulsory elective module in mathematics (BSc21) (6.0 ECTS) \newline Elective in Data (MScData24) (6.0 ECTS) \newline Mathematical Seminar (MSc14, BSc21) (6.0 ECTS) \newline Mathematical Seminar (MScData24) (6.0 ECTS)
            \end{flushleft} }}}
\\
& 1
& 1
\\[2ex] \hline
\hline \rule[0mm]{0cm}{.6cm}PL:  \rule[-3mm]{0cm}{0cm}
 &
 &
\makecell[c]{\xmark}
\\
\hline \rule[0mm]{0cm}{.6cm}SL:  \rule[-3mm]{0cm}{0cm}
 &
\makecell[c]{\xmark}
 &
\makecell[c]{\xmark}
\\
\end{tabularx}


\clearpage\hrule\vskip1pt\hrule
\section*{\Large Medical Data Science}
\addcontentsline{toc}{subsection}{Medical Data Science\ \textcolor{gray}{(\em Harald Binder)}}
\vskip-2ex
{\it Harald Binder}
\hfill{D/E}\\
Seminar: Wed, 10--11: 30 h, HS Medizinische Biometrie, 1. OG, \href{https://www.openstreetmap.org/?mlat=48.002530&mlon=7.846776#map=19/48.002530/7.846776}{Stefan-Meier-Str. 26}\\
Preregistration \\
Preliminary Meeting 17.07., HS Medizinische Biometrie, 1. OG, \href{https://www.openstreetmap.org/?mlat=48.002530&mlon=7.846776#map=19/48.002530/7.846776}{Stefan-Meier-Str. 26}\\
\subsubsection*{\large
    Content:
}
To answer complex biomedical questions from large amounts of data, a wide range of analysis tools is often necessary, e.g. deep learning or general machine learning techniques, which is often summarized under the term ``Medical Data Science''. Statistical approaches play an important rôle as the basis for this. A selection of approaches is to be presented in the seminar lectures that are based on recent original work. The exact thematic orientation is still to be determined.
\subsubsection*{\large
    Literature:
}
Information on introductory literature is given in the preliminary meeting.
\subsubsection*{\large
    Prerequisites:
}
Good knowledge of probability theory and mathematical statistics.
\subsubsection*{\large
    Remarks:
}
The seminar can serve as basis for a bachelor's or master's thesis. \\
The seminar can also be used as a \emph{Proseminar}, but note the previous knowlegde that is demanded.
\subsubsection*{\large
    Usability and assessments:
}

\begin{tabularx}{\textwidth}{ p{.5\textwidth}
    |X
    |X
}
 &
\makecell[c]{\rotatebox[origin=l]{90}{\parbox{
            4
            cm}{\begin{flushleft}
                Additional module in mathematics (MEd18) (3.0 ECTS) \newline Elective (MSc14) (6.0 ECTS) \newline Elective for individual studying (2HfB21) (6.0 ECTS)
            \end{flushleft} }}}
 &
\makecell[c]{\rotatebox[origin=l]{90}{\parbox{
            4
            cm}{\begin{flushleft}
                Compulsory elective module in mathematics (BSc21) (6.0 ECTS) \newline Elective in Data (MScData24) (6.0 ECTS) \newline Mathematical Seminar (MSc14, BSc21) (6.0 ECTS) \newline Mathematical Seminar (MScData24) (6.0 ECTS)
            \end{flushleft} }}}
\\
& 1
& 1
\\[2ex] \hline
\hline \rule[0mm]{0cm}{.6cm}PL:  \rule[-3mm]{0cm}{0cm}
 &
 &
\makecell[c]{\xmark}
\\
\hline \rule[0mm]{0cm}{.6cm}SL:  \rule[-3mm]{0cm}{0cm}
 &
\makecell[c]{\xmark}
 &
\makecell[c]{\xmark}
\\
\end{tabularx}


\clearpage\hrule\vskip1pt\hrule
\section*{\Large \href{https://home.mathematik.uni-freiburg.de/analysis/2024_WiSe_Lehre/2024_WiSe_Seminar_MinimalSurfaces/}{Minimal Surfaces}}
\addcontentsline{toc}{subsection}{Minimal Surfaces\ \textcolor{gray}{(\em Guofang Wang)}}
\vskip-2ex
{\it Guofang Wang, Assistant: Xuwen Zhang}
\hfill{D/E}\\
Seminar: Wed, 16--18 h, SR 125, \href{https://www.openstreetmap.org/?mlat=48.000637&mlon=7.846006#map=19/48.000636/7.846006}{Ernst-Zermelo-Str. 1}\\
Preliminary Meeting 17.07., 16 h\\
% Webseite: \url{https://home.mathematik.uni-freiburg.de/analysis/2024_WiSe_Lehre/2024_WiSe_Seminar_MinimalSurfaces/}
\subsubsection*{\large
    Content:
}
Minimal surfaces are surfaces in space with a “minimal” area and can be described using holomorphic functions. They occur, for example in the investigation of soap skins and the construction of stable objects (e.g. in architecture). In the investigation of minimal surfaces elegant methods from various mathematical fields such as function theory, calculus of variations, differential geometry and partial differential equations. are applied.
\subsubsection*{\large
    Literature:
}
\begin{itemize}
\item
R. Osserman: \emph{A survey of minimal surfaces}, Van Nostrand 1969. 
\item
J.-H. Eschenburg, J. Jost: \emph{Differentialgeometrie und Minimalflächen}, Springer 2007.
\item
E. Kuwert: \emph{Einführung in die Theorie der Minimalflächen}, Skript 1998.
\item
W. H. Meeks III, J. Pérez: \emph{A survey on classical minimal surface theory}.
\item
T. Colding, W. P. Minicozzi: \emph{Minimal Surfaces}, New York University 1999.
\end{itemize}
\subsubsection*{\large
    Prerequisites:
}
Required: Analysis III or knowledge about multidimensional integration and complex analysis. \\ Recommended: Elementary knowledge about differential geometry.
\subsubsection*{\large
    Remarks:
}
If students are interested and have the required theoretic background, seminars can can also be used as a proseminar. 

\subsubsection*{\large
    Usability and assessments:
}

\begin{tabularx}{\textwidth}{ p{.5\textwidth}
    |X
    |X
}
 &
\makecell[c]{\rotatebox[origin=l]{90}{\parbox{
            4
            cm}{\begin{flushleft}
                Additional module in mathematics (MEd18) (3.0 ECTS) \newline Elective (MSc14) (6.0 ECTS) \newline Elective (MScData24) (6.0 ECTS) \newline Elective for individual studying (2HfB21) (6.0 ECTS)
            \end{flushleft} }}}
 &
\makecell[c]{\rotatebox[origin=l]{90}{\parbox{
            4
            cm}{\begin{flushleft}
                Compulsory elective module in mathematics (BSc21) (6.0 ECTS) \newline Mathematical Seminar (MSc14, BSc21) (6.0 ECTS)
            \end{flushleft} }}}
\\
& 1
& 1
\\[2ex] \hline
\hline \rule[0mm]{0cm}{.6cm}PL:  \rule[-3mm]{0cm}{0cm}
 &
 &
\makecell[c]{\xmark}
\\
\hline \rule[0mm]{0cm}{.6cm}SL:  \rule[-3mm]{0cm}{0cm}
 &
\makecell[c]{\xmark}
 &
\makecell[c]{\xmark}
\\
\end{tabularx}


\clearpage\hrule\vskip1pt\hrule
\section*{\Large \href{ https://home.mathematik.uni-freiburg.de/geometrie/lehre/ws2024/Seminar/}{Seminar on Algebraic Topology}}
\addcontentsline{toc}{subsection}{Seminar on Algebraic Topology\ \textcolor{gray}{(\em Sebastian Goette)}}
\vskip-2ex
{\it Sebastian Goette, Assistant: Mikhael Tëmkin}
\hfill{D/E}\\
Seminar: Tue, 14--16 h, SR 125, \href{https://www.openstreetmap.org/?mlat=48.000637&mlon=7.846006#map=19/48.000636/7.846006}{Ernst-Zermelo-Str. 1}\\
Preliminary Meeting 16.07., SR 125, \href{https://www.openstreetmap.org/?mlat=48.000637&mlon=7.846006#map=19/48.000636/7.846006}{Ernst-Zermelo-Str. 1}\\
% Webseite: \url{ https://home.mathematik.uni-freiburg.de/geometrie/lehre/ws2024/Seminar/}
\subsubsection*{\large
    Content:
}
We will discuss advanced topics in algebraic topology.
Depending on the interest of the participants we could work on one of the following topics---if you have other topic suggestions, please contact the lecturer.
\begin{itemize}
\item The Steenrod algebra. An additional structure on the cohomology modulo $p$
allows finer statements on the existence of continuous mappings, such as the existence of of linearly independent vector fields on spheres. The Wu formulas provide a connection to characteristic classes of manifolds.
\item Structured spectra. In order to represent multiplicative (co-)homology functors
by spectra, one needs a closed monoidal category of spectra, for example a category of spectra, for example
symmetric or orthogonal spectra. In this context
we also get to know model structures better.
\item $K$-theory and index theory. Elliptic differential operators on compact manifolds are manifolds are Fredholm operators. Their index can be defined by the theorem of Atiyah--Singer topologically. We prove this theorem using (mainly) topological methods and give some geometric applications.
\end{itemize} 
\subsubsection*{\large
    Prerequisites:
}
Algebraic Topology~I and II
\subsubsection*{\large
    Remarks:
}
Participants take on one or, if interested, several presentations. For the rest of the time, we continue the event as a reading course or lecture. \\ If students are interested and have the required theoretic background, seminars can also be used as a proseminar. \\ This course can be offered in English.
\subsubsection*{\large
    Usability and assessments:
}

\begin{tabularx}{\textwidth}{ p{.5\textwidth}
    |X
    |X
}
 &
\makecell[c]{\rotatebox[origin=l]{90}{\parbox{
            4
            cm}{\begin{flushleft}
                Additional module in mathematics (MEd18) (3.0 ECTS) \newline Elective (MSc14) (6.0 ECTS) \newline Elective (MScData24) (6.0 ECTS) \newline Elective for individual studying (2HfB21) (6.0 ECTS)
            \end{flushleft} }}}
 &
\makecell[c]{\rotatebox[origin=l]{90}{\parbox{
            4
            cm}{\begin{flushleft}
                Compulsory elective module in mathematics (BSc21) (6.0 ECTS) \newline Mathematical Seminar (MSc14, BSc21) (6.0 ECTS)
            \end{flushleft} }}}
\\
& 1
& 1
\\[2ex] \hline
\hline \rule[0mm]{0cm}{.6cm}PL:  \rule[-3mm]{0cm}{0cm}
 &
 &
\makecell[c]{\xmark}
\\
\hline \rule[0mm]{0cm}{.6cm}SL:  \rule[-3mm]{0cm}{0cm}
 &
\makecell[c]{\xmark}
 &
\makecell[c]{\xmark}
\\
\end{tabularx}


\clearpage\hrule\vskip1pt\hrule
\section*{\Large \href{http://home.mathematik.uni-freiburg.de/arithgeom/lehre.html}{Theory of Non-Commutative Algebras}}
\addcontentsline{toc}{subsection}{Theory of Non-Commutative Algebras\ \textcolor{gray}{(\em Annette Huber-Klawitter)}}
\vskip-2ex
{\it Annette Huber-Klawitter, Assistant: Xier Ren}
\hfill{D/E}\\
Seminar: Fri, 8--10 h, SR 404, \href{https://www.openstreetmap.org/?mlat=48.000637&mlon=7.846006#map=19/48.000636/7.846006}{Ernst-Zermelo-Str. 1}\\
Preregistration \\
Preliminary Meeting 15.07., 11 h, SR 318, \href{https://www.openstreetmap.org/?mlat=48.000637&mlon=7.846006#map=19/48.000636/7.846006}{Ernst-Zermelo-Str. 1}\\
% Webseite: \url{http://home.mathematik.uni-freiburg.de/arithgeom/lehre.html}
\subsubsection*{\large
    Content:
}
In this seminar, we are going to study finite dimensional (unital, possibly non-commutative) algebras over a (commutative) field $k$. Prototypes are the rings of square matrices over $k$, finite field extensions, or the algebra $k^n$ with diagonal multiplication. 

We will concentrate on path algebras of finite quivers (German: Köcher). Modules over them are equivalently described as representations of the quiver. Many algebraic properties can be directly understood from properties of the quiver. 
\subsubsection*{\large
    Literature:
}
\begin{itemize}
\item
Frank Anderson, Kent Fuller: \emph{Rings and Categories of Modules}, GTM 13, Springer, 1992 
\item
Ralf Schiffler: \emph{Quiver Representations}, CMS Books in Mathematics, Springer, 2014 
\item
Alexander Kirillov Jr.: \emph{Quiver Representations}, GSM 174, AMS, 2016
\end{itemize}
\subsubsection*{\large
    Prerequisites:
}
Required: Linear Algebra \\ Recommended: Algebra and Number Theory, Commutative Algebra and Introduction to Algebraic Geometry
\subsubsection*{\large
    Remarks:
}
Communication with the assistent will be in English. Talks can be given in German or English.\\
If students are interested and have the required theoretic background, seminars can also be used as a proseminar. \\
\subsubsection*{\large
    Usability and assessments:
}

\begin{tabularx}{\textwidth}{ p{.5\textwidth}
    |X
    |X
}
 &
\makecell[c]{\rotatebox[origin=l]{90}{\parbox{
            4
            cm}{\begin{flushleft}
                Additional module in mathematics (MEd18) (3.0 ECTS) \newline Elective (MSc14) (6.0 ECTS) \newline Elective (MScData24) (6.0 ECTS) \newline Elective for individual studying (2HfB21) (6.0 ECTS)
            \end{flushleft} }}}
 &
\makecell[c]{\rotatebox[origin=l]{90}{\parbox{
            4
            cm}{\begin{flushleft}
                Compulsory elective module in mathematics (BSc21) (6.0 ECTS) \newline Mathematical Seminar (MSc14, BSc21) (6.0 ECTS)
            \end{flushleft} }}}
\\
& 1
& 1
\\[2ex] \hline
\hline \rule[0mm]{0cm}{.6cm}PL:  \rule[-3mm]{0cm}{0cm}
 &
 &
\makecell[c]{\xmark}
\\
\hline \rule[0mm]{0cm}{.6cm}SL:  \rule[-3mm]{0cm}{0cm}
 &
\makecell[c]{\xmark}
 &
\makecell[c]{\xmark}
\\
\end{tabularx}


\end{document}